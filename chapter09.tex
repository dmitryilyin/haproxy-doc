\chapter{Statistics and monitoring}

It is possible to query HAProxy about its status. The most commonly used
mechanism is the HTTP statistics page. This page also exposes an alternative
CSV output format for monitoring tools. The same format is provided on the
Unix socket.

\section{CSV format}

The statistics may be consulted either from the unix socket or from the HTTP
page. Both means provide a CSV format whose fields follow.

\begin{verbatim}
  0. pxname: proxy name
  1. svname: service name (FRONTEND for frontend, BACKEND for backend, any name
    for server)
  2. qcur: current queued requests
  3. qmax: max queued requests
  4. scur: current sessions
  5. smax: max sessions
  6. slim: sessions limit
  7. stot: total sessions
  8. bin: bytes in
  9. bout: bytes out
 10. dreq: denied requests
 11. dresp: denied responses
 12. ereq: request errors
 13. econ: connection errors
 14. eresp: response errors (among which srv_abrt)
 15. wretr: retries (warning)
 16. wredis: redispatches (warning)
 17. status: status (UP/DOWN/NOLB/MAINT/MAINT(via)...)
 18. weight: server weight (server), total weight (backend)
 19. act: server is active (server), number of active servers (backend)
 20. bck: server is backup (server), number of backup servers (backend)
 21. chkfail: number of failed checks
 22. chkdown: number of UP->DOWN transitions
 23. lastchg: last status change (in seconds)
 24. downtime: total downtime (in seconds)
 25. qlimit: queue limit
 26. pid: process id (0 for first instance, 1 for second, ...)
 27. iid: unique proxy id
 28. sid: service id (unique inside a proxy)
 29. throttle: warm up status
 30. lbtot: total number of times a server was selected
 31. tracked: id of proxy/server if tracking is enabled
 32. type (0=frontend, 1=backend, 2=server, 3=socket)
 33. rate: number of sessions per second over last elapsed second
 34. rate_lim: limit on new sessions per second
 35. rate_max: max number of new sessions per second
 36. check_status: status of last health check, one of:
        UNK     -> unknown
        INI     -> initializing
        SOCKERR -> socket error
        L4OK    -> check passed on layer 4, no upper layers testing enabled
        L4TMOUT -> layer 1-4 timeout
        L4CON   -> layer 1-4 connection problem, for example
                   "Connection refused" (tcp rst) or "No route to host" (icmp)
        L6OK    -> check passed on layer 6
        L6TOUT  -> layer 6 (SSL) timeout
        L6RSP   -> layer 6 invalid response - protocol error
        L7OK    -> check passed on layer 7
        L7OKC   -> check conditionally passed on layer 7, for example 404 with
                   disable-on-404
        L7TOUT  -> layer 7 (HTTP/SMTP) timeout
        L7RSP   -> layer 7 invalid response - protocol error
        L7STS   -> layer 7 response error, for example HTTP 5xx
 37. check_code: layer5-7 code, if available
 38. check_duration: time in ms took to finish last health check
 39. hrsp_1xx: http responses with 1xx code
 40. hrsp_2xx: http responses with 2xx code
 41. hrsp_3xx: http responses with 3xx code
 42. hrsp_4xx: http responses with 4xx code
 43. hrsp_5xx: http responses with 5xx code
 44. hrsp_other: http responses with other codes (protocol error)
 45. hanafail: failed health checks details
 46. req_rate: HTTP requests per second over last elapsed second
 47. req_rate_max: max number of HTTP requests per second observed
 48. req_tot: total number of HTTP requests received
 49. cli_abrt: number of data transfers aborted by the client
 50. srv_abrt: number of data transfers aborted by the server (inc. in eresp)
\end{verbatim}

\section{Unix Socket commands}

The following commands are supported on the UNIX stats socket ; all of them
must be terminated by a line feed. The socket supports pipelining, so that it
is possible to chain multiple commands at once provided they are delimited by
a semi-colon or a line feed, although the former is more reliable as it has no
risk of being truncated over the network. The responses themselves will each be
followed by an empty line, so it will be easy for an external script to match a
given response with a given request. By default one command line is processed
then the connection closes, but there is an interactive allowing multiple lines
to be issued one at a time.

It is important to understand that when multiple haproxy processes are started
on the same sockets, any process may pick up the request and will output its
own stats.

\subsubsection[clear counters]{clear counters}
  Clear the max values of the statistics counters in each proxy (frontend \&
  backend) and in each server. The cumulated counters are not affected. This
  can be used to get clean counters after an incident, without having to
  restart nor to clear traffic counters. This command is restricted and can
  only be issued on sockets configured for levels "operator" or "admin".

\subsubsection[clear counters all]{clear counters all}
  Clear all statistics counters in each proxy (frontend \& backend) and in each
  server. This has the same effect as restarting. This command is restricted
  and can only be issued on sockets configured for level "admin".

\subsubsection[clear table]{clear table <table> [ data.<type> <operator> <value> ] | [ key <key> ]}
  Remove entries from the stick-table <table>.

  This is typically used to unblock some users complaining they have been
  abusively denied access to a service, but this can also be used to clear some
  stickiness entries matching a server that is going to be replaced (see "show
  table" below for details).  Note that sometimes, removal of an entry will be
  refused because it is currently tracked by a session. Retrying a few seconds
  later after the session ends is usual enough.

  In the case where no options arguments are given all entries will be removed.

  When the "data." form is used entries matching a filter applied using the
  stored data (see "stick-table" in section 4.2) are removed.  A stored data
  type must be specified in <type>, and this data type must be stored in the
  table otherwise an error is reported. The data is compared according to
  <operator> with the 64-bit integer <value>.  Operators are the same as with
  the ACLs:
  
  \begin{description}
  \item[eq] match entries whose data is equal to this value
  \item[ne] match entries whose data is not equal to this value
  \item[le] match entries whose data is less than or equal to this value
  \item[ge] match entries whose data is greater than or equal to this value
  \item[lt] match entries whose data is less than this value
  \item[gt] match entries whose data is greater than this value
  \end{description}

  When the key form is used the entry <key> is removed.  The key must be of the
  same type as the table, which currently is limited to IPv4, IPv6, integer and
  string.

  Example:
  \begin{verbatim}
        $ echo "show table http_proxy" | socat stdio /tmp/sock1
    >>> # table: http_proxy, type: ip, size:204800, used:2
    >>> 0x80e6a4c: key=127.0.0.1 use=0 exp=3594729 gpc0=0 conn_rate(30000)=1 \
          bytes_out_rate(60000)=187
    >>> 0x80e6a80: key=127.0.0.2 use=0 exp=3594740 gpc0=1 conn_rate(30000)=10 \
          bytes_out_rate(60000)=191

        $ echo "clear table http_proxy key 127.0.0.1" | socat stdio /tmp/sock1

        $ echo "show table http_proxy" | socat stdio /tmp/sock1
    >>> # table: http_proxy, type: ip, size:204800, used:1
    >>> 0x80e6a80: key=127.0.0.2 use=0 exp=3594740 gpc0=1 conn_rate(30000)=10 \
          bytes_out_rate(60000)=191
        $ echo "clear table http_proxy data.gpc0 eq 1" | socat stdio /tmp/sock1
        $ echo "show table http_proxy" | socat stdio /tmp/sock1
    >>> # table: http_proxy, type: ip, size:204800, used:1
 \end{verbatim}

\subsubsection[disable frontend]{disable frontend <frontend>}
  Mark the frontend as temporarily stopped. This corresponds to the mode which
  is used during a soft restart : the frontend releases the port but can be
  enabled again if needed. This should be used with care as some non-Linux OSes
  are unable to enable it back. This is intended to be used in environments
  where stopping a proxy is not even imaginable but a misconfigured proxy must
  be fixed. That way it's possible to release the port and bind it into another
  process to restore operations. The frontend will appear with status "STOP"
  on the stats page.

  The frontend may be specified either by its name or by its numeric ID,
  prefixed with a sharp ('\#').

  This command is restricted and can only be issued on sockets configured for
  level "admin".

\subsubsection[disable server]{disable server <backend>/<server>}
  Mark the server DOWN for maintenance. In this mode, no more checks will be
  performed on the server until it leaves maintenance.
  If the server is tracked by other servers, those servers will be set to DOWN
  during the maintenance.

  In the statistics page, a server DOWN for maintenance will appear with a
  "MAINT" status, its tracking servers with the "MAINT(via)" one.

  Both the backend and the server may be specified either by their name or by
  their numeric ID, prefixed with a sharp ('\#').

  This command is restricted and can only be issued on sockets configured for
  level "admin".

\subsubsection[enable frontend]{enable frontend <frontend>}
  Resume a frontend which was temporarily stopped. It is possible that some of
  the listening ports won't be able to bind anymore (eg: if another process
  took them since the 'disable frontend' operation). If this happens, an error
  is displayed. Some operating systems might not be able to resume a frontend
  which was disabled.

  The frontend may be specified either by its name or by its numeric ID,
  prefixed with a sharp ('\#').

  This command is restricted and can only be issued on sockets configured for
  level "admin".

\subsubsection[enable server]{enable server <backend>/<server>}
  If the server was previously marked as DOWN for maintenance, this marks the
  server UP and checks are re-enabled.

  Both the backend and the server may be specified either by their name or by
  their numeric ID, prefixed with a sharp ('\#').

  This command is restricted and can only be issued on sockets configured for
  level "admin".

\subsubsection[get weight]{get weight <backend>/<server>}
  Report the current weight and the initial weight of server <server> in
  backend <backend> or an error if either doesn't exist. The initial weight is
  the one that appears in the configuration file. Both are normally equal
  unless the current weight has been changed. Both the backend and the server
  may be specified either by their name or by their numeric ID, prefixed with a
  sharp ('\#').

\subsubsection[help]{help}
  Print the list of known keywords and their basic usage. The same help screen
  is also displayed for unknown commands.

\subsubsection[prompt]{prompt}
  Toggle the prompt at the beginning of the line and enter or leave interactive
  mode. In interactive mode, the connection is not closed after a command
  completes. Instead, the prompt will appear again, indicating the user that
  the interpreter is waiting for a new command. The prompt consists in a right
  angle bracket followed by a space "> ". This mode is particularly convenient
  when one wants to periodically check information such as stats or errors.
  It is also a good idea to enter interactive mode before issuing a "help"
  command.

\subsubsection[quit]{quit}
  Close the connection when in interactive mode.

\subsubsection[set maxconn frontend]{set maxconn frontend <frontend> <value>}
  Dynamically change the specified frontend's maxconn setting. Any non-null
  positive value is allowed, but setting values larger than the global maxconn
  does not make much sense. If the limit is increased and connections were
  pending, they will immediately be accepted. If it is lowered to a value below
  the current number of connections, new connections acceptation will be
  delayed until the threshold is reached. The frontend might be specified by
  either its name or its numeric ID prefixed with a sharp ('\#').

\subsubsection[set maxconn global]{set maxconn global <maxconn>}
  Dynamically change the global maxconn setting within the range defined by the
  initial global maxconn setting. If it is increased and connections were
  pending, they will immediately be accepted. If it is lowered to a value below
  the current number of connections, new connections acceptation will be
  delayed until the threshold is reached. A value of zero restores the initial
  setting.

\subsubsection[set rate-limit connections global]{set rate-limit connections global <value>}
  Change the process-wide connection rate limit, which is set by the global
  'maxconnrate' setting. A value of zero disables the limitation. This limit
  applies to all frontends and the change has an immediate effect. The value
  is passed in number of connections per second.

\subsubsection[set timeout cli]{set timeout cli <delay>}
  Change the CLI interface timeout for current connection. This can be useful
  during long debugging sessions where the user needs to constantly inspect
  some indicators without being disconnected. The delay is passed in seconds.

\subsubsection[set weight]{set weight <backend>/<server> <weight>[\%]}
  Change a server's weight to the value passed in argument. If the value ends
  with the '\%' sign, then the new weight will be relative to the initially
  configured weight. Relative weights are only permitted between 0 and 100\%,
  and absolute weights are permitted between 0 and 256. Servers which are part
  of a farm running a static load-balancing algorithm have stricter limitations
  because the weight cannot change once set. Thus for these servers, the only
  accepted values are 0 and 100\% (or 0 and the initial weight). Changes take
  effect immediately, though certain LB algorithms require a certain amount of
  requests to consider changes. A typical usage of this command is to disable
  a server during an update by setting its weight to zero, then to enable it
  again after the update by setting it back to 100\%. This command is restricted
  and can only be issued on sockets configured for level "admin". Both the
  backend and the server may be specified either by their name or by their
  numeric ID, prefixed with a sharp ('\#').

\subsubsection[show errors]{show errors [<iid>]}
  Dump last known request and response errors collected by frontends and
  backends. If <iid> is specified, the limit the dump to errors concerning
  either frontend or backend whose ID is <iid>. This command is restricted
  and can only be issued on sockets configured for levels "operator" or
  "admin".

  The errors which may be collected are the last request and response errors
  caused by protocol violations, often due to invalid characters in header
  names. The report precisely indicates what exact character violated the
  protocol. Other important information such as the exact date the error was
  detected, frontend and backend names, the server name (when known), the
  internal session ID and the source address which has initiated the session
  are reported too.

  All characters are returned, and non-printable characters are encoded. The
  most common ones (\textbackslash t = 9, \textbackslash n = 10, \textbackslash
  r = 13 and \textbackslash e = 27) are encoded as one
  letter following a backslash. The backslash itself is encoded as
  '\textbackslash \textbackslash' to
  avoid confusion. Other non-printable characters are encoded '\textbackslash xNN' where
  NN is the two-digits hexadecimal representation of the character's ASCII
  code.

  Lines are prefixed with the position of their first character, starting at 0
  for the beginning of the buffer. At most one input line is printed per line,
  and large lines will be broken into multiple consecutive output lines so that
  the output never goes beyond 79 characters wide. It is easy to detect if a
  line was broken, because it will not end with '\textbackslash n' and the next line's offset
  will be followed by a '+' sign, indicating it is a continuation of previous
  line.

\begin{samepage}
  Example:
  \begin{verbatim}
        $ echo "show errors" | socat stdio /tmp/sock1
    >>> [04/Mar/2009:15:46:56.081] backend http-in (#2) : invalid response
          src 127.0.0.1, session #54, frontend fe-eth0 (#1), server s2 (#1)
          response length 213 bytes, error at position 23:

          00000  HTTP/1.0 200 OK\r\n
          00017  header/bizarre:blah\r\n
          00038  Location: blah\r\n
          00054  Long-line: this is a very long line which should b
          00104+ e broken into multiple lines on the output buffer,
          00154+  otherwise it would be too large to print in a ter
          00204+ minal\r\n
          00211  \r\n
  \end{verbatim}
\end{samepage}

    In the example above, we see that the backend "http-in" which has internal
    ID 2 has blocked an invalid response from its server s2 which has internal
    ID 1. The request was on session 54 initiated by source 127.0.0.1 and
    received by frontend fe-eth0 whose ID is 1. The total response length was
    213 bytes when the error was detected, and the error was at byte 23. This
    is the slash ('/') in header name "header/bizarre", which is not a valid
    HTTP character for a header name.

\subsubsection[show info]{show info}
  Dump info about haproxy status on current process.

\subsubsection[show sess]{show sess}
  Dump all known sessions. Avoid doing this on slow connections as this can
  be huge. This command is restricted and can only be issued on sockets
  configured for levels "operator" or "admin".

\subsubsection[show sess]{show sess <id>}
  Display a lot of internal information about the specified session identifier.
  This identifier is the first field at the beginning of the lines in the dumps
  of "show sess" (it corresponds to the session pointer). Those information are
  useless to most users but may be used by haproxy developers to troubleshoot a
  complex bug. The output format is intentionally not documented so that it can
  freely evolve depending on demands.

\subsubsection[show stat]{show stat [<iid> <type> <sid>]}
  Dump statistics in the CSV format. By passing <id>, <type> and <sid>, it is
  possible to dump only selected items:
  \begin{itemize}
  \item[-] <iid> is a proxy ID, -1 to dump everything
  \item[-] <type> selects the type of dumpable objects : 1 for frontends, 2 for
       backends, 4 for servers, -1 for everything. These values can be ORed,
       for example:
       \begin{verbatim}
          1 + 2     = 3   -> frontend + backend.
          1 + 2 + 4 = 7   -> frontend + backend + server.
       \end{verbatim}
  \item[-] <sid> is a server ID, -1 to dump everything from the selected proxy.
  \end{itemize}

\begin{samepage}
  Example:
  \begin{verbatim}
        $ echo "show info;show stat" | socat stdio unix-connect:/tmp/sock1
    >>> Name: HAProxy
        Version: 1.4-dev2-49
        Release_date: 2009/09/23
        Nbproc: 1
        Process_num: 1
        (...)

        # pxname,svname,qcur,qmax,scur,smax,slim,stot,bin,bout,dreq,  (...)
        stats,FRONTEND,,,0,0,1000,0,0,0,0,0,0,,,,,OPEN,,,,,,,,,1,1,0, (...)
        stats,BACKEND,0,0,0,0,1000,0,0,0,0,0,,0,0,0,0,UP,0,0,0,,0,250,(...)
        (...)
        www1,BACKEND,0,0,0,0,1000,0,0,0,0,0,,0,0,0,0,UP,1,1,0,,0,250, (...)
    \end{verbatim}
\end{samepage}

    Here, two commands have been issued at once. That way it's easy to find
    which process the stats apply to in multi-process mode. Notice the empty
    line after the information output which marks the end of the first block.
    A similar empty line appears at the end of the second block (stats) so that
    the reader knows the output has not been truncated.

\subsubsection[show table]{show table}
  Dump general information on all known stick-tables. Their name is returned
  (the name of the proxy which holds them), their type (currently zero, always
  IP), their size in maximum possible number of entries, and the number of
  entries currently in use.

  Example :
  \begin{verbatim}
        $ echo "show table" | socat stdio /tmp/sock1
    >>> # table: front_pub, type: ip, size:204800, used:171454
    >>> # table: back_rdp, type: ip, size:204800, used:0
  \end{verbatim}

\subsubsection[show table]{show table <name> [ data.<type> <operator> <value> ] | [ key <key> ]}
  Dump contents of stick-table <name>. In this mode, a first line of generic
  information about the table is reported as with "show table", then all
  entries are dumped. Since this can be quite heavy, it is possible to specify
  a filter in order to specify what entries to display.

  When the "data." form is used the filter applies to the stored data (see
  "stick-table" in section 4.2).  A stored data type must be specified
  in <type>, and this data type must be stored in the table otherwise an
  error is reported. The data is compared according to <operator> with the
  64-bit integer <value>.  Operators are the same as with the ACLs:

  \begin{description}
  \item[eq] match entries whose data is equal to this value
  \item[ne] match entries whose data is not equal to this value
  \item[le] match entries whose data is less than or equal to this value
  \item[ge] match entries whose data is greater than or equal to this value
  \item[lt] match entries whose data is less than this value
  \item[gt] match entries whose data is greater than this value
  \end{description}

  When the key form is used the entry <key> is shown.  The key must be of the
  same type as the table, which currently is limited to IPv4, IPv6, integer,
  and string.

  Example:
  \begin{verbatim}
        $ echo "show table http_proxy" | socat stdio /tmp/sock1
    >>> # table: http_proxy, type: ip, size:204800, used:2
    >>> 0x80e6a4c: key=127.0.0.1 use=0 exp=3594729 gpc0=0 conn_rate(30000)=1  \
          bytes_out_rate(60000)=187
    >>> 0x80e6a80: key=127.0.0.2 use=0 exp=3594740 gpc0=1 conn_rate(30000)=10 \
          bytes_out_rate(60000)=191

        $ echo "show table http_proxy data.gpc0 gt 0" | socat stdio /tmp/sock1
    >>> # table: http_proxy, type: ip, size:204800, used:2
    >>> 0x80e6a80: key=127.0.0.2 use=0 exp=3594740 gpc0=1 conn_rate(30000)=10 \
          bytes_out_rate(60000)=191

        $ echo "show table http_proxy data.conn_rate gt 5" | \
            socat stdio /tmp/sock1
    >>> # table: http_proxy, type: ip, size:204800, used:2
    >>> 0x80e6a80: key=127.0.0.2 use=0 exp=3594740 gpc0=1 conn_rate(30000)=10 \
          bytes_out_rate(60000)=191

        $ echo "show table http_proxy key 127.0.0.2" | \
            socat stdio /tmp/sock1
    >>> # table: http_proxy, type: ip, size:204800, used:2
    >>> 0x80e6a80: key=127.0.0.2 use=0 exp=3594740 gpc0=1 conn_rate(30000)=10 \
          bytes_out_rate(60000)=191
  \end{verbatim}

  When the data criterion applies to a dynamic value dependent on time such as
  a bytes rate, the value is dynamically computed during the evaluation of the
  entry in order to decide whether it has to be dumped or not. This means that
  such a filter could match for some time then not match anymore because as
  time goes, the average event rate drops.

  It is possible to use this to extract lists of IP addresses abusing the
  service, in order to monitor them or even blacklist them in a firewall.

\begin{samepage}
  Example:
  \begin{verbatim}
        $ echo "show table http_proxy data.gpc0 gt 0" \
          | socat stdio /tmp/sock1 \
          | fgrep 'key=' | cut -d' ' -f2 | cut -d= -f2 > abusers-ip.txt
          ( or | awk '/key/{ print a[split($2,a,"=")]; }' )
  \end{verbatim}
\end{samepage}

\subsubsection[shutdown frontend]{shutdown frontend <frontend>}
  Completely delete the specified frontend. All the ports it was bound to will
  be released. It will not be possible to enable the frontend anymore after
  this operation. This is intended to be used in environments where stopping a
  proxy is not even imaginable but a misconfigured proxy must be fixed. That
  way it's possible to release the port and bind it into another process to
  restore operations. The frontend will not appear at all on the stats page
  once it is terminated.

  The frontend may be specified either by its name or by its numeric ID,
  prefixed with a sharp ('\#').

  This command is restricted and can only be issued on sockets configured for
  level "admin".

\subsubsection[shutdown session]{shutdown session <id>}
  Immediately terminate the session matching the specified session identifier.
  This identifier is the first field at the beginning of the lines in the dumps
  of "show sess" (it corresponds to the session pointer). This can be used to
  terminate a long-running session without waiting for a timeout or when an
  endless transfer is ongoing. Such terminated sessions are reported with a 'K'
  flag in the logs.

\subsubsection[shutdown sessions]{shutdown sessions <backend>/<server>}
  Immediately terminate all the sessions attached to the specified server. This
  can be used to terminate long-running sessions after a server is put into
  maintenance mode, for instance. Such terminated sessions are reported with a
  'K' flag in the logs.