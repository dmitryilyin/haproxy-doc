\chapter{Global parameters}
\label{chap:global_parameters}

Parameters in the \cfgsec{global} section are process-wide and often OS-specific. They
are generally set once for all and do not need being changed once correct. Some
of them have command-line equivalents.

The following keywords are supported in the \cfgsec{global} section:

\begin{multicols}{2}
\paragraph*{Process management and security}
  \begin{itemize}
    \item[-] \kwl{chroot}
    \item[-] \kwl{daemon}
    \item[-] \kwl{gid}
    \item[-] \kwl{group}
    \item[-] \kwl{log}
    \item[-] \kwl{log-send-hostname}
    \item[-] \kwl{nbproc}
    \item[-] \kwl{pidfile}
    \item[-] \kwl{uid}
    \item[-] \kwl{ulimit-n}
    \item[-] \kwl{user}
    \item[-] \kwl{stats}
    \item[-] \kwl{node}
    \item[-] \kwl{description}
    \item[-] \kwl{unix-bind}
  \end{itemize}
\paragraph*{Performance tuning}
  \begin{itemize}
    \item[-] \kwl{maxconn}
    \item[-] \kwl{maxconnrate}
    \item[-] \kwl{maxpipes}
    \item[-] \kwl{noepoll}
    \item[-] \kwl{nokqueue}
    \item[-] \kwl{nopoll}
    \item[-] \kwl{nosepoll}
    \item[-] \kwl{nosplice}
    \item[-] \kwl{spread-checks}
    \item[-] \kwl{tune.bufsize}
    \item[-] \kwl{tune.chksize}
    \item[-] \kwl{tune.http.maxhdr}
    \item[-] \kwl{tune.maxaccept}
    \item[-] \kwl{tune.maxpollevents}
    \item[-] \kwl{tune.maxrewrite}
    \item[-] \kwl{tune.pipesize}
    \item[-] \kwl{tune.rcvbuf.client}
    \item[-] \kwl{tune.rcvbuf.server}
    \item[-] \kwl{tune.sndbuf.client}
    \item[-] \kwl{tune.sndbuf.server}
  \end{itemize}
\paragraph*{Debugging}
  \begin{itemize}
    \item[-] \kwl{debug}
    \item[-] \kwl{quiet}
  \end{itemize}
\end{multicols}

\section{Process management and security}
\label{sec:process_management}

\subsubsection[chroot]{\kw{chroot} <jail\_dir>}
\index{chroot}

Changes current directory to "jail\_dir" and performs a chroot() there before
dropping privileges. This increases the security level in case an unknown
vulnerability would be exploited, since it would make it very hard for the
attacker to exploit the system. This only works when the process is started
with superuser privileges. It is important to ensure that "jail\_dir" is both
empty and unwritable to anyone.
  
\subsubsection[daemon]{\kw{daemon}}
\index{daemon}

Makes the process fork into background. This is the recommended mode of
operation. It is equivalent to the command line \cmd{-D} argument. It can be
disabled by the command line \cmd{-db} argument.
  
\subsubsection[gid]{\kw{gid} <number>}
\index{gid}

Changes the process' group ID to "number". It is recommended that the group
ID is dedicated to HAProxy or to a small set of similar daemons. HAProxy must
be started with a user belonging to this group, or with superuser privileges.
See also \kwl{group} and \kwl{uid}.
  
\subsubsection[group]{\kw{group} <group\_name>}
\index{group}

Similar to \kwl{gid} but uses the GID of group name "group\_name" from \cmd{/etc/group}.
See also \kwl{gid} and \kwl{user}.

\subsubsection[log]{\kw{log} <address> <facility> [max level [min level]]}
\index{log}

Adds a global syslog server. Up to two global servers can be defined. They
will receive logs for startups and exits, as well as all logs from proxies
configured with \kwl{log global}.

"address" can be one of:

\begin{itemize}
  \item[-] An IPv4 address optionally followed by a colon and a UDP port. If
        no port is specified, 514 is used by default (the standard syslog
        port).

  \item[-] An IPv6 address followed by a colon and optionally a UDP port. If
        no port is specified, 514 is used by default (the standard syslog
        port).

  \item[-] A filesystem path to a UNIX domain socket, keeping in mind
        considerations for chroot (be sure the path is accessible inside
        the chroot) and uid/gid (be sure the path is appropriately
        writeable).
\end{itemize}

\index{Log facility}  
"facility" must be one of the 24 standard syslog facilities:

\textbf{
\begin{tabular}{llllllll}
  kern   & user   & mail   & daemon & auth   & syslog & lpr    & news \\
  uucp   & cron   & auth2  & ftp    & ntp    & audit  & alert  & cron2 \\
  local0 & local1 & local2 & local3 & local4 & local5 & local6 & local7
\end{tabular}
}
    
An optional level can be specified to filter outgoing messages. By default,
all messages are sent. If a maximum level is specified, only messages with a
severity at least as important as this level will be sent. An optional minimum
level can be specified. If it is set, logs emitted with a more severe level
than this one will be capped to this level. This is used to avoid sending
"emerg" messages on all terminals on some default syslog configurations.
  
\index{Log level}
Eight levels are known:

\textbf {
\begin{tabular}{llllllll}
        emerg & alert & crit & err & warning & notice & info & debug
\end{tabular}
}
  
\subsubsection[log-send-hostname]{\kw{log-send-hostname} [string]}
\index{log-send-hostname}

Sets the hostname field in the syslog header. If optional "string" parameter
is set the header is set to the string contents, otherwise uses the hostname
of the system. Generally used if one is not relaying logs through an
intermediate syslog server or for simply customizing the hostname printed in
the logs.

\subsubsection[log-tag]{\kw{log-tag} <string>}
\index{log-tag}

Sets the tag field in the syslog header to this string. It defaults to the
program name as launched from the command line, which usually is "haproxy".
Sometimes it can be useful to differentiate between multiple processes
running on the same host.

\subsubsection[nbproc]{\kw{nbproc} <number>}
\index{nbproc}

Creates <number> processes when going daemon. This requires the "daemon"
mode. By default, only one process is created, which is the recommended mode
of operation. For systems limited to small sets of file descriptors per
process, it may be needed to fork multiple daemons.

\begin{note}{Note:}
Using multiple processes
is harder to debug and is really discouraged.
\end{note}

See also \kwl{daemon}.

\subsubsection[pidfile]{\kw{pidfile} <pidfile>}
\index{pidfile}

  Writes pids of all daemons into file "pidfile". This option is equivalent to
  the \cmd{-p} command line argument. The file must be accessible to the user
  starting the process. See also \kwl{daemon}.

\subsubsection[stats socket]{\kw{stats socket} <path> [ \{uid | user\} <uid> ] [ \{gid | group\} <gid> ] [mode <mode>] [level <level>]}
\index{stats socket}

Creates a UNIX socket in stream mode at location "path". Any previously
existing socket will be backed up then replaced. Connections to this socket
will return various statistics outputs and even allow some commands to be
issued. Please consult section~\ref{sec:socket_commands}
"~\nameref{sec:socket_commands}" for more details.

An optional "level" parameter can be specified to restrict the nature of
the commands that can be issued on the socket:

\begin{itemize}
\item[-] "user" is the least privileged level; only non-sensitive stats can be
  read, and no change is allowed. It would make sense on systems where it
  is not easy to restrict access to the socket.

\item[-] "operator" is the default level and fits most common uses. All data can
  be read, and only non-sensitive changes are permitted (eg: clear max
  counters).

\item[-] "admin" should be used with care, as everything is permitted (eg: clear
  all counters).
\end{itemize}
  
On platforms which support it, it is possible to restrict access to this
socket by specifying numerical IDs after "uid" and "gid", or valid user and
group names after the "user" and "group" keywords. It is also possible to
restrict permissions on the socket by passing an octal value after the "mode"
keyword (same syntax as chmod). Depending on the platform, the permissions on
the socket will be inherited from the directory which hosts it, or from the
user the process is started with.

\subsubsection[stats timeout]{\kw{stats timeout} <timeout, in milliseconds>}
\index{stats timeout}

The default timeout on the stats socket is set to 10 seconds. It is possible
to change this value with "stats timeout". The value must be passed in
milliseconds, or be suffixed by a time unit among { us, ms, s, m, h, d }.

\subsubsection[stats maxconn]{\kw{stats maxconn} <connections>}
\index{stats maxconn}

By default, the stats socket is limited to 10 concurrent connections. It is
possible to change this value with "stats maxconn".

\subsubsection[uid]{\kw{uid} <number>}
\index{uid}

Changes the process' user ID to <number>. It is recommended that the user ID
is dedicated to HAProxy or to a small set of similar daemons. HAProxy must
be started with superuser privileges in order to be able to switch to another
one. See also \kwl{gid} and \kwl{user}.

\subsubsection[ulimit-n]{\kw{ulimit-n} <number>}
\index{ulimit-n}

Sets the maximum number of per-process file-descriptors to <number>. By
default, it is automatically computed, so it is recommended not to use this
option.

\subsubsection[unix-bind]{\kw{unix-bind} [ prefix <prefix> ] [ mode <mode> ] 
[ user <user> ] [ uid <uid> ] [ group <group> ] [ gid <gid> ]}
\index{unix-bind}

Fixes common settings to UNIX listening sockets declared in "bind" statements.
This is mainly used to simplify declaration of those UNIX sockets and reduce
the risk of errors, since those settings are most commonly required but are
also process-specific. The <prefix> setting can be used to force all socket
path to be relative to that directory. This might be needed to access another
component's chroot. Note that those paths are resolved before haproxy chroots
itself, so they are absolute. The <mode>, <user>, <uid>, <group> and <gid>
all have the same meaning as their homonyms used by the "bind" statement. If
both are specified, the "bind" statement has priority, meaning that the
"unix-bind" settings may be seen as process-wide default settings.

\subsubsection[user]{\kw{user} <user name>}
\index{user}

Similar to "uid" but uses the UID of user name <user name> from \cmd{/etc/passwd}.
See also "uid" and "group".

\subsubsection[node]{\kw{node} <name>}
\index{node}

Only letters, digits, hyphen and underscore are allowed, like in DNS names.

This statement is useful in HA configurations where two or more processes or
servers share the same IP address. By setting a different node-name on all
nodes, it becomes easy to immediately spot what server is handling the
traffic.

\subsubsection[description]{\kw{description} <text>}
\index{description}

Add a text that describes the instance.

Please note that it is required to escape certain characters (\chr{\#} for example)
and this text is inserted into a html page so you should avoid using
\chr{<} and \chr{>} characters.

\section{Performance tuning}
\label{sec:performance}

\subsubsection[maxconn]{\kw{maxconn} <number>}
\index{maxconn}

Sets the maximum per-process number of concurrent connections to <number>. It
is equivalent to the command-line argument \cmd{-n}. Proxies will stop accepting
connections when this limit is reached. The "ulimit-n" parameter is
automatically adjusted according to this value.

See also \kwl{ulimit-n}.

\subsubsection[maxconnrate]{\kw{maxconnrate} <number>}
\index{maxconnrate}

Sets the maximum per-process number of connections per second to <number>.
Proxies will stop accepting connections when this limit is reached. It can be
used to limit the global capacity regardless of each frontend capacity. It is
important to note that this can only be used as a service protection measure,
as there will not necessarily be a fair share between frontends when the
limit is reached, so it's a good idea to also limit each frontend to some
value close to its expected share.

Also, lowering \kwl{tune.maxaccept} can improve fairness.

\subsubsection[maxpipes]{\kw{maxpipes} <number>}
\index{maxpipes}
Sets the maximum per-process number of pipes to <number>. Currently, pipes
are only used by kernel-based tcp splicing. Since a pipe contains two file
descriptors, the "ulimit-n" value will be increased accordingly. The default
value is maxconn/4, which seems to be more than enough for most heavy usages.
The splice code dynamically allocates and releases pipes, and can fall back
to standard copy, so setting this value too low may only impact performance.

\subsubsection[noepoll]{\kw{noepoll}}
\index{noepoll}

Disables the use of the "epoll" event polling system on Linux. It is
equivalent to the command-line argument \cmd{-de}. The next polling system
used will generally be "poll".

See also \kwl{nosepoll}, and \kwl{nopoll}.

\subsubsection[nokqueue]{\kw{nokqueue}}
\index{nokqueue}

Disables the use of the "kqueue" event polling system on BSD. It is
equivalent to the command-line argument \cmd{-dk}. The next polling system
used will generally be "poll".

See also \kwl{nopoll}.

\subsubsection[nopoll]{\kw{nopoll}}
\index{nopoll}

Disables the use of the "poll" event polling system. It is equivalent to the
command-line argument \cmd{-dp}. The next polling system used will be "select".
It should never be needed to disable "poll" since it's available on all
platforms supported by HAProxy.

See also \kwl{nosepoll}, and \kwl{nopoll} and kwl{nokqueue}.

\subsubsection[nosepoll]{\kw{nosepoll}}
\index{nosepoll}

Disables the use of the "speculative epoll" event polling system on Linux. It
is equivalent to the command-line argument \cmd{-ds}. The next polling system
used will generally be "epoll".

See also \kwl{nosepoll}, and \kwl{nopoll}.

\subsubsection[nosplice]{\kw{nosplice}}
\index{nosplice}

Disables the use of kernel tcp splicing between sockets on Linux. It is
equivalent to the command line argument \cmd{-dS}.  Data will then be copied
using conventional and more portable recv/send calls. Kernel tcp splicing is
limited to some very recent instances of kernel 2.6. Most versions between
2.6.25 and 2.6.28 are buggy and will forward corrupted data, so they must not
be used. This option makes it easier to globally disable kernel splicing in
case of doubt. See also \optl{splice-auto}, \optl{splice-request} and
\optl{splice-response}.

\subsubsection[spread-checks]{\kw{spread-checks} <0..50, in percent>}
\index{spread-checks}

Sometimes it is desirable to avoid sending health checks to servers at exact
intervals, for instance when many logical servers are located on the same
physical server. With the help of this parameter, it becomes possible to add
some randomness in the check interval between 0 and +/- 50\%. A value between
2 and 5 seems to show good results. The default value remains at 0.

\subsubsection[tune.bufsize]{\kw{tune.bufsize} <number>}
\index{tune.bufsize}

Sets the buffer size to this size (in bytes). Lower values allow more
sessions to coexist in the same amount of RAM, and higher values allow some
applications with very large cookies to work. The default value is 16384 and
can be changed at build time. It is strongly recommended not to change this
from the default value, as very low values will break some services such as
statistics, and values larger than default size will increase memory usage,
possibly causing the system to run out of memory. At least the global \kwl{maxconn}
parameter should be decreased by the same factor as this one is increased.

\subsubsection[tune.chksize]{\kw{tune.chksize} <number>}
\index{tune.chksize}

Sets the check buffer size to this size (in bytes). Higher values may help
find string or regex patterns in very large pages, though doing so may imply
more memory and CPU usage. The default value is 16384 and can be changed at
build time. It is not recommended to change this value, but to use better
checks whenever possible.

\subsubsection[tune.http.maxhdr]{\kw{tune.http.maxhdr} <number>}
\index{tune.http.maxhdr}

Sets the maximum number of headers in a request. When a request comes with a
number of headers greater than this value (including the first line), it is
rejected with a "400 Bad Request" status code. Similarly, too large responses
are blocked with "502 Bad Gateway". The default value is 101, which is enough
for all usages, considering that the widely deployed Apache server uses the
same limit. It can be useful to push this limit further to temporarily allow
a buggy application to work by the time it gets fixed. Keep in mind that each
new header consumes 32bits of memory for each session, so don't push this
limit too high.

\subsubsection[tune.maxaccept]{\kw{tune.maxaccept} <number>}
\index{tune.maxaccept}

Sets the maximum number of consecutive accepts that a process may perform on
a single wake up. High values give higher priority to high connection rates,
while lower values give higher priority to already established connections.
This value is limited to 100 by default in single process mode. However, in
multi-process mode (nbproc > 1), it defaults to 8 so that when one process
wakes up, it does not take all incoming connections for itself and leaves a
part of them to other processes. Setting this value to -1 completely disables
the limitation. It should normally not be needed to tweak this value.

\subsubsection[tune.maxpollevents]{\kw{tune.maxpollevents} <number>}
\index{tune.maxpollevents}

Sets the maximum amount of events that can be processed at once in a call to
the polling system. The default value is adapted to the operating system. It
has been noticed that reducing it below 200 tends to slightly decrease
latency at the expense of network bandwidth, and increasing it above 200
tends to trade latency for slightly increased bandwidth.

\subsubsection[tune.maxrewrite]{\kw{tune.maxrewrite} <number>}
\index{tune.maxrewrite}

Sets the reserved buffer space to this size in bytes. The reserved space is
used for header rewriting or appending. The first reads on sockets will never
fill more than bufsize-maxrewrite. Historically it has defaulted to half of
bufsize, though that does not make much sense since there are rarely large
numbers of headers to add. Setting it too high prevents processing of large
requests or responses. Setting it too low prevents addition of new headers
to already large requests or to POST requests. It is generally wise to set it
to about 1024. It is automatically readjusted to half of bufsize if it is
larger than that. This means you don't have to worry about it when changing
bufsize.

\subsubsection[tune.pipesize]{\kw{tune.pipesize} <number>}
\index{tune.pipesize}

Sets the kernel pipe buffer size to this size (in bytes). By default, pipes
are the default size for the system. But sometimes when using TCP splicing,
it can improve performance to increase pipe sizes, especially if it is
suspected that pipes are not filled and that many calls to splice() are
performed. This has an impact on the kernel's memory footprint, so this must
not be changed if impacts are not understood.

\subsubsection[tune.rcvbuf.client]{\kw{tune.rcvbuf.client} <number>}
\subsubsection[tune.rcvbuf.server]{\kw{tune.rcvbuf.server} <number>}
\index{tune.rcvbuf.client}
\index{tune.rcvbuf.server}

Forces the kernel socket receive buffer size on the client or the server side
to the specified value in bytes. This value applies to all TCP/HTTP frontends
and backends. It should normally never be set, and the default size (0) lets
the kernel autotune this value depending on the amount of available memory.
However it can sometimes help to set it to very low values (eg: 4096) in
order to save kernel memory by preventing it from buffering too large amounts
of received data. Lower values will significantly increase CPU usage though.

\subsubsection[tune.sndbuf.client]{\kw{tune.sndbuf.client} <number>}
\subsubsection[tune.sndbuf.server]{\kw{tune.sndbuf.server} <number>}
\index{tune.sndbuf.client}
\index{tune.sndbuf.server}

Forces the kernel socket send buffer size on the client or the server side to
the specified value in bytes. This value applies to all TCP/HTTP frontends
and backends. It should normally never be set, and the default size (0) lets
the kernel autotune this value depending on the amount of available memory.
However it can sometimes help to set it to very low values (eg: 4096) in
order to save kernel memory by preventing it from buffering too large amounts
of received data. Lower values will significantly increase CPU usage though.
Another use case is to prevent write timeouts with extremely slow clients due
to the kernel waiting for a large part of the buffer to be read before
notifying haproxy again.

\section{Debugging}
\label{sec:debugging}

\subsubsection[debug]{\kw{debug}}
\index{debug}

Enables debug mode which dumps to stdout all exchanges, and disables forking
into background. It is the equivalent of the command-line argument \cmd{-d}. It
should never be used in a production configuration since it may prevent full
system startup.

\subsubsection[quiet]{\kw{quiet}}
\index{quiet}

Do not display any message during startup. It is equivalent to the command-
line argument \cmd{-q}.

\section{Userlists}
\label{sec:userlists}

It is possible to control access to frontend/backend/listen sections or to
http stats by allowing only authenticated and authorized users. To do this,
it is required to create at least one userlist and to define users.

\subsubsection[userlist]{\kw{userlist} <listname>}
\index{userlist}

Creates new userlist with name <listname>. Many independent userlists can be
used to store authentication \& authorization data for independent customers.

\subsubsection[group]{\kw{group} <groupname> [users <user>,<user>,(...)]}
\index{group}

Adds group <groupname> to the current userlist. It is also possible to
attach users to this group by using a comma separated list of names
proceeded by "users" keyword.

\subsubsection[user]{\kw{user} <username> [password | insecure-password <password>]
  [groups <group>, (...)]}
\index{user}

Adds user <username> to the current userlist. Both secure (encrypted) and
insecure (unencrypted) passwords can be used. Encrypted passwords are
evaluated using the crypt(3) function so depending of the system's
capabilities, different algorithms are supported. For example modern Glibc
based Linux system supports MD5, SHA-256, SHA-512 and of course classic,
DES-based method of crypting passwords.

\begin{example}{Example:}
        userlist L1
          group G1 users tiger,scott
          group G2 users xdb,scott

          user tiger password $6$k6y3o.eP$JlKBx9za9667qe4(...)xHSwRv6J.C0/D7cV91
          user scott insecure-password elgato
          user xdb insecure-password hello

        userlist L2
          group G1
          group G2

          user tiger password $6$k6y3o.eP$JlKBx(...)xHSwRv6J.C0/D7cV91 groups G1
          user scott insecure-password elgato groups G1,G2
          user xdb insecure-password hello groups G2
\end{example}

  Please note that both lists are functionally identical.

\section{Peers}
\label{sec:peers}

It is possible to synchronize server entries in stick tables between several
haproxy instances over TCP connections in a multi-master fashion. Each instance
pushes its local updates and insertions to remote peers. Server IDs are used to
identify servers remotely, so it is important that configurations look similar
or at least that the same IDs are forced on each server on all participants.
Interrupted exchanges are automatically detected and recovered from the last
known point. In addition, during a soft restart, the old process connects to
the new one using such a TCP connection to push all its entries before the new
process tries to connect to other peers. That ensures very fast replication
during a reload, it typically takes a fraction of a second even for large
tables.

\subsubsection[peers]{\kw{peers} <peersect>}
\index{peers}

Creates a new peer list with name <peersect>. It is an independant section,
which is referenced by one or more stick-tables.

\subsubsection[peer]{\kw{peer} <peername> <ip>:<port>}
\index{peer}

Defines a peer inside a peers section.
If <peername> is set to the local peer name (by default hostname, or forced
using "-L" command line option), haproxy will listen for incoming remote peer
connection on <ip>:<port>. Otherwise, <ip>:<port> defines where to connect to
to join the remote peer, and <peername> is used at the protocol level to
identify and validate the remote peer on the server side.

During a soft restart, local peer <ip>:<port> is used by the old instance to
connect the new one and initiate a complete replication (teaching process).

It is strongly recommended to have the exact same peers declaration on all
peers and to only rely on the "-L" command line argument to change the local
peer name. This makes it easier to maintain coherent configuration files
across all peers.

\begin{example}{Example:}
    peers mypeers
        peer haproxy1 192.168.0.1:1024
        peer haproxy2 192.168.0.2:1024
        peer haproxy3 10.2.0.1:1024

    backend mybackend
        mode tcp
        balance roundrobin
        stick-table type ip size 20k peers mypeers
        stick on src

        server srv1 192.168.0.30:80
        server srv2 192.168.0.31:80
\end{example}
