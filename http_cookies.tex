HTTP cookies are not uniformly supported across browsers, which makes it very
hard to build a widely compatible implementation. At least four conflicting
documents exist to describe how cookies should be handled, and browsers
generally don't respect any but a sensibly selected mix of them:

\begin{itemize}
\item[-] Netscape's original spec (also mirrored at Curl's site among others):

    \verb|http://web.archive.org/web/20070805052634/|
    
    \verb|http://wp.netscape.com/newsref/std/cookie_spec.html|
    
    \verb|http://curl.haxx.se/rfc/cookie_spec.html|

    \emph{Issues:} uses an unquoted "Expires" field that includes a comma.

\item[-] RFC 2109:

    \verb|http://www.ietf.org/rfc/rfc2109.txt|

    \emph{Issues:} specifies use of "Max-Age" (not universally implemented) and does
            not talk about "Expires" (generally supported). References quoted
            strings, not generally supported (eg: MSIE). Stricter than browsers
            about domains. Ambiguous about allowed spaces in values and attrs.

\item[-] RFC 2965:
    
    \verb|http://www.ietf.org/rfc/rfc2965.txt|

    \emph{Issues:} same as RFC2109 + describes Set-Cookie2 which only Opera supports.

\item[-] Current internet draft:
    
    \verb|https://datatracker.ietf.org/wg/httpstate/charter/|

    \emph{Issues:} as of -p10, does not explain how the Set-Cookie2 header must be
            emitted/handled, while suggesting a stricter approach for Cookie.
            Documents reality and as such reintroduces the widely used unquoted
            "Expires" attribute with its error-prone syntax. States that a
            server should not emit more than one cookie per Set-Cookie header,
            which is incompatible with HTTP which says that multiple headers
            are allowed only if they can be folded.
\end{itemize}

See also the following URL for a browser * feature matrix:

   \verb|http://code.google.com/p/browsersec/wiki/Part2#Same-origin_policy_for_cookies|

In short, MSIE and Safari neither support quoted strings nor max-age, which
make it mandatory to continue to send an unquoted Expires value (maybe the
day of week could be omitted though). Only Safari supports comma-separated
lists of Set-Cookie headers. Support for cross-domains is not uniform either.
