\chapter{Configuring HAProxy}
\label{chap:configuring_haproxy}

\section{Configuration file format}
\label{sec:config_format}

HAProxy's configuration process involves 3 major sources of parameters:

\begin{enumerate}
\item the arguments from the command-line, which always take precedence
\item the "global" section, which sets process-wide parameters
\item the proxies sections which can take form of "defaults", "listen",
    "frontend" and "backend".
\end{enumerate}

The configuration file syntax consists in lines beginning with a keyword
referenced in this manual, optionally followed by one or several parameters
delimited by spaces. If spaces have to be entered in strings, then they must be
preceded by a backslash \chr{\bslash} to be escaped. Backslashes also have to be
escaped by doubling them.

\section{Time format}
\label{sec:time_format}

Some parameters involve values representing time, such as timeouts. These
values are generally expressed in milliseconds (unless explicitly stated
otherwise) but may be expressed in any other unit by suffixing the unit to the
numeric value. It is important to consider this because it will not be repeated
for every keyword.

\index{Time format}

Supported units are:

\begin{description}
\item[us:] microseconds. 1 microsecond = 1/1000000 second
\item[ms:] milliseconds. 1 millisecond = 1/1000 second. This is the default.
\item[s:]  seconds. 1s = 1000ms
\item[m:]  minutes. 1m = 60s = 60000ms
\item[h:]  hours.   1h = 60m = 3600s = 3600000ms
\item[d:]  days.    1d = 24h = 1440m = 86400s = 86400000ms
\end{description}

\section{Examples}
\label{sec:examples}

\begin{example}{Simple configuration for an HTTP proxy}
    # Simple configuration for an HTTP proxy listening on port 80 on all
    # interfaces and forwarding requests to a single backend "servers" with a
    # single server "server1" listening on 127.0.0.1:8000
    global
        daemon
        maxconn 256

    defaults
        mode http
        timeout connect 5000ms
        timeout client 50000ms
        timeout server 50000ms

    frontend http-in
        bind *:80
        default_backend servers

    backend servers
        server server1 127.0.0.1:8000 maxconn 32


    # The same configuration defined with a single listen block. Shorter but
    # less expressive, especially in HTTP mode.
    global
        daemon
        maxconn 256

    defaults
        mode http
        timeout connect 5000ms
        timeout client 50000ms
        timeout server 50000ms

    listen http-in
        bind *:80
        server server1 127.0.0.1:8000 maxconn 32
\end{example}

Assuming haproxy is in \cmd{\$PATH}, test these configurations in a shell with:

\cmd{sudo haproxy -f configuration.conf -c}
