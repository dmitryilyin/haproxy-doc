HTTP transactions supports 5 possible modes:

\vspace{3mm}
\begin{tabular}{ll}
WANT\_TUN             & default, nothing changed \\
WANT\_TUN + httpclose & headers set for close in both dirs \\
WANT\_KAL             & keep-alive desired in both dirs \\
WANT\_SCL             & want close with the server and KA with the client \\
WANT\_CLO             & want close on both sides. \\
\end{tabular}
\vspace{3mm}

When only WANT\_TUN is set, nothing is changed nor analyzed, so for commodity
below, we'll refer to WANT\_TUN + httpclose as WANT\_TUN.

The mode is adjusted in 3 steps:
\begin{itemize}
\item[-] configuration sets initial mode
\item[-] request headers set required request mode
\item[-] response headers set the final mode
\end{itemize}

\section{Adjusting the initial mode via the configuration}

\vspace{3mm}
\begin{tabular}{lcl}
option httpclose         & => & TUN \\
option http-keep-alive   & => & KAL \\
option http-server-close & => & SCL \\
option forceclose        & => & CLO \\
\end{tabular}
\vspace{3mm}

Note that option httpclose combined with any other option is equivalent to
forceclose.

\section{Adjusting the request mode once the request is parsed}

If we cannot determine the body length from the headers, we set the mode to CLO
but later we'll switch to tunnel mode once forwarding the body. That way, all
parties are informed of the correct mode.

Depending on the request version and request Connection header, we may have to
adjust the current transaction mode and to update the connection header.

\vspace{3mm}
\begin{longtable}{lcccl}
\head{mode} & \head{req\_ver} & \head{req\_hdr} & \head{new\_mode} & \head{hdr\_change} \\
\hline
TUN  & 1.0  &      -      & TUN   &    - \\
TUN  & 1.0  &     ka      & TUN   &   del\_ka \\
TUN  & 1.0  &    close    & TUN   &   del\_close \\
TUN  & 1.0  &    both     & TUN   &   del\_ka, del\_close \\
\hline
TUN  & 1.1  &      -      & TUN   &   add\_close \\
TUN  & 1.1  &     ka      & TUN   &   del\_ka, add\_close \\
TUN  & 1.1  &    close    & TUN   &    - \\
TUN  & 1.1  &    both     & TUN   &   del\_ka \\
\hline
KAL  & 1.0  &      -      & CLO   &    - \\
KAL  & 1.0  &     ka      & KAL   &    - \\
KAL  & 1.0  &    close    & CLO   &   del\_close \\
KAL  & 1.0  &    both     & CLO   &   del\_ka, del\_close \\
\hline
KAL  & 1.1  &      -      & KAL   &    - \\
KAL  & 1.1  &     ka      & KAL   &   del\_ka \\
KAL  & 1.1  &    close    & CLO   &    - \\
KAL  & 1.1  &    both     & CLO   &   del\_ka \\
\hline
SCL  & 1.0  &      -      & CLO   &    - \\
SCL  & 1.0  &     ka      & SCL   &   del\_ka \\
SCL  & 1.0  &    close    & CLO   &   del\_close \\
SCL  & 1.0  &    both     & CLO   &   del\_ka, del\_close \\
\hline
SCL  & 1.1  &      -      & SCL   &   add\_close \\
SCL  & 1.1  &     ka      & SCL   &   del\_ka, add\_close \\
SCL  & 1.1  &    close    & CLO   &    - \\
SCL  & 1.1  &    both     & CLO   &   del\_ka \\
\hline
CLO  & 1.0  &      -      & CLO   &    - \\
CLO  & 1.0  &     ka      & CLO   &   del\_ka \\
CLO  & 1.0  &    close    & CLO   &   del\_close \\
CLO  & 1.0  &    both     & CLO   &   del\_ka, del\_close \\
\hline
CLO  & 1.1  &      -      & CLO   &   add\_close \\
CLO  & 1.1  &     ka      & CLO   &   del\_ka, add\_close \\
CLO  & 1.1  &    close    & CLO   &    - \\
CLO  & 1.1  &    both     & CLO   &   del\_ka \\
\end{longtable}
\vspace{3mm}

\textbf{Summary:}
\begin{itemize}
\item KAL and SCL are only possible with the same requests:
     \begin{itemize}
     \item[-] 1.0 + ka
     \item[-] 1.1 + ka or nothing
     \end{itemize}

\item CLO is assumed for any non-TUN request which contains at least a close
     header, as well as for any 1.0 request without a keep-alive header.

\item del\_ka is set whenever we want a CLO or SCL or TUN and req contains a KA,
     or when the req is 1.1 and contains a KA.

\item del\_close is set whenever a 1.0 request contains a close.

\item add\_close is set whenever a 1.1 request must be switched to TUN, SCL, CLO
     and did not have a close hdr.
\end{itemize}

Note that the request processing is performed in two passes, one with the
frontend's config and a second one with the backend's config. It is only
possible to "raise" the mode between them, so during the second pass, we have
no reason to re-add a header that we previously removed. As an exception, the
TUN mode is converted to CLO once combined because in fact it's an httpclose
option set on a TUN mode connection:

\begin{verbatim}
                    BE (2)
            | TUN  KAL  SCL  CLO
        ----+----+----+----+----
        TUN | TUN  CLO  CLO  CLO
            +
        KAL | CLO  KAL  SCL  CLO
    FE      +
    (1) SCL | CLO  SCL  SCL  CLO
            +
        CLO | CLO  CLO  CLO  CLO
\end{verbatim}

\section{Adjusting the final mode once the response is parsed}

This part becomes trickier. It is possible that the server responds with a
version that the client does not necessarily understand. Obviously, 1.1 clients
are assumed to understand 1.0 responses. The problematic case is a 1.0 client
receiving a 1.1 response without any Connection header. Some 1.0 clients might
know that in 1.1 this means "keep-alive" while others might ignore the version
and assume a "close". Since we know the version on both sides, we may have to
adjust some responses to remove any ambiguous case. That's the reason why the
following table considers both the request and the response version. If the
response length cannot be determined, we switch to CLO mode.

\vspace{3mm}
\begin{longtable}{lccccl}
\head{mode} & \head{res\_ver} & \head{res\_hdr} & \head{req\_ver} & \head{new\_mode} & \head{hdr\_change} \\
\hline
TUN  & 1.0   &   -     &  any   &   TUN   &    - \\
TUN  & 1.0   &  ka     &  any   &   TUN   &   del\_ka \\
TUN  & 1.0   &  close  &  any   &   TUN   &   del\_close \\
TUN  & 1.0   &  both   &  any   &   TUN   &   del\_ka, del\_close \\
\hline
TUN  & 1.1   &    -    &  any   &   TUN   &   add\_close \\
TUN  & 1.1   &   ka    &  any   &   TUN   &   del\_ka, add\_close \\
TUN  & 1.1   &  close  &  any   &   TUN   &    - \\
TUN  & 1.1   &  both   &  any   &   TUN   &   del\_ka \\
\hline
KAL  & 1.0   &    -    &  any   &   SCL   &   add\_ka \\
KAL  & 1.0   &   ka    &  any   &   KAL   &    - \\
KAL  & 1.0   &  close  &  any   &   SCL   &   del\_close, add\_ka \\
KAL  & 1.0   &  both   &  any   &   SCL   &   del\_close \\
\hline
KAL  & 1.1   &    -    &  1.0   &   KAL   &   add\_ka \\
KAL  & 1.1   &    -    &  1.1   &   KAL   &    - \\
KAL  & 1.1   &   ka    &  1.0   &   KAL   &    - \\
KAL  & 1.1   &   ka    &  1.1   &   KAL   &   del\_ka \\
KAL  & 1.1   &  close  &  1.0   &   SCL   &   del\_close, add\_ka \\
KAL  & 1.1   &  close  &  1.1   &   SCL   &   del\_close \\
KAL  & 1.1   &  both   &  1.0   &   SCL   &   del\_close \\
KAL  & 1.1   &  both   &  1.1   &   SCL   &   del\_ka, del\_close \\
\hline
SCL  & 1.0   &    -    &  any   &   SCL   &   add\_ka \\
SCL  & 1.0   &   ka    &  any   &   SCL   &    - \\
SCL  & 1.0   &  close  &  any   &   SCL   &   del\_close, add\_ka \\
SCL  & 1.0   &  both   &  any   &   SCL   &   del\_close \\
\hline
SCL  & 1.1   &    -    &  1.0   &   SCL   &   add\_ka \\
SCL  & 1.1   &    -    &  1.1   &   SCL   &    - \\
SCL  & 1.1   &   ka    &  1.0   &   SCL   &    - \\
SCL  & 1.1   &   ka    &  1.1   &   SCL   &   del\_ka \\
SCL  & 1.1   &  close  &  1.0   &   SCL   &   del\_close, add\_ka \\
SCL  & 1.1   &  close  &  1.1   &   SCL   &   del\_close \\
SCL  & 1.1   &  both   &  1.0   &   SCL   &   del\_close \\
SCL  & 1.1   &  both   &  1.1   &   SCL   &   del\_ka, del\_close \\
\hline
CLO  & 1.0   &    -    &  any   &   CLO   &    - \\
CLO  & 1.0   &   ka    &  any   &   CLO   &   del\_ka \\
CLO  & 1.0   &  close  &  any   &   CLO   &   del\_close \\
CLO  & 1.0   &  both   &  any   &   CLO   &   del\_ka, del\_close \\
\hline
CLO  & 1.1   &    -    &  any   &   CLO   &   add\_close \\
CLO  & 1.1   &   ka    &  any   &   CLO   &   del\_ka, add\_close \\
CLO  & 1.1   &  close  &  any   &   CLO   &    - \\
CLO  & 1.1   &  both   &  any   &   CLO   &   del\_ka \\
\end{longtable}
\vspace{3mm}

\textbf{In summary:}
\begin{itemize}
\item[-] the header operations do not depend on the initial mode, they only depend
    on versions and current connection header(s).

\item[-] both CLO and TUN modes work similarly, they need to set a close mode on the
    reponse. A 1.1 response will exclusively need the close header, while a 1.0
    response will have it removed. Any keep-alive header is always removed when
    found.

\item[-] a KAL request where the server wants to close turns into an SCL response so
    that we release the server but still maintain the connection to the client.

\item[-] the KAL and SCL modes work the same way as we need to set keep-alive on the
    response. So a 1.0 response will only have the keep-alive header with any
    close header removed. A 1.1 response will have the keep-alive header added
    for 1.0 requests and the close header removed for all requests.
\end{itemize}

Note that the SCL and CLO modes will automatically cause the server connection
to be closed at the end of the data transfer.