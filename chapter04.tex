\chapter{Proxies}

Proxy configuration can be located in a set of sections:
\begin{itemize}
\item[-] defaults <name>
\item[-] frontend <name>
\item[-] backend  <name>
\item[-] listen   <name>
\end{itemize}

\begin{description}
\item[A "defaults" section] sets default parameters for all other sections following
its declaration. Those default parameters are reset by the next "defaults"
section. See below for the list of parameters which can be set in a "defaults"
section. The name is optional but its use is encouraged for better readability.

\item[A "frontend" section] describes a set of listening sockets accepting client
connections.

\item[A "backend" section] describes a set of servers to which the proxy will connect
to forward incoming connections.

\item[A "listen" section] defines a complete proxy with its frontend and backend
parts combined in one section. It is generally useful for TCP-only traffic.
\end{description}

\index{proxy}
All proxy names must be formed from upper and lower case letters, digits,
'-' (dash), '\_' (underscore) , '.' (dot) and ':' (colon). ACL names are
case-sensitive, which means that "www" and "WWW" are two different proxies.

Historically, all proxy names could overlap, it just caused troubles in the
logs. Since the introduction of content switching, it is mandatory that two
proxies with overlapping capabilities (frontend/backend) have different names.
However, it is still permitted that a frontend and a backend share the same
name, as this configuration seems to be commonly encountered.

Right now, two major proxy modes are supported: "tcp", also known as layer 4,
and "http", also known as layer 7. In layer 4 mode, HAProxy simply forwards
bidirectional traffic between two sides. In layer 7 mode, HAProxy analyzes the
protocol, and can interact with it by allowing, blocking, switching, adding,
modifying, or removing arbitrary contents in requests or responses, based on
arbitrary criteria.

\section{Proxy keywords matrix}
The following list of keywords is supported. Most of them may only be used in a
limited set of section types. Some of them are marked as "deprecated" because
they are inherited from an old syntax which may be confusing or functionally
limited, and there are new recommended keywords to replace them. Keywords
marked with "(*)" can be optionally inverted using the "no" prefix, eg. "no
option contstats". This makes sense when the option has been enabled by default
and must be disabled for a specific instance. Such options may also be prefixed
with "default" in order to restore default settings regardless of what has been
specified in a previous "defaults" section.

\begin{longtable}{|lr|c|c|c|c|}
\hline
\head{Keyword}&\head{Mark}&\head{Defaults}&\head{Frontend}&\head{Listen}&\head{Backend}\\
\hline
acl                                 &              & ---      & $\times$ & $\times$ & $\times$ \\
\hline
appsession                          &              & ---      & ---      & $\times$ & $\times$ \\
\hline
backlog                             &              & $\times$ & $\times$ & $\times$ & ---      \\
\hline
balance                             &              & $\times$ & ---      & $\times$ & $\times$ \\
\hline
bind                                &              & ---      & $\times$ & $\times$ & ---      \\
\hline
bind-process                        &              & $\times$ & $\times$ & $\times$ & $\times$ \\
\hline
block                               &              & ---      & $\times$ & $\times$ & $\times$ \\
\hline
capture cookie                      &              & ---      & $\times$ & $\times$ & ---      \\
\hline
capture request header              &              & ---      & $\times$ & $\times$ & ---      \\
\hline
capture response header             &              & ---      & $\times$ & $\times$ & ---      \\
\hline
clitimeout                          & (deprecated) & $\times$ & $\times$ & $\times$ & ---      \\
\hline
contimeout                          & (deprecated) & $\times$ & ---      & $\times$ & $\times$ \\
\hline
cookie                              &              & $\times$ & ---      & $\times$ & $\times$ \\
\hline
default-server                      &              & $\times$ & ---      & $\times$ & $\times$ \\
\hline
default\_backend                    &              & $\times$ & $\times$ & $\times$ & ---      \\
\hline
description                         &              & ---      & $\times$ & $\times$ & $\times$ \\
\hline
disabled                            &              & $\times$ & $\times$ & $\times$ & $\times$ \\
\hline
dispatch                            &              & ---      & ---      & $\times$ & $\times$ \\
\hline
enabled                             &              & $\times$ & $\times$ & $\times$ & $\times$ \\
\hline
errorfile                           &              & $\times$ & $\times$ & $\times$ & $\times$ \\
\hline
errorloc                            &              & $\times$ & $\times$ & $\times$ & $\times$ \\
\hline
errorloc302                         &              & $\times$ & $\times$ & $\times$ & $\times$ \\
\hline
errorloc303                         &              & $\times$ & $\times$ & $\times$ & $\times$ \\
\hline
force-persist                       &              & ---      & $\times$ & $\times$ & $\times$ \\
\hline
fullconn                            &              & $\times$ & ---      & $\times$ & $\times$ \\
\hline
grace                               &              & $\times$ & $\times$ & $\times$ & $\times$ \\
\hline
hash-type                           &              & $\times$ & ---      & $\times$ & $\times$ \\
\hline
http-check disable-on-404           &              & $\times$ & ---      & $\times$ & $\times$ \\
\hline
http-check expect                   &              & ---      & ---      & $\times$ & $\times$ \\
\hline
http-check send-state               &              & $\times$ & ---      & $\times$ & $\times$ \\
\hline
http-request                        &              & ---      & $\times$ & $\times$ & $\times$ \\
\hline
id                                  &              & ---      & $\times$ & $\times$ & $\times$ \\
\hline
ignore-persist                      &              & ---      & $\times$ & $\times$ & $\times$ \\
\hline
log                                 & (*)          & $\times$ & $\times$ & $\times$ & $\times$ \\
\hline
maxconn                             &              & $\times$ & $\times$ & $\times$ & ---      \\
\hline
mode                                &              & $\times$ & $\times$ & $\times$ & $\times$ \\
\hline
monitor fail                        &              & ---      & $\times$ & $\times$ & ---      \\
\hline
monitor-net                         &              & $\times$ & $\times$ & $\times$ & ---      \\
\hline
monitor-uri                         &              & $\times$ & $\times$ & $\times$ & ---      \\
\hline
option abortonclose                 & (*)          & $\times$ & ---      & $\times$ & $\times$ \\
\hline
option accept-invalid-http-request  & (*)          & $\times$ & $\times$ & $\times$ & ---      \\
\hline
option accept-invalid-http-response & (*)          & $\times$ & ---      & $\times$ & $\times$ \\
\hline
option allbackups                   & (*)          & $\times$ & ---      & $\times$ & $\times$ \\
\hline
option checkcache                   & (*)          & $\times$ & ---      & $\times$ & $\times$ \\
\hline
option clitcpka                     & (*)          & $\times$ & $\times$ & $\times$ & ---      \\
\hline
option contstats                    & (*)          & $\times$ & $\times$ & $\times$ & ---      \\
\hline
option dontlog-normal               & (*)          & $\times$ & $\times$ & $\times$ & ---      \\
\hline
option dontlognull                  & (*)          & $\times$ & $\times$ & $\times$ & ---      \\
\hline
option forceclose                   & (*)          & $\times$ & $\times$ & $\times$ & $\times$ \\
\hline
option forwardfor                   &              & $\times$ & $\times$ & $\times$ & $\times$ \\
\hline
option http-no-delay                & (*)          & $\times$ & $\times$ & $\times$ & $\times$ \\
\hline
option http-pretend-keepalive       & (*)          & $\times$ & $\times$ & $\times$ & $\times$ \\
\hline
option http-server-close            & (*)          & $\times$ & $\times$ & $\times$ & $\times$ \\
\hline
option http-use-proxy-header        & (*)          & $\times$ & $\times$ & $\times$ & ---      \\
\hline
option httpchk                      &              & $\times$ & ---      & $\times$ & $\times$ \\
\hline
option httpclose                    & (*)          & $\times$ & $\times$ & $\times$ & $\times$ \\
\hline
option httplog                      &              & $\times$ & $\times$ & $\times$ & $\times$ \\
\hline
option http\_proxy                  & (*)          & $\times$ & $\times$ & $\times$ & $\times$ \\
\hline
option independant-streams          & (*)          & $\times$ & $\times$ & $\times$ & $\times$ \\
\hline
option ldap-check                   &              & $\times$ & ---      & $\times$ & $\times$ \\
\hline
option log-health-checks            & (*)          & $\times$ & ---      & $\times$ & $\times$ \\
\hline
option log-separate-errors          & (*)          & $\times$ & $\times$ & $\times$ & ---      \\
\hline
option logasap                      & (*)          & $\times$ & $\times$ & $\times$ & ---      \\
\hline
option mysql-check                  &              & $\times$ & ---      & $\times$ & $\times$ \\
\hline
option pgsql-check                  &              & $\times$ & ---      & $\times$ & $\times$ \\
\hline
option nolinger                     & (*)          & $\times$ & $\times$ & $\times$ & $\times$ \\
\hline
option originalto                   &              & $\times$ & $\times$ & $\times$ & $\times$ \\
\hline
option persist                      & (*)          & $\times$ & ---      & $\times$ & $\times$ \\
\hline
option redispatch                   & (*)          & $\times$ & ---      & $\times$ & $\times$ \\
\hline
option redis-check                  &              & $\times$ & ---      & $\times$ & $\times$ \\
\hline
option smtpchk                      &              & $\times$ & ---      & $\times$ & $\times$ \\
\hline
option socket-stats                 & (*)          & $\times$ & $\times$ & $\times$ & ---      \\
\hline
option splice-auto                  & (*)          & $\times$ & $\times$ & $\times$ & $\times$ \\
\hline
option splice-request               & (*)          & $\times$ & $\times$ & $\times$ & $\times$ \\
\hline
option splice-response              & (*)          & $\times$ & $\times$ & $\times$ & $\times$ \\
\hline
option srvtcpka                     & (*)          & $\times$ & ---      & $\times$ & $\times$ \\
\hline
option ssl-hello-chk                &              & $\times$ & ---      & $\times$ & $\times$ \\
\hline
option tcp-smart-accept             & (*)          & $\times$ & $\times$ & $\times$ & ---      \\
\hline
option tcp-smart-connect            & (*)          & $\times$ & ---      & $\times$ & $\times$ \\
\hline
option tcpka                        &              & $\times$ & $\times$ & $\times$ & $\times$ \\
\hline
option tcplog                       &              & $\times$ & $\times$ & $\times$ & $\times$ \\
\hline
option transparent                  & (*)          & $\times$ & ---      & $\times$ & $\times$ \\
\hline
persist rdp-cookie                  &              & $\times$ & ---      & $\times$ & $\times$ \\
\hline
rate-limit sessions                 &              & $\times$ & $\times$ & $\times$ & ---      \\
\hline
redirect                            &              & ---      & $\times$ & $\times$ & $\times$ \\
\hline
redisp                              & (deprecated) & $\times$ & ---      & $\times$ & $\times$ \\
\hline
redispatch                          & (deprecated) & $\times$ & ---      & $\times$ & $\times$ \\
\hline
reqadd                              &              & ---      & $\times$ & $\times$ & $\times$ \\
\hline
reqallow                            &              & ---      & $\times$ & $\times$ & $\times$ \\
\hline
reqdel                              &              & ---      & $\times$ & $\times$ & $\times$ \\
\hline
reqdeny                             &              & ---      & $\times$ & $\times$ & $\times$ \\
\hline
reqiallow                           &              & ---      & $\times$ & $\times$ & $\times$ \\
\hline
reqidel                             &              & ---      & $\times$ & $\times$ & $\times$ \\
\hline
reqideny                            &              & ---      & $\times$ & $\times$ & $\times$ \\
\hline
reqipass                            &              & ---      & $\times$ & $\times$ & $\times$ \\
\hline
reqirep                             &              & ---      & $\times$ & $\times$ & $\times$ \\
\hline
reqisetbe                           &              & ---      & $\times$ & $\times$ & $\times$ \\
\hline
reqitarpit                          &              & ---      & $\times$ & $\times$ & $\times$ \\
\hline
reqpass                             &              & ---      & $\times$ & $\times$ & $\times$ \\
\hline
reqrep                              &              & ---      & $\times$ & $\times$ & $\times$ \\
\hline
reqsetbe                            &              & ---      & $\times$ & $\times$ & $\times$ \\
\hline
reqtarpit                           &              & ---      & $\times$ & $\times$ & $\times$ \\
\hline
retries                             &              & $\times$ & ---      & $\times$ & $\times$ \\
\hline
rspadd                              &              & ---      & $\times$ & $\times$ & $\times$ \\
\hline
rspdel                              &              & ---      & $\times$ & $\times$ & $\times$ \\
\hline
rspdeny                             &              & ---      & $\times$ & $\times$ & $\times$ \\
\hline
rspidel                             &              & ---      & $\times$ & $\times$ & $\times$ \\
\hline
rspideny                            &              & ---      & $\times$ & $\times$ & $\times$ \\
\hline
rspirep                             &              & ---      & $\times$ & $\times$ & $\times$ \\
\hline
rsprep                              &              & ---      & $\times$ & $\times$ & $\times$ \\
\hline
server                              &              & ---      & ---      & $\times$ & $\times$ \\
\hline
source                              &              & $\times$ & ---      & $\times$ & $\times$ \\
\hline
srvtimeout                          & (deprecated) & $\times$ & ---      & $\times$ & $\times$ \\
\hline
stats admin                         &              & ---      & ---      & $\times$ & $\times$ \\
\hline
stats auth                          &              & $\times$ & ---      & $\times$ & $\times$ \\
\hline
stats enable                        &              & $\times$ & ---      & $\times$ & $\times$ \\
\hline
stats hide-version                  &              & $\times$ & ---      & $\times$ & $\times$ \\
\hline
stats http-request                  &              & ---      & ---      & $\times$ & $\times$ \\
\hline
stats realm                         &              & $\times$ & ---      & $\times$ & $\times$ \\
\hline
stats refresh                       &              & $\times$ & ---      & $\times$ & $\times$ \\
\hline
stats scope                         &              & $\times$ & ---      & $\times$ & $\times$ \\
\hline
stats show-desc                     &              & $\times$ & ---      & $\times$ & $\times$ \\
\hline
stats show-legends                  &              & $\times$ & ---      & $\times$ & $\times$ \\
\hline
stats show-node                     &              & $\times$ & ---      & $\times$ & $\times$ \\
\hline
stats uri                           &              & $\times$ & ---      & $\times$ & $\times$ \\
\hline
stick match                         &              & ---      & ---      & $\times$ & $\times$ \\
\hline
stick on                            &              & ---      & ---      & $\times$ & $\times$ \\
\hline
stick store-request                 &              & ---      & ---      & $\times$ & $\times$ \\
\hline
stick store-response                &              & ---      & ---      & $\times$ & $\times$ \\
\hline
stick-table                         &              & ---      & ---      & $\times$ & $\times$ \\
\hline
tcp-request connection              &              & ---      & $\times$ & $\times$ & ---      \\
\hline
tcp-request content                 &              & ---      & $\times$ & $\times$ & $\times$ \\
\hline
tcp-request inspect-delay           &              & ---      & $\times$ & $\times$ & $\times$ \\
\hline
tcp-response content                &              & ---      & ---      & $\times$ & $\times$ \\
\hline
tcp-response inspect-delay          &              & ---      & ---      & $\times$ & $\times$ \\
\hline
timeout check                       &              & $\times$ & ---      & $\times$ & $\times$ \\
\hline
timeout client                      &              & $\times$ & $\times$ & $\times$ & ---      \\
\hline
timeout clitimeout                  & (deprecated) & $\times$ & $\times$ & $\times$ & ---      \\
\hline
timeout connect                     &              & $\times$ & ---      & $\times$ & $\times$ \\
\hline
timeout contimeout                  & (deprecated) & $\times$ & ---      & $\times$ & $\times$ \\
\hline
timeout http-keep-alive             &              & $\times$ & $\times$ & $\times$ & $\times$ \\
\hline
timeout http-request                &              & $\times$ & $\times$ & $\times$ & $\times$ \\
\hline
timeout queue                       &              & $\times$ & ---      & $\times$ & $\times$ \\
\hline
timeout server                      &              & $\times$ & ---      & $\times$ & $\times$ \\
\hline
timeout srvtimeout                  & (deprecated) & $\times$ & ---      & $\times$ & $\times$ \\
\hline
timeout tarpit                      &              & $\times$ & $\times$ & $\times$ & $\times$ \\
\hline
timeout tunnel                      &              & $\times$ & ---      & $\times$ & $\times$ \\
\hline
transparent                         & (deprecated) & $\times$ & ---      & $\times$ & $\times$ \\
\hline
unique-id-format                    &              & $\times$ & $\times$ & $\times$ & ---      \\
\hline
unique-id-header                    &              & $\times$ & $\times$ & $\times$ & ---      \\
\hline
use\_backend                        &              & ---      & $\times$ & $\times$ & ---      \\
\hline
use-server                          &              & ---      & ---      & $\times$ & $\times$ \\
\hline

\end{longtable}

\section{Alphabetically sorted keywords reference}

This section provides a description of each keyword and its usage.

\subsubsection[acl]{acl <aclname> <criterion> [flags] [operator] <value> ...}
\index{acl}
  Declare or complete an access list.
  
  \dflb{no}{yes}{yes}{yes}
  
  Example:
  \begin{verbatim}
        acl invalid_src  src          0.0.0.0/7 224.0.0.0/3
        acl invalid_src  src_port     0:1023
        acl local_dst    hdr(host) -i localhost
  \end{verbatim}

  See section 7 about ACL usage.

\subsubsection[appsession]{appsession <cookie> len <length> timeout <holdtime> [request-learn] [prefix] [mode <path-parameters|query-string>]}
\index{appsession}
  Define session stickiness on an existing application cookie.
  
  \dflb{no}{no}{yes}{yes}

  Arguments:
  \begin{description}

  \item[<cookie>]   this is the name of the cookie used by the application and which
               HAProxy will have to learn for each new session.

  \item[<length>]   this is the max number of characters that will be memorized and
               checked in each cookie value.

  \item[<holdtime>] this is the time after which the cookie will be removed from
               memory if unused. If no unit is specified, this time is in
               milliseconds.

  \item[request-learn]
               If this option is specified, then haproxy will be able to learn
               the cookie found in the request in case the server does not
               specify any in response. This is typically what happens with
               PHPSESSID cookies, or when haproxy's session expires before
               the application's session and the correct server is selected.
               It is recommended to specify this option to improve reliability.

  \item[prefix]     When this option is specified, haproxy will match on the cookie
               prefix (or URL parameter prefix). The appsession value is the
               data following this prefix.

               Example:
               
               \verb|appsession ASPSESSIONID len 64 timeout 3h prefix|

               This will match the cookie ASPSESSIONIDXXXX=XXXXX,
               the appsession value will be XXXX=XXXXX.

  \item[mode]       This option allows to change the URL parser mode.
               2 modes are currently supported:
               \begin{itemize}
               \item[-] path-parameters:
                 The parser looks for the appsession in the path parameters
                 part (each parameter is separated by a semi-colon), which is
                 convenient for JSESSIONID for example.
                 This is the default mode if the option is not set.
               \item[-] query-string:
                 In this mode, the parser will look for the appsession in the
                 query string.
              \end{itemize}
  \end{description}

  When an application cookie is defined in a backend, HAProxy will check when
  the server sets such a cookie, and will store its value in a table, and
  associate it with the server's identifier. Up to <length> characters from
  the value will be retained. On each connection, haproxy will look for this
  cookie both in the "Cookie:" headers, and as a URL parameter (depending on
  the mode used). If a known value is found, the client will be directed to the
  server associated with this value. Otherwise, the load balancing algorithm is
  applied. Cookies are automatically removed from memory when they have been
  unused for a duration longer than <holdtime>.

  The definition of an application cookie is limited to one per backend.

  \emph{Note:}
         Consider not using this feature in multi-process mode (nbproc > 1)
         unless you know what you do: memory is not shared between the
         processes, which can result in random behaviours.
         
  \verb|appsession JSESSIONID len 52 timeout 3h|

  See also : "cookie", "capture cookie", "balance", "stick", "stick-table",
             "ignore-persist", "nbproc" and "bind-process".

\subsubsection[backlog]{backlog <conns>}
\index{backlog}
  Give hints to the system about the approximate listen backlog desired size
 
  \dflb{yes}{yes}{yes}{no}
  
  Arguments:
  \begin{description}
  \item[<conns>]
         is the number of pending connections. Depending on the operating
              system, it may represent the number of already acknowledged
              connections, of non-acknowledged ones, or both.
  \end{description}

  In order to protect against SYN flood attacks, one solution is to increase
  the system's SYN backlog size. Depending on the system, sometimes it is just
  tunable via a system parameter, sometimes it is not adjustable at all, and
  sometimes the system relies on hints given by the application at the time of
  the listen() syscall. By default, HAProxy passes the frontend's maxconn value
  to the listen() syscall. On systems which can make use of this value, it can
  sometimes be useful to be able to specify a different value, hence this
  backlog parameter.

  On Linux 2.4, the parameter is ignored by the system. On Linux 2.6, it is
  used as a hint and the system accepts up to the smallest greater power of
  two, and never more than some limits (usually 32768).

  See also : "maxconn" and the target operating system's tuning guide.

\subsubsection[balance] {balance <algorithm> [ <arguments> ]}
\subsubsection*{balance url\_param <param> [check\_post [<max\_wait>]]}
\index{balance}
  Define the load balancing algorithm to be used in a backend.

  \dflb{yes}{no}{yes}{yes}
  
  Arguments:
    \paragraph*{<algorithm>}
                is the algorithm used to select a server when doing load
                balancing. This only applies when no persistence information
                is available, or when a connection is redispatched to another
                server. <algorithm> may be one of the following:
      \begin{description}
      \item[roundrobin]  Each server is used in turns, according to their weights.
                  This is the smoothest and fairest algorithm when the server's
                  processing time remains equally distributed. This algorithm
                  is dynamic, which means that server weights may be adjusted
                  on the fly for slow starts for instance. It is limited by
                  design to 4128 active servers per backend. Note that in some
                  large farms, when a server becomes up after having been down
                  for a very short time, it may sometimes take a few hundreds
                  requests for it to be re-integrated into the farm and start
                  receiving traffic. This is normal, though very rare. It is
                  indicated here in case you would have the chance to observe
                  it, so that you don't worry.

      \item[static-rr]   Each server is used in turns, according to their weights.
                  This algorithm is as similar to roundrobin except that it is
                  static, which means that changing a server's weight on the
                  fly will have no effect. On the other hand, it has no design
                  limitation on the number of servers, and when a server goes
                  up, it is always immediately reintroduced into the farm, once
                  the full map is recomputed. It also uses slightly less CPU to
                  run (around -1\%).

      \item[leastconn]   The server with the lowest number of connections receives the
                  connection. Round-robin is performed within groups of servers
                  of the same load to ensure that all servers will be used. Use
                  of this algorithm is recommended where very long sessions are
                  expected, such as LDAP, SQL, TSE, etc... but is not very well
                  suited for protocols using short sessions such as HTTP. This
                  algorithm is dynamic, which means that server weights may be
                  adjusted on the fly for slow starts for instance.

      \item[first]       The first server with available connection slots receives the
                  connection. The servers are choosen from the lowest numeric
                  identifier to the highest (see server parameter "id"), which
                  defaults to the server's position in the farm. Once a server
                  reaches its maxconn value, the next server is used. It does
                  not make sense to use this algorithm without setting maxconn.
                  The purpose of this algorithm is to always use the smallest
                  number of servers so that extra servers can be powered off
                  during non-intensive hours. This algorithm ignores the server
                  weight, and brings more benefit to long session such as RDP
                  or IMAP than HTTP, though it can be useful there too. In
                  order to use this algorithm efficiently, it is recommended
                  that a cloud controller regularly checks server usage to turn
                  them off when unused, and regularly checks backend queue to
                  turn new servers on when the queue inflates. Alternatively,
                  using "http-check send-state" may inform servers on the load.

      \item[source]      The source IP address is hashed and divided by the total
                  weight of the running servers to designate which server will
                  receive the request. This ensures that the same client IP
                  address will always reach the same server as long as no
                  server goes down or up. If the hash result changes due to the
                  number of running servers changing, many clients will be
                  directed to a different server. This algorithm is generally
                  used in TCP mode where no cookie may be inserted. It may also
                  be used on the Internet to provide a best-effort stickiness
                  to clients which refuse session cookies. This algorithm is
                  static by default, which means that changing a server's
                  weight on the fly will have no effect, but this can be
                  changed using "hash-type".

      \item[uri]         This algorithm hashes either the left part of the URI (before
                  the question mark) or the whole URI (if the "whole" parameter
                  is present) and divides the hash value by the total weight of
                  the running servers. The result designates which server will
                  receive the request. This ensures that the same URI will
                  always be directed to the same server as long as no server
                  goes up or down. This is used with proxy caches and
                  anti-virus proxies in order to maximize the cache hit rate.
                  Note that this algorithm may only be used in an HTTP backend.
                  This algorithm is static by default, which means that
                  changing a server's weight on the fly will have no effect,
                  but this can be changed using "hash-type".

                  This algorithm supports two optional parameters "len" and
                  "depth", both followed by a positive integer number. These
                  options may be helpful when it is needed to balance servers
                  based on the beginning of the URI only. The "len" parameter
                  indicates that the algorithm should only consider that many
                  characters at the beginning of the URI to compute the hash.
                  Note that having "len" set to 1 rarely makes sense since most
                  URIs start with a leading "/".

                  The "depth" parameter indicates the maximum directory depth
                  to be used to compute the hash. One level is counted for each
                  slash in the request. If both parameters are specified, the
                  evaluation stops when either is reached.

      \item[url\_param]   The URL parameter specified in argument will be looked up in
                  the query string of each HTTP GET request.

                  If the modifier "check\_post" is used, then an HTTP POST
                  request entity will be searched for the parameter argument,
                  when it is not found in a query string after a question mark
                  ('?') in the URL. Optionally, specify a number of octets to
                  wait for before attempting to search the message body. If the
                  entity can not be searched, then round robin is used for each
                  request. For instance, if your clients always send the LB
                  parameter in the first 128 bytes, then specify that. The
                  default is 48. The entity data will not be scanned until the
                  required number of octets have arrived at the gateway, this
                  is the minimum of: (default/max\_wait, Content-Length or first
                  chunk length). If Content-Length is missing or zero, it does
                  not need to wait for more data than the client promised to
                  send. When Content-Length is present and larger than
                  <max\_wait>, then waiting is limited to <max\_wait> and it is
                  assumed that this will be enough data to search for the
                  presence of the parameter. In the unlikely event that
                  Transfer-Encoding: chunked is used, only the first chunk is
                  scanned. Parameter values separated by a chunk boundary, may
                  be randomly balanced if at all.

                  If the parameter is found followed by an equal sign ('=') and
                  a value, then the value is hashed and divided by the total
                  weight of the running servers. The result designates which
                  server will receive the request.

                  This is used to track user identifiers in requests and ensure
                  that a same user ID will always be sent to the same server as
                  long as no server goes up or down. If no value is found or if
                  the parameter is not found, then a round robin algorithm is
                  applied. Note that this algorithm may only be used in an HTTP
                  backend. This algorithm is static by default, which means
                  that changing a server's weight on the fly will have no
                  effect, but this can be changed using "hash-type".

      \item[hdr(<name>)] The HTTP header <name> will be looked up in each HTTP
                  request. Just as with the equivalent ACL 'hdr()' function,
                  the header name in parenthesis is not case sensitive. If the
                  header is absent or if it does not contain any value, the
                  roundrobin algorithm is applied instead.

                  An optional 'use\_domain\_only' parameter is available, for
                  reducing the hash algorithm to the main domain part with some
                  specific headers such as 'Host'. For instance, in the Host
                  value "haproxy.1wt.eu", only "1wt" will be considered.

                  This algorithm is static by default, which means that
                  changing a server's weight on the fly will have no effect,
                  but this can be changed using "hash-type".

      \item[rdp-cookie(<name>)]
                  The RDP cookie <name> (or "mstshash" if omitted) will be
                  looked up and hashed for each incoming TCP request. Just as
                  with the equivalent ACL 'req\_rdp\_cookie()' function, the name
                  is not case-sensitive. This mechanism is useful as a degraded
                  persistence mode, as it makes it possible to always send the
                  same user (or the same session ID) to the same server. If the
                  cookie is not found, the normal roundrobin algorithm is
                  used instead.

                  Note that for this to work, the frontend must ensure that an
                  RDP cookie is already present in the request buffer. For this
                  you must use 'tcp-request content accept' rule combined with
                  a 'req\_rdp\_cookie\_cnt' ACL.

                  This algorithm is static by default, which means that
                  changing a server's weight on the fly will have no effect,
                  but this can be changed using "hash-type".

                  See also the rdp\_cookie pattern fetch function.
       \end{description}

    \paragraph*{<arguments>}
                is an optional list of arguments which may be needed by some
                algorithms. Right now, only "url\_param" and "uri" support an
                optional argument.

                \begin{verbatim}
                balance uri [len <len>] [depth <depth>]
                balance url_param <param> [check_post [<max_wait>]]
                \end{verbatim}

  The load balancing algorithm of a backend is set to roundrobin when no other
  algorithm, mode nor option have been set. The algorithm may only be set once
  for each backend.

  Examples:
  \begin{verbatim}
        balance roundrobin
        balance url_param userid
        balance url_param session_id check_post 64
        balance hdr(User-Agent)
        balance hdr(host)
        balance hdr(Host) use_domain_only
  \end{verbatim}

  \emph{Note:} the following caveats and limitations on using the "check\_post"
  extension with "url\_param" must be considered:
  
  \begin{itemize}
  \item[-] all POST requests are eligible for consideration, because there is no way
      to determine if the parameters will be found in the body or entity which
      may contain binary data. Therefore another method may be required to
      restrict consideration of POST requests that have no URL parameters in
      the body. (see acl reqideny http\_end)

  \item[-] using a <max\_wait> value larger than the request buffer size does not
      make sense and is useless. The buffer size is set at build time, and
      defaults to 16 kB.

  \item[-] Content-Encoding is not supported, the parameter search will probably
      fail; and load balancing will fall back to Round Robin.

  \item[-] Expect: 100-continue is not supported, load balancing will fall back to
      Round Robin.

  \item[-] Transfer-Encoding (RFC2616 3.6.1) is only supported in the first chunk.
      If the entire parameter value is not present in the first chunk, the
      selection of server is undefined (actually, defined by how little
      actually appeared in the first chunk).

  \item[-] This feature does not support generation of a 100, 411 or 501 response.

  \item[-] In some cases, requesting "check\_post" MAY attempt to scan the entire
      contents of a message body. Scanning normally terminates when linear
      white space or control characters are found, indicating the end of what
      might be a URL parameter list. This is probably not a concern with SGML
      type message bodies.
  \end{itemize}

  See also: "dispatch", "cookie", "appsession", "transparent", "hash-type" and
             "http\_proxy".

\subsubsection[bind]{bind [<address>]:<port\_range> [, ...]}
\subsubsection*{bind [<address>]:<port\_range> [, ...] interface <interface>}
\subsubsection*{bind [<address>]:<port\_range> [, ...] mss <maxseg>}
\subsubsection*{bind [<address>]:<port\_range> [, ...] transparent}
\subsubsection*{bind [<address>]:<port\_range> [, ...] id <id>}
\subsubsection*{bind [<address>]:<port\_range> [, ...] name <name>}
\subsubsection*{bind [<address>]:<port\_range> [, ...] defer-accept}
\subsubsection*{bind [<address>]:<port\_range> [, ...] accept-proxy}
\subsubsection*{bind /<path> [, ...]}
\subsubsection*{bind /<path> [, ...] mode <mode>}
\subsubsection*{bind /<path> [, ...] [ user <user> | uid <uid> ]}
\subsubsection*{bind /<path> [, ...] [ group <user> | gid <gid> ]}

\index{bind}
  Define one or several listening addresses and/or ports in a frontend.
    
  \dflb{no}{yes}{yes}{no}
                                
  Arguments:
  \begin{description}
  \item[<address>]     is optional and can be a host name, an IPv4 address, an IPv6
                  address, or '*'. It designates the address the frontend will
                  listen on. If unset, all IPv4 addresses of the system will be
                  listened on. The same will apply for '*' or the system's
                  special address "0.0.0.0". The IPv6 equivalent is '::'.

  \item[<port\_range>]  is either a unique TCP port, or a port range for which the
                  proxy will accept connections for the IP address specified
                  above. The port is mandatory for TCP listeners. Note that in
                  the case of an IPv6 address, the port is always the number
                  after the last colon (':'). A range can either be:
                  
                  \begin{itemize}
                  \item[-] a numerical port (ex: '80')
                  \item[-] a dash-delimited ports range explicitly stating the lower
                     and upper bounds (ex: '2000-2100') which are included in
                     the range.
                  \end{itemize}

                  Particular care must be taken against port ranges, because
                  every <address:port> couple consumes one socket (= a file
                  descriptor), so it's easy to consume lots of descriptors
                  with a simple range, and to run out of sockets. Also, each
                  <address:port> couple must be used only once among all
                  instances running on a same system. Please note that binding
                  to ports lower than 1024 generally require particular
                  privileges to start the program, which are independant of
                  the 'uid' parameter.

  \item[<path>]        is a UNIX socket path beginning with a slash ('/'). This is
                  alternative to the TCP listening port. Haproxy will then
                  receive UNIX connections on the socket located at this place.
                  The path must begin with a slash and by default is absolute.
                  It can be relative to the prefix defined by "unix-bind" in
                  the global section. Note that the total length of the prefix
                  followed by the socket path cannot exceed some system limits
                  for UNIX sockets, which commonly are set to 107 characters.

  \item[<interface>]   is an optional physical interface name. This is currently
                  only supported on Linux. The interface must be a physical
                  interface, not an aliased interface. When specified, all
                  addresses on the same line will only be accepted if the
                  incoming packet physically come through the designated
                  interface. It is also possible to bind multiple frontends to
                  the same address if they are bound to different interfaces.
                  Note that binding to a physical interface requires root
                  privileges. This parameter is only compatible with TCP
                  sockets.

  \item[<maxseg>]      is an optional TCP Maximum Segment Size (MSS) value to be
                  advertised on incoming connections. This can be used to force
                  a lower MSS for certain specific ports, for instance for
                  connections passing through a VPN. Note that this relies on a
                  kernel feature which is theorically supported under Linux but
                  was buggy in all versions prior to 2.6.28. It may or may not
                  work on other operating systems. It may also not change the
                  advertised value but change the effective size of outgoing
                  segments. The commonly advertised value on Ethernet networks
                  is 1460 = 1500(MTU) - 40(IP+TCP). If this value is positive,
                  it will be used as the advertised MSS. If it is negative, it
                  will indicate by how much to reduce the incoming connection's
                  advertised MSS for outgoing segments. This parameter is only
                  compatible with TCP sockets.

  \item[<id>]          is a persistent value for socket ID. Must be positive and
                  unique in the proxy. An unused value will automatically be
                  assigned if unset. Can only be used when defining only a
                  single socket.

  \item[<name>]        is an optional name provided for stats

  \item[<mode>]        is the octal mode used to define access permissions on the
                  UNIX socket. It can also be set by default in the global
                  section's "unix-bind" statement. Note that some platforms
                  simply ignore this.

  \item[<user>]        is the name of user that will be marked owner of the UNIX
                  socket.  It can also be set by default in the global
                  section's "unix-bind" statement. Note that some platforms
                  simply ignore this.

  \item[<group>]       is the name of a group that will be used to create the UNIX
                  socket. It can also be set by default in the global section's
                  "unix-bind" statement. Note that some platforms simply ignore
                  this.

  \item[<uid>]         is the uid of user that will be marked owner of the UNIX
                  socket. It can also be set by default in the global section's
                  "unix-bind" statement. Note that some platforms simply ignore
                  this.

  \item[<gid>]         is the gid of a group that will be used to create the UNIX
                  socket. It can also be set by default in the global section's
                  "unix-bind" statement. Note that some platforms simply ignore
                  this.

  \item[transparent]   is an optional keyword which is supported only on certain
                  Linux kernels. It indicates that the addresses will be bound
                  even if they do not belong to the local machine. Any packet
                  targeting any of these addresses will be caught just as if
                  the address was locally configured. This normally requires
                  that IP forwarding is enabled. Caution! do not use this with
                  the default address '*', as it would redirect any traffic for
                  the specified port. This keyword is available only when
                  HAProxy is built with USE\_LINUX\_TPROXY=1. This parameter is
                  only compatible with TCP sockets.

  \item[defer-accept]  is an optional keyword which is supported only on certain
                  Linux kernels. It states that a connection will only be
                  accepted once some data arrive on it, or at worst after the
                  first retransmit. This should be used only on protocols for
                  which the client talks first (eg: HTTP). It can slightly
                  improve performance by ensuring that most of the request is
                  already available when the connection is accepted. On the
                  other hand, it will not be able to detect connections which
                  don't talk. It is important to note that this option is
                  broken in all kernels up to 2.6.31, as the connection is
                  never accepted until the client talks. This can cause issues
                  with front firewalls which would see an established
                  connection while the proxy will only see it in SYN\_RECV.

  \item[accept-proxy]  is an optional keyword which enforces use of the PROXY
                  protocol over any connection accepted by this listener. The
                  PROXY protocol dictates the layer 3/4 addresses of the
                  incoming connection to be used everywhere an address is used,
                  with the only exception of "tcp-request connection" rules
                  which will only see the real connection address. Logs will
                  reflect the addresses indicated in the protocol, unless it is
                  violated, in which case the real address will still be used.
                  This keyword combined with support from external components
                  can be used as an efficient and reliable alternative to the
                  X-Forwarded-For mechanism which is not always reliable and
                  not even always usable.
  \end{description}

  It is possible to specify a list of address:port combinations delimited by
  commas. The frontend will then listen on all of these addresses. There is no
  fixed limit to the number of addresses and ports which can be listened on in
  a frontend, as well as there is no limit to the number of "bind" statements
  in a frontend.

  Example:
  \begin{verbatim}
        listen http_proxy
            bind :80,:443
            bind 10.0.0.1:10080,10.0.0.1:10443
            bind /var/run/ssl-frontend.sock user root mode 600 accept-proxy
  \end{verbatim}

  See also: "source", "option forwardfor", "unix-bind" and the PROXY protocol
             documentation.

\subsubsection[bind-process]{bind-process [ all | odd | even | <number 1-32> ] ...}
\index{bind-process}
  Limit visibility of an instance to a certain set of processes numbers.
  
  \dflb{yes}{yes}{yes}{yes}
                                 
  Arguments:
  \begin{description}
  \item[all]   All process will see this instance. This is the default. It
                  may be used to override a default value.

  \item[odd]   This instance will be enabled on processes 1,3,5,...31. This
                  option may be combined with other numbers.

  \item[even]  This instance will be enabled on processes 2,4,6,...32. This
                  option may be combined with other numbers. Do not use it
                  with less than 2 processes otherwise some instances might be
                  missing from all processes.

  \item[number] The instance will be enabled on this process number, between
                  1 and 32. You must be careful not to reference a process
                  number greater than the configured global.nbproc, otherwise
                  some instances might be missing from all processes.
  \end{description}

  This keyword limits binding of certain instances to certain processes. This
  is useful in order not to have too many processes listening to the same
  ports. For instance, on a dual-core machine, it might make sense to set
  'nbproc 2' in the global section, then distributes the listeners among 'odd'
  and 'even' instances.

  At the moment, it is not possible to reference more than 32 processes using
  this keyword, but this should be more than enough for most setups. Please
  note that 'all' really means all processes and is not limited to the first
  32.

  If some backends are referenced by frontends bound to other processes, the
  backend automatically inherits the frontend's processes.

  Example:
  \begin{verbatim}
        listen app_ip1
            bind 10.0.0.1:80
            bind-process odd

        listen app_ip2
            bind 10.0.0.2:80
            bind-process even

        listen management
            bind 10.0.0.3:80
            bind-process 1 2 3 4
  \end{verbatim}

  See also: "nbproc" in global section.

\subsubsection[block]{block { if | unless } <condition>}
\index{block}
  Block a layer 7 request if/unless a condition is matched

  \dflb{no}{yes}{yes}{yes}

  The HTTP request will be blocked very early in the layer 7 processing
  if/unless <condition> is matched. A 403 error will be returned if the request
  is blocked. The condition has to reference ACLs (see section 7). This is
  typically used to deny access to certain sensitive resources if some
  conditions are met or not met. There is no fixed limit to the number of
  "block" statements per instance.

  Example:
  \begin{verbatim}
        acl invalid_src  src          0.0.0.0/7 224.0.0.0/3
        acl invalid_src  src_port     0:1023
        acl local_dst    hdr(host) -i localhost
        block if invalid_src || local_dst
  \end{verbatim}

  See section 7 about ACL usage.

\subsubsection[capture cookie]{capture cookie <name> len <length>}
  Capture and log a cookie in the request and in the response.

  \dflb{no}{yes}{yes}{no}
                                  
  Arguments:
  \begin{description} 
  \item[<name>]    is the beginning of the name of the cookie to capture. In order
              to match the exact name, simply suffix the name with an equal
              sign ('='). The full name will appear in the logs, which is
              useful with application servers which adjust both the cookie name
              and value (eg: ASPSESSIONXXXXX).

  \item[<length>]  is the maximum number of characters to report in the logs, which
              include the cookie name, the equal sign and the value, all in the
              standard "name=value" form. The string will be truncated on the
              right if it exceeds <length>.
  \end{description}

  Only the first cookie is captured. Both the "cookie" request headers and the
  "set-cookie" response headers are monitored. This is particularly useful to
  check for application bugs causing session crossing or stealing between
  users, because generally the user's cookies can only change on a login page.

  When the cookie was not presented by the client, the associated log column
  will report "-". When a request does not cause a cookie to be assigned by the
  server, a "-" is reported in the response column.

  The capture is performed in the frontend only because it is necessary that
  the log format does not change for a given frontend depending on the
  backends. This may change in the future. Note that there can be only one
  "capture cookie" statement in a frontend. The maximum capture length is
  configured in the sources by default to 64 characters. It is not possible to
  specify a capture in a "defaults" section.

  Example:

  \verb|capture cookie ASPSESSION len 32|

  See also: "capture request header", "capture response header" as well as
            section 8 about logging.

\subsubsection[capture request header]{capture request header <name> len <length>}
\index{capture request header}
  
  Capture and log the first occurrence of the specified request header.

  \dflb{no}{yes}{yes}{no}
  
  Arguments:
  \begin{description}
  \item[<name>]    is the name of the header to capture. The header names are not
              case-sensitive, but it is a common practice to write them as they
              appear in the requests, with the first letter of each word in
              upper case. The header name will not appear in the logs, only the
              value is reported, but the position in the logs is respected.

  \item[<length>]  is the maximum number of characters to extract from the value and
              report in the logs. The string will be truncated on the right if
              it exceeds <length>.
  \end{description}

  Only the first value of the last occurrence of the header is captured. The
  value will be added to the logs between braces ('\{\}'). If multiple headers
  are captured, they will be delimited by a vertical bar ('|') and will appear
  in the same order they were declared in the configuration. Non-existent
  headers will be logged just as an empty string. Common uses for request
  header captures include the "Host" field in virtual hosting environments, the
  "Content-length" when uploads are supported, "User-agent" to quickly
  differentiate between real users and robots, and "X-Forwarded-For" in proxied
  environments to find where the request came from.

  Note that when capturing headers such as "User-agent", some spaces may be
  logged, making the log analysis more difficult. Thus be careful about what
  you log if you know your log parser is not smart enough to rely on the
  braces.

  There is no limit to the number of captured request headers, but each capture
  is limited to 64 characters. In order to keep log format consistent for a
  same frontend, header captures can only be declared in a frontend. It is not
  possible to specify a capture in a "defaults" section.

  Example:
  \begin{verbatim}
        capture request header Host len 15
        capture request header X-Forwarded-For len 15
        capture request header Referrer len 15
  \end{verbatim}

  See also: "capture cookie", "capture response header" as well as section 8
             about logging.

\subsubsection[capture response header]{capture response header <name> len <length>}
\index{capture response header}
  Capture and log the first occurrence of the specified response header.
  
  \dflb{no}{yes}{yes}{no}

  Arguments:
  \begin{description}
  \item[<name>]    is the name of the header to capture. The header names are not
              case-sensitive, but it is a common practice to write them as they
              appear in the response, with the first letter of each word in
              upper case. The header name will not appear in the logs, only the
              value is reported, but the position in the logs is respected.

  \item[<length>]  is the maximum number of characters to extract from the value and
              report in the logs. The string will be truncated on the right if
              it exceeds <length>.
  \end{description}

  Only the first value of the last occurrence of the header is captured. The
  result will be added to the logs between braces ('\{\}') after the captured
  request headers. If multiple headers are captured, they will be delimited by
  a vertical bar ('|') and will appear in the same order they were declared in
  the configuration. Non-existent headers will be logged just as an empty
  string. Common uses for response header captures include the "Content-length"
  header which indicates how many bytes are expected to be returned, the
  "Location" header to track redirections.

  There is no limit to the number of captured response headers, but each
  capture is limited to 64 characters. In order to keep log format consistent
  for a same frontend, header captures can only be declared in a frontend. It
  is not possible to specify a capture in a "defaults" section.

  Example:
  \begin{verbatim}
        capture response header Content-length len 9
        capture response header Location len 15
  \end{verbatim}

  See also: "capture cookie", "capture request header" as well as section 8
             about logging.

\subsubsection[clitimeout]{clitimeout <timeout> (deprecated)}
\index{clitimeout}
  Set the maximum inactivity time on the client side.
  
  \dflb{yes}{yes}{yes}{no}
  
  Arguments:
  \begin{description}
  \item[<timeout>] is the timeout value is specified in milliseconds by default, but
              can be in any other unit if the number is suffixed by the unit,
              as explained at the top of this document.  
  \end{description}

  The inactivity timeout applies when the client is expected to acknowledge or
  send data. In HTTP mode, this timeout is particularly important to consider
  during the first phase, when the client sends the request, and during the
  response while it is reading data sent by the server. The value is specified
  in milliseconds by default, but can be in any other unit if the number is
  suffixed by the unit, as specified at the top of this document. In TCP mode
  (and to a lesser extent, in HTTP mode), it is highly recommended that the
  client timeout remains equal to the server timeout in order to avoid complex
  situations to debug. It is a good practice to cover one or several TCP packet
  losses by specifying timeouts that are slightly above multiples of 3 seconds
  (eg: 4 or 5 seconds).

  This parameter is specific to frontends, but can be specified once for all in
  "defaults" sections. This is in fact one of the easiest solutions not to
  forget about it. An unspecified timeout results in an infinite timeout, which
  is not recommended. Such a usage is accepted and works but reports a warning
  during startup because it may results in accumulation of expired sessions in
  the system if the system's timeouts are not configured either.

  This parameter is provided for compatibility but is currently deprecated.
  Please use "timeout client" instead.

  See also: "timeout client", "timeout http-request", "timeout server", and
             "srvtimeout".

\subsubsection[contimeout]{contimeout <timeout> (deprecated)}
\index{contimeout}
  Set the maximum time to wait for a connection attempt to a server to succeed.
  \dflb{yes}{no}{yes}{yes}

  Arguments:
  \begin{description}
  \item[<timeout>] is the timeout value is specified in milliseconds by default, but
              can be in any other unit if the number is suffixed by the unit,
              as explained at the top of this document.  
  \end{description}

  If the server is located on the same LAN as haproxy, the connection should be
  immediate (less than a few milliseconds). Anyway, it is a good practice to
  cover one or several TCP packet losses by specifying timeouts that are
  slightly above multiples of 3 seconds (eg: 4 or 5 seconds). By default, the
  connect timeout also presets the queue timeout to the same value if this one
  has not been specified. Historically, the contimeout was also used to set the
  tarpit timeout in a listen section, which is not possible in a pure frontend.

  This parameter is specific to backends, but can be specified once for all in
  "defaults" sections. This is in fact one of the easiest solutions not to
  forget about it. An unspecified timeout results in an infinite timeout, which
  is not recommended. Such a usage is accepted and works but reports a warning
  during startup because it may results in accumulation of failed sessions in
  the system if the system's timeouts are not configured either.

  This parameter is provided for backwards compatibility but is currently
  deprecated. Please use "timeout connect", "timeout queue" or "timeout tarpit"
  instead.

  See also: "timeout connect", "timeout queue", "timeout tarpit",
             "timeout server", "contimeout".


\subsubsection[cookie]{cookie <name> [ rewrite | insert | prefix ] [ indirect ]  [ nocache ] [ postonly ] [ preserve ] [ httponly ] [ secure ] [ domain <domain> ]* [ maxidle <idle> ] [ maxlife <life> ]}
\index{cookie}
  Enable cookie-based persistence in a backend.
  
  \dflb{yes}{no}{yes}{yes}
  
  Arguments:
  \begin{description}
  \item[<name>]    is the name of the cookie which will be monitored, modified or
              inserted in order to bring persistence. This cookie is sent to
              the client via a "Set-Cookie" header in the response, and is
              brought back by the client in a "Cookie" header in all requests.
              Special care should be taken to choose a name which does not
              conflict with any likely application cookie. Also, if the same
              backends are subject to be used by the same clients (eg:
              HTTP/HTTPS), care should be taken to use different cookie names
              between all backends if persistence between them is not desired.

  \item[rewrite]   This keyword indicates that the cookie will be provided by the
              server and that haproxy will have to modify its value to set the
              server's identifier in it. This mode is handy when the management
              of complex combinations of "Set-cookie" and "Cache-control"
              headers is left to the application. The application can then
              decide whether or not it is appropriate to emit a persistence
              cookie. Since all responses should be monitored, this mode only
              works in HTTP close mode. Unless the application behaviour is
              very complex and/or broken, it is advised not to start with this
              mode for new deployments. This keyword is incompatible with
              "insert" and "prefix".

  \item[insert]    This keyword indicates that the persistence cookie will have to
              be inserted by haproxy in server responses if the client did not

              already have a cookie that would have permitted it to access this
              server. When used without the "preserve" option, if the server
              emits a cookie with the same name, it will be remove before
              processing.  For this reason, this mode can be used to upgrade
              existing configurations running in the "rewrite" mode. The cookie
              will only be a session cookie and will not be stored on the
              client's disk. By default, unless the "indirect" option is added,
              the server will see the cookies emitted by the client. Due to
              caching effects, it is generally wise to add the "nocache" or
              "postonly" keywords (see below). The "insert" keyword is not
              compatible with "rewrite" and "prefix".

  \item[prefix]    This keyword indicates that instead of relying on a dedicated
              cookie for the persistence, an existing one will be completed.
              This may be needed in some specific environments where the client
              does not support more than one single cookie and the application
              already needs it. In this case, whenever the server sets a cookie
              named <name>, it will be prefixed with the server's identifier
              and a delimiter. The prefix will be removed from all client
              requests so that the server still finds the cookie it emitted.
              Since all requests and responses are subject to being modified,
              this mode requires the HTTP close mode. The "prefix" keyword is
              not compatible with "rewrite" and "insert". Note: it is highly
              recommended not to use "indirect" with "prefix", otherwise server
              cookie updates would not be sent to clients.

  \item[indirect]  When this option is specified, no cookie will be emitted to a
              client which already has a valid one for the server which has
              processed the request. If the server sets such a cookie itself,
              it will be removed, unless the "preserve" option is also set. In
              "insert" mode, this will additionally remove cookies from the
              requests transmitted to the server, making the persistence
              mechanism totally transparent from an application point of view.
              Note: it is highly recommended not to use "indirect" with
              "prefix", otherwise server cookie updates would not be sent to
              clients.

  \item[nocache]   This option is recommended in conjunction with the insert mode
              when there is a cache between the client and HAProxy, as it
              ensures that a cacheable response will be tagged non-cacheable if
              a cookie needs to be inserted. This is important because if all
              persistence cookies are added on a cacheable home page for
              instance, then all customers will then fetch the page from an
              outer cache and will all share the same persistence cookie,
              leading to one server receiving much more traffic than others.
              See also the "insert" and "postonly" options.

  \item[postonly]  This option ensures that cookie insertion will only be performed
              on responses to POST requests. It is an alternative to the
              "nocache" option, because POST responses are not cacheable, so
              this ensures that the persistence cookie will never get cached.
              Since most sites do not need any sort of persistence before the
              first POST which generally is a login request, this is a very
              efficient method to optimize caching without risking to find a
              persistence cookie in the cache.
              See also the "insert" and "nocache" options.

  \item[preserve]  This option may only be used with "insert" and/or "indirect". It
              allows the server to emit the persistence cookie itself. In this
              case, if a cookie is found in the response, haproxy will leave it
              untouched. This is useful in order to end persistence after a
              logout request for instance. For this, the server just has to
              emit a cookie with an invalid value (eg: empty) or with a date in
              the past. By combining this mechanism with the "disable-on-404"
              check option, it is possible to perform a completely graceful
              shutdown because users will definitely leave the server after
              they logout.

  \item[httponly]  This option tells haproxy to add an "HttpOnly" cookie attribute
              when a cookie is inserted. This attribute is used so that a
              user agent doesn't share the cookie with non-HTTP components.
              Please check RFC6265 for more information on this attribute.

  \item[secure]    This option tells haproxy to add a "Secure" cookie attribute when
              a cookie is inserted. This attribute is used so that a user agent
              never emits this cookie over non-secure channels, which means
              that a cookie learned with this flag will be presented only over
              SSL/TLS connections. Please check RFC6265 for more information on
              this attribute.

  \item[domain]    This option allows to specify the domain at which a cookie is
              inserted. It requires exactly one parameter: a valid domain
              name. If the domain begins with a dot, the browser is allowed to
              use it for any host ending with that name. It is also possible to
              specify several domain names by invoking this option multiple
              times. Some browsers might have small limits on the number of
              domains, so be careful when doing that. For the record, sending
              10 domains to MSIE 6 or Firefox 2 works as expected.

  \item[maxidle]   This option allows inserted cookies to be ignored after some idle
              time. It only works with insert-mode cookies. When a cookie is
              sent to the client, the date this cookie was emitted is sent too.
              Upon further presentations of this cookie, if the date is older
              than the delay indicated by the parameter (in seconds), it will
              be ignored. Otherwise, it will be refreshed if needed when the
              response is sent to the client. This is particularly useful to
              prevent users who never close their browsers from remaining for
              too long on the same server (eg: after a farm size change). When
              this option is set and a cookie has no date, it is always
              accepted, but gets refreshed in the response. This maintains the
              ability for admins to access their sites. Cookies that have a
              date in the future further than 24 hours are ignored. Doing so
              lets admins fix timezone issues without risking kicking users off
              the site.

  \item[maxlife]   This option allows inserted cookies to be ignored after some life
              time, whether they're in use or not. It only works with insert
              mode cookies. When a cookie is first sent to the client, the date
              this cookie was emitted is sent too. Upon further presentations
              of this cookie, if the date is older than the delay indicated by
              the parameter (in seconds), it will be ignored. If the cookie in
              the request has no date, it is accepted and a date will be set.
              Cookies that have a date in the future further than 24 hours are
              ignored. Doing so lets admins fix timezone issues without risking
              kicking users off the site. Contrary to maxidle, this value is
              not refreshed, only the first visit date counts. Both maxidle and
              maxlife may be used at the time. This is particularly useful to
              prevent users who never close their browsers from remaining for
              too long on the same server (eg: after a farm size change). This
              is stronger than the maxidle method in that it forces a
              redispatch after some absolute delay.
  \end{description}

  There can be only one persistence cookie per HTTP backend, and it can be
  declared in a defaults section. The value of the cookie will be the value
  indicated after the "cookie" keyword in a "server" statement. If no cookie
  is declared for a given server, the cookie is not set.

  Examples:
  \begin{verbatim}
        cookie JSESSIONID prefix
        cookie SRV insert indirect nocache
        cookie SRV insert postonly indirect
        cookie SRV insert indirect nocache maxidle 30m maxlife 8h
  \end{verbatim}

  See also: "appsession", "balance source", "capture cookie", "server"
             and "ignore-persist".

\subsubsection[default-server]{default-server [param*]}
\index{default-server}
  Change default options for a server in a backend
  
  \dflb{yes}{no}{yes}{yes}
  
  Arguments:
  \begin{description}
  \item[<param*>]  is a list of parameters for this server. The "default-server"
              keyword accepts an important number of options and has a complete
              section dedicated to it. Please refer to section 5 for more
              details.
  \end{description}

  Example:
  \begin{verbatim}
        default-server inter 1000 weight 13
  \end{verbatim}

  See also: "server" and section 5 about server options

\subsubsection[default\_backend]{default\_backend <backend>}
\index{default\_backend}
  Specify the backend to use when no "use\_backend" rule has been matched.
  
  \dflb{yes}{yes}{yes}{no}
  
  Arguments:
  \begin{description}
  \item[<backend>] is the name of the backend to use.
  \end{description}

  When doing content-switching between frontend and backends using the
  "use\_backend" keyword, it is often useful to indicate which backend will be
  used when no rule has matched. It generally is the dynamic backend which
  will catch all undetermined requests.

  Example:
  \begin{verbatim}
        use_backend     dynamic  if  url_dyn
        use_backend     static   if  url_css url_img extension_img
        default_backend dynamic
  \end{verbatim}

  See also: "use\_backend", "reqsetbe", "reqisetbe"

\subsubsection[disabled]{disabled}
\index{disabled}
  Disable a proxy, frontend or backend.
  
  \dflb{yes}{yes}{yes}{yes}

  Arguments: none

  The "disabled" keyword is used to disable an instance, mainly in order to
  liberate a listening port or to temporarily disable a service. The instance
  will still be created and its configuration will be checked, but it will be
  created in the "stopped" state and will appear as such in the statistics. It
  will not receive any traffic nor will it send any health-checks or logs. It
  is possible to disable many instances at once by adding the "disabled"
  keyword in a "defaults" section.

  See also: "enabled"

\subsubsection[dispatch]{dispatch <address>:<port>}
\index{dispatch}
  Set a default server address
  
  \dflb{no}{no}{yes}{yes}
  
  Arguments:
  \begin{description}
  \item[<address>] is the IPv4 address of the default server. Alternatively, a
              resolvable hostname is supported, but this name will be resolved
              during start-up.

  \item[<ports>]   is a mandatory port specification. All connections will be sent
              to this port, and it is not permitted to use port offsets as is
              possible with normal servers.
  \end{description}

  The "dispatch" keyword designates a default server for use when no other
  server can take the connection. In the past it was used to forward non
  persistent connections to an auxiliary load balancer. Due to its simple
  syntax, it has also been used for simple TCP relays. It is recommended not to
  use it for more clarity, and to use the "server" directive instead.

  See also: "server"

\subsubsection[enabled]{enabled}
\index{enabled}
  Enable a proxy, frontend or backend.
  
  \dflb{yes}{yes}{yes}{yes}
  
  Arguments: none

  The "enabled" keyword is used to explicitly enable an instance, when the
  defaults has been set to "disabled". This is very rarely used.

  See also: "disabled"

\subsubsection[errorfile]{errorfile <code> <file>}
\index{errorfile}
  Return a file contents instead of errors generated by HAProxy
  
  \dflb{yes}{yes}{yes}{yes}

  Arguments:
  \begin{description}
  \item[<code>]    is the HTTP status code. Currently, HAProxy is capable of
              generating codes 200, 400, 403, 408, 500, 502, 503, and 504.

   \item[<file>]    designates a file containing the full HTTP response. It is
              recommended to follow the common practice of appending ".http" to
              the filename so that people do not confuse the response with HTML
              error pages, and to use absolute paths, since files are read
              before any chroot is performed.
  \end{description}

  It is important to understand that this keyword is not meant to rewrite
  errors returned by the server, but errors detected and returned by HAProxy.
  This is why the list of supported errors is limited to a small set.

  Code 200 is emitted in response to requests matching a "monitor-uri" rule.

  The files are returned verbatim on the TCP socket. This allows any trick such
  as redirections to another URL or site, as well as tricks to clean cookies,
  force enable or disable caching, etc... The package provides default error
  files returning the same contents as default errors.

  The files should not exceed the configured buffer size (BUFSIZE), which
  generally is 8 or 16 kB, otherwise they will be truncated. It is also wise
  not to put any reference to local contents (eg: images) in order to avoid
  loops between the client and HAProxy when all servers are down, causing an
  error to be returned instead of an image. For better HTTP compliance, it is
  recommended that all header lines end with CR-LF and not LF alone.

  The files are read at the same time as the configuration and kept in memory.
  For this reason, the errors continue to be returned even when the process is
  chrooted, and no file change is considered while the process is running. A
  simple method for developing those files consists in associating them to the
  403 status code and interrogating a blocked URL.

  See also: "errorloc", "errorloc302", "errorloc303"

  Example:
  \begin{verbatim}
        errorfile 400 /etc/haproxy/errorfiles/400badreq.http
        errorfile 403 /etc/haproxy/errorfiles/403forbid.http
        errorfile 503 /etc/haproxy/errorfiles/503sorry.http
  \end{verbatim}

\subsubsection[errorloc]{errorloc <code> <url>}
\subsubsection[errorloc302]{errorloc302 <code> <url>}
\index{errorloc}
\index{errorloc302}
  Return an HTTP redirection to a URL instead of errors generated by HAProxy
  
  \dflb{yes}{yes}{yes}{yes}
  
  Arguments:
  \begin{description}
  \item[<code>]    is the HTTP status code. Currently, HAProxy is capable of
              generating codes 200, 400, 403, 408, 500, 502, 503, and 504.

  \item[<url>]     it is the exact contents of the "Location" header. It may contain
              either a relative URI to an error page hosted on the same site,
              or an absolute URI designating an error page on another site.
              Special care should be given to relative URIs to avoid redirect
              loops if the URI itself may generate the same error (eg: 500).
  \end{description}

  It is important to understand that this keyword is not meant to rewrite
  errors returned by the server, but errors detected and returned by HAProxy.
  This is why the list of supported errors is limited to a small set.

  Code 200 is emitted in response to requests matching a "monitor-uri" rule.

  Note that both keyword return the HTTP 302 status code, which tells the
  client to fetch the designated URL using the same HTTP method. This can be
  quite problematic in case of non-GET methods such as POST, because the URL
  sent to the client might not be allowed for something other than GET. To
  workaround this problem, please use "errorloc303" which send the HTTP 303
  status code, indicating to the client that the URL must be fetched with a GET
  request.

  See also: "errorfile", "errorloc303"

\subsubsection[errorloc303]{errorloc303 <code> <url>}
\index{errorloc303}
  Return an HTTP redirection to a URL instead of errors generated by HAProxy
  
  \dflb{yes}{yes}{yes}{yes}
  
  Arguments:
  \begin{description}
  \item[<code>]    is the HTTP status code. Currently, HAProxy is capable of
              generating codes 400, 403, 408, 500, 502, 503, and 504.

  \item[<url>]     it is the exact contents of the "Location" header. It may contain
              either a relative URI to an error page hosted on the same site,
              or an absolute URI designating an error page on another site.
              Special care should be given to relative URIs to avoid redirect
              loops if the URI itself may generate the same error (eg: 500).
  \end{description}

  It is important to understand that this keyword is not meant to rewrite
  errors returned by the server, but errors detected and returned by HAProxy.
  This is why the list of supported errors is limited to a small set.

  Code 200 is emitted in response to requests matching a "monitor-uri" rule.

  Note that both keyword return the HTTP 303 status code, which tells the
  client to fetch the designated URL using the same HTTP GET method. This
  solves the usual problems associated with "errorloc" and the 302 code. It is
  possible that some very old browsers designed before HTTP/1.1 do not support
  it, but no such problem has been reported till now.

  See also: "errorfile", "errorloc", "errorloc302"

\subsubsection[force-persist]{force-persist { if | unless } <condition>}
\index{force-persist}
  Declare a condition to force persistence on down servers
  
  \dflb{no}{yes}{yes}{yes}

  By default, requests are not dispatched to down servers. It is possible to
  force this using "option persist", but it is unconditional and redispatches
  to a valid server if "option redispatch" is set. That leaves with very little
  possibilities to force some requests to reach a server which is artificially
  marked down for maintenance operations.

  The "force-persist" statement allows one to declare various ACL-based
  conditions which, when met, will cause a request to ignore the down status of
  a server and still try to connect to it. That makes it possible to start a
  server, still replying an error to the health checks, and run a specially
  configured browser to test the service. Among the handy methods, one could
  use a specific source IP address, or a specific cookie. The cookie also has
  the advantage that it can easily be added/removed on the browser from a test
  page. Once the service is validated, it is then possible to open the service
  to the world by returning a valid response to health checks.

  The forced persistence is enabled when an "if" condition is met, or unless an
  "unless" condition is met. The final redispatch is always disabled when this
  is used.

  See also: "option redispatch", "ignore-persist", "persist",
             and section 7 about ACL usage.

\subsubsection[fullconn]{fullconn <conns>}
\index{fullconn}
  Specify at what backend load the servers will reach their maxconn
  
  \dflb{yes}{no}{yes}{yes}
  
  Arguments:
  \begin{description}
  \item[<conns>]   is the number of connections on the backend which will make the
              servers use the maximal number of connections.
  \end{description}

  When a server has a "maxconn" parameter specified, it means that its number
  of concurrent connections will never go higher. Additionally, if it has a
  "minconn" parameter, it indicates a dynamic limit following the backend's
  load. The server will then always accept at least <minconn> connections,
  never more than <maxconn>, and the limit will be on the ramp between both
  values when the backend has less than <conns> concurrent connections. This
  makes it possible to limit the load on the servers during normal loads, but
  push it further for important loads without overloading the servers during
  exceptional loads.

  Since it's hard to get this value right, haproxy automatically sets it to
  10\% of the sum of the maxconns of all frontends that may branch to this
  backend. That way it's safe to leave it unset.

  Example:
  
  \begin{verbatim}
     # The servers will accept between 100 and 1000 concurrent connections each
     # and the maximum of 1000 will be reached when the backend reaches 10000
     # connections.
     backend dynamic
        fullconn   10000
        server     srv1   dyn1:80 minconn 100 maxconn 1000
        server     srv2   dyn2:80 minconn 100 maxconn 1000
  \end{verbatim}

  See also: "maxconn", "server"

\subsubsection[grace]{grace <time>}
\index{grace}
  Maintain a proxy operational for some time after a soft stop
  
  \dflb{yes}{yes}{yes}{yes}
  
  Arguments:
  \begin{description}
  \item[<time>]    is the time (by default in milliseconds) for which the instance
              will remain operational with the frontend sockets still listening
              when a soft-stop is received via the SIGUSR1 signal.
  \end{description}

  This may be used to ensure that the services disappear in a certain order.
  This was designed so that frontends which are dedicated to monitoring by an
  external equipment fail immediately while other ones remain up for the time
  needed by the equipment to detect the failure.

  Note that currently, there is very little benefit in using this parameter,
  and it may in fact complicate the soft-reconfiguration process more than
  simplify it.

\subsubsection[hash-type]{hash-type <method>}
\index{hash-type}
  Specify a method to use for mapping hashes to servers
  
  \dflb{yes}{no}{yes}{yes}
  
  Arguments:
  \begin{description}
  \item[map-based]   the hash table is a static array containing all alive servers.
                The hashes will be very smooth, will consider weights, but will
                be static in that weight changes while a server is up will be
                ignored. This means that there will be no slow start. Also,
                since a server is selected by its position in the array, most
                mappings are changed when the server count changes. This means
                that when a server goes up or down, or when a server is added
                to a farm, most connections will be redistributed to different
                servers. This can be inconvenient with caches for instance.

  \item[avalanche]   this mechanism uses the default map-based hashing described
                above but applies a full avalanche hash before performing the
                mapping. The result is a slightly less smooth hash for most
                situations, but the hash becomes better than pure map-based
                hashes when the number of servers is a multiple of the size of
                the input set. When using URI hash with a number of servers
                multiple of 64, it's desirable to change the hash type to
                this value.

  \item[consistent]  the hash table is a tree filled with many occurrences of each
                server. The hash key is looked up in the tree and the closest
                server is chosen. This hash is dynamic, it supports changing
                weights while the servers are up, so it is compatible with the
                slow start feature. It has the advantage that when a server
                goes up or down, only its associations are moved. When a server
                is added to the farm, only a few part of the mappings are
                redistributed, making it an ideal algorithm for caches.
                However, due to its principle, the algorithm will never be very
                smooth and it may sometimes be necessary to adjust a server's
                weight or its ID to get a more balanced distribution. In order
                to get the same distribution on multiple load balancers, it is
                important that all servers have the same IDs.
  \end{description}

  The default hash type is "map-based" and is recommended for most usages.

  See also: "balance", "server"

\subsubsection[http-check disable-on-404]{http-check disable-on-404}
\index{http-check disable-on-404}
  Enable a maintenance mode upon HTTP/404 response to health-checks
  
  \dflb{yes}{no}{yes}{yes}
  
  Arguments: none

  When this option is set, a server which returns an HTTP code 404 will be
  excluded from further load-balancing, but will still receive persistent
  connections. This provides a very convenient method for Web administrators
  to perform a graceful shutdown of their servers. It is also important to note
  that a server which is detected as failed while it was in this mode will not
  generate an alert, just a notice. If the server responds 2xx or 3xx again, it
  will immediately be reinserted into the farm. The status on the stats page
  reports "NOLB" for a server in this mode. It is important to note that this
  option only works in conjunction with the "httpchk" option. If this option
  is used with "http-check expect", then it has precedence over it so that 404
  responses will still be considered as soft-stop.

  See also : "option httpchk", "http-check expect"

\subsubsection[http-check expect]{http-check expect [!] <match> <pattern>}
\index{http-check expect}
  Make HTTP health checks consider response contents or specific status codes
  
  \dflb{yes}{no}{yes}{yes}
  
  Arguments:
  \begin{description}
  \item[<match>]   is a keyword indicating how to look for a specific pattern in the
              response. The keyword may be one of "status", "rstatus",
              "string", or "rstring". The keyword may be preceded by an
              exclamation mark ("!") to negate the match. Spaces are allowed
              between the exclamation mark and the keyword. See below for more
              details on the supported keywords.

  \item[<pattern>] is the pattern to look for. It may be a string or a regular
              expression. If the pattern contains spaces, they must be escaped
              with the usual backslash ('\textbackslash').
  \end{description}

  By default, "option httpchk" considers that response statuses 2xx and 3xx
  are valid, and that others are invalid. When "http-check expect" is used,
  it defines what is considered valid or invalid. Only one "http-check"
  statement is supported in a backend. If a server fails to respond or times
  out, the check obviously fails.
  
  The available matches are:
  \begin{description}
  \item[status <string>] test the exact string match for the HTTP status code.
                      A health check respose will be considered valid if the
                      response's status code is exactly this string. If the
                      "status" keyword is prefixed with "!", then the response
                      will be considered invalid if the status code matches.

    \item[rstatus <regex>] test a regular expression for the HTTP status code.
                      A health check respose will be considered valid if the
                      response's status code matches the expression. If the
                      "rstatus" keyword is prefixed with "!", then the response
                      will be considered invalid if the status code matches.
                      This is mostly used to check for multiple codes.

    \item[string <string>] test the exact string match in the HTTP response body.
                      A health check respose will be considered valid if the
                      response's body contains this exact string. If the
                      "string" keyword is prefixed with "!", then the response
                      will be considered invalid if the body contains this
                      string. This can be used to look for a mandatory word at
                      the end of a dynamic page, or to detect a failure when a
                      specific error appears on the check page (eg: a stack
                      trace).

    \item[rstring <regex>] test a regular expression on the HTTP response body.
                      A health check respose will be considered valid if the
                      response's body matches this expression. If the "rstring"
                      keyword is prefixed with "!", then the response will be
                      considered invalid if the body matches the expression.
                      This can be used to look for a mandatory word at the end
                      of a dynamic page, or to detect a failure when a specific
                      error appears on the check page (eg: a stack trace).
  \end{description}

  It is important to note that the responses will be limited to a certain size
  defined by the global "tune.chksize" option, which defaults to 16384 bytes.
  Thus, too large responses may not contain the mandatory pattern when using
  "string" or "rstring". If a large response is absolutely required, it is
  possible to change the default max size by setting the global variable.
  However, it is worth keeping in mind that parsing very large responses can
  waste some CPU cycles, especially when regular expressions are used, and that
  it is always better to focus the checks on smaller resources.

  Last, if "http-check expect" is combined with "http-check disable-on-404",
  then this last one has precedence when the server responds with 404.

  Examples:
  \begin{verbatim}
         # only accept status 200 as valid
         http-check expect status 200

         # consider SQL errors as errors
         http-check expect ! string SQL\ Error

         # consider status 5xx only as errors
         http-check expect ! rstatus ^5

         # check that we have a correct hexadecimal tag before /html
         http-check expect rstring <!--tag:[0-9a-f]*</html>
  \end{verbatim}

  See also: "option httpchk", "http-check disable-on-404"

\subsubsection[http-check send-state]{http-check send-state}
\index{http-check send-state}
  Enable emission of a state header with HTTP health checks
  
  \dflb{yes}{no}{yes}{yes}
  
  Arguments: none

  When this option is set, haproxy will systematically send a special header
  "X-Haproxy-Server-State" with a list of parameters indicating to each server
  how they are seen by haproxy. This can be used for instance when a server is
  manipulated without access to haproxy and the operator needs to know whether
  haproxy still sees it up or not, or if the server is the last one in a farm.

  The header is composed of fields delimited by semi-colons, the first of which
  is a word ("UP", "DOWN", "NOLB"), possibly followed by a number of valid
  checks on the total number before transition, just as appears in the stats
  interface. Next headers are in the form "<variable>=<value>", indicating in
  no specific order some values available in the stats interface:
  
  \begin{itemize}
  \item[-] a variable "name", containing the name of the backend followed by a slash
      ("/") then the name of the server. This can be used when a server is
      checked in multiple backends.

  \item[-] a variable "node" containing the name of the haproxy node, as set in the
      global "node" variable, otherwise the system's hostname if unspecified.

  \item[-] a variable "weight" indicating the weight of the server, a slash ("/")
      and the total weight of the farm (just counting usable servers). This
      helps to know if other servers are available to handle the load when this
      one fails.

  \item[-] a variable "scur" indicating the current number of concurrent connections
      on the server, followed by a slash ("/") then the total number of
      connections on all servers of the same backend.

  \item[-] a variable "qcur" indicating the current number of requests in the
      server's queue.
  \end{itemize}

  Example of a header received by the application server:
  \begin{verbatim}
    >>>  X-Haproxy-Server-State: UP 2/3; name=bck/srv2; node=lb1; weight=1/2; \
           scur=13/22; qcur=0
  \end{verbatim}

  See also: "option httpchk", "http-check disable-on-404"

\subsubsection[http-request]{http-request \{ allow | deny | auth [realm <realm>] \} [ \{ if | unless \} <condition> ]}
\index{http-request}
  Access control for Layer 7 requests

  \dflb{no}{yes}{yes}{yes}                       

  These set of options allow to fine control access to a
  frontend/listen/backend. Each option may be followed by if/unless and acl.
  First option with matched condition (or option without condition) is final.
  For "deny" a 403 error will be returned, for "allow" normal processing is
  performed, for "auth" a 401/407 error code is returned so the client
  should be asked to enter a username and password.

  There is no fixed limit to the number of http-request statements per
  instance.

  Example:
  \begin{verbatim}
        acl nagios src 192.168.129.3
        acl local_net src 192.168.0.0/16
        acl auth_ok http_auth(L1)

        http-request allow if nagios
        http-request allow if local_net auth_ok
        http-request auth realm Gimme if local_net auth_ok
        http-request deny
    \end{verbatim}

  Example:
\begin{verbatim}
        acl auth_ok http_auth_group(L1) G1

        http-request auth unless auth_ok
\end{verbatim}

  See also : "stats http-request", section 3.4 about userlists and section 7
             about ACL usage.

\subsubsection[http-send-name-header]{http-send-name-header [<header>]}
\index{http-send-name-header}
  Add the server name to a request. Use the header string given by <header>

  \dflb{yes}{no}{yes}{yes}

  Arguments:
  
\begin{description}
\item[ <header>]  The header string to use to send the server name
\end{description}

  The "http-send-name-header" statement causes the name of the target
  server to be added to the headers of an HTTP request.  The name
  is added with the header string proved.

  See also : "server"

\subsubsection[id]{id <value>}
\index{id}
  Set a persistent ID to a proxy.
  
  \dflb{no}{yes}{yes}{yes}
  
  Arguments:
\begin{description}
\item[<id>] ID of proxy 
\end{description}

  Set a persistent ID for the proxy. This ID must be unique and positive.
  An unused ID will automatically be assigned if unset. The first assigned
  value will be 1. This ID is currently only returned in statistics.

\subsubsection[ignore-persist]{ignore-persist \{ if | unless \} <condition>}
\index{ignore-persist}
  Declare a condition to ignore persistence

  \dflb{no}{yes}{yes}{yes}

  By default, when cookie persistence is enabled, every requests containing
  the cookie are unconditionally persistent (assuming the target server is up
  and running).

  The "ignore-persist" statement allows one to declare various ACL-based
  conditions which, when met, will cause a request to ignore persistence.
  This is sometimes useful to load balance requests for static files, which
  oftenly don't require persistence. This can also be used to fully disable
  persistence for a specific User-Agent (for example, some web crawler bots).

  Combined with "appsession", it can also help reduce HAProxy memory usage, as
  the appsession table won't grow if persistence is ignored.

  The persistence is ignored when an "if" condition is met, or unless an
  "unless" condition is met.

  See also: "force-persist", "cookie", and section 7 about ACL usage.

\subsubsection[log global]{log global}
\subsubsection[log]{log <address> <facility> [<level> [<minlevel>]]}
\subsubsection[no log]{no log}

\index{log global}
\index{log}
\index{no log}

  Enable per-instance logging of events and traffic.
  
  \dflb{yes}{yes}{yes}{yes}

  Prefix:
  
\begin{description}
\item[no] should be used when the logger list must be flushed. For example,
               if you don't want to inherit from the default logger list. This
               prefix does not allow arguments.
\end{description}

  Arguments:

\begin{description}
\item[global] should be used when the instance's logging parameters are the
               same as the global ones. This is the most common usage. "global"
               replaces <address>, <facility> and <level> with those of the log
               entries found in the "global" section. Only one "log global"
               statement may be used per instance, and this form takes no other
               parameter.

\item[<address>]  indicates where to send the logs. It takes the same format as
               for the "global" section's logs, and can be one of:
               \begin{itemize}
               \item[-] An IPv4 address optionally followed by a colon (':') and a UDP
                 port. If no port is specified, 514 is used by default (the
                 standard syslog port).
               \item[-] An IPv6 address followed by a colon (':') and optionally a UDP
                 port. If no port is specified, 514 is used by default (the
                 standard syslog port).
               \item[-] A filesystem path to a UNIX domain socket, keeping in mind
                 considerations for chroot (be sure the path is accessible
                 inside the chroot) and uid/gid (be sure the path is
                 appropriately writeable).
                 \end{itemize}

\item[<facility>] must be one of the 24 standard syslog facilities:
\index{Log facility}
  \vspace{5mm}
  \textbf{
  \begin{tabular}{llllllll}
    kern   & user   & mail   & daemon & auth   & syslog & lpr    & news \\
    uucp   & cron   & auth2  & ftp    & ntp    & audit  & alert  & cron2 \\
    local0 & local1 & local2 & local3 & local4 & local5 & local6 & local7
  \end{tabular}
  }
  \vspace{5mm}

\item[<level>]    is optional and can be specified to filter outgoing messages. By
               default, all messages are sent. If a level is specified, only
               messages with a severity at least as important as this level
               will be sent. An optional minimum level can be specified. If it
               is set, logs emitted with a more severe level than this one will
               be capped to this level. This is used to avoid sending "emerg"
               messages on all terminals on some default syslog configurations.
\index{Log level}
               Eight levels are known:
               
  \vspace{3mm}
  \textbf {
  \begin{tabular}{llllllll}
          emerg & alert & crit & err & warning & notice & info & debug
  \end{tabular}
  }

\end{description}

  It is important to keep in mind that it is the frontend which decides what to
  log from a connection, and that in case of content switching, the log entries
  from the backend will be ignored. Connections are logged at level "info".

  However, backend log declaration define how and where servers status changes
  will be logged. Level "notice" will be used to indicate a server going up,
  "warning" will be used for termination signals and definitive service
  termination, and "alert" will be used for when a server goes down.

  \emph{Note:} According to RFC3164, messages are truncated to 1024 bytes before
         being emitted.

  Example:

\begin{verbatim}
    log global
    log 127.0.0.1:514 local0 notice         # only send important events
    log 127.0.0.1:514 local0 notice notice  # same but limit output level
\end{verbatim}

\subsubsection[log-format]{log-format <string>}
\index{log-format}
   Allows you to custom a log line.

   See also: Custom Log Format (8.2.4)

\subsubsection[maxconn]{maxconn <conns>}
\index{maxconn}
  Fix the maximum number of concurrent connections on a frontend

\dflb{yes}{yes}{yes}{no}

  Arguments:
\begin{description}
\item[<conns>]   is the maximum number of concurrent connections the frontend will
              accept to serve. Excess connections will be queued by the system
              in the socket's listen queue and will be served once a connection
              closes.

\end{description}

  If the system supports it, it can be useful on big sites to raise this limit
  very high so that haproxy manages connection queues, instead of leaving the
  clients with unanswered connection attempts. This value should not exceed the
  global maxconn. Also, keep in mind that a connection contains two buffers
  of 8kB each, as well as some other data resulting in about 17 kB of RAM being
  consumed per established connection. That means that a medium system equipped
  with 1GB of RAM can withstand around 40000-50000 concurrent connections if
  properly tuned.

  Also, when <conns> is set to large values, it is possible that the servers
  are not sized to accept such loads, and for this reason it is generally wise
  to assign them some reasonable connection limits.

  See also: "server", global section's "maxconn", "fullconn"

\subsubsection[mode]{mode \{ tcp | http | health \}}
\index{mode}
  Set the running mode or protocol of the instance

\dflb{yes}{yes}{yes}{yes}

  Arguments :
\begin{description}
\item[tcp]       The instance will work in pure TCP mode. A full-duplex connection
              will be established between clients and servers, and no layer 7
              examination will be performed. This is the default mode. It
              should be used for SSL, SSH, SMTP, ...

\item[http]      The instance will work in HTTP mode. The client request will be
              analyzed in depth before connecting to any server. Any request
              which is not RFC-compliant will be rejected. Layer 7 filtering,
              processing and switching will be possible. This is the mode which
              brings HAProxy most of its value.

\item[health]    The instance will work in "health" mode. It will just reply "OK"
              to incoming connections and close the connection. Nothing will be
              logged. This mode is used to reply to external components health
              checks. This mode is deprecated and should not be used anymore as
              it is possible to do the same and even better by combining TCP or
              HTTP modes with the "monitor" keyword.
\end{description}

  When doing content switching, it is mandatory that the frontend and the
  backend are in the same mode (generally HTTP), otherwise the configuration
  will be refused.

  Example:
\begin{verbatim}
     defaults http_instances
         mode http
\end{verbatim}

  See also: "monitor", "monitor-net"

\subsubsection[monitor fail]{monitor fail \{ if | unless \} <condition>}
\index{monitor fail}
  Add a condition to report a failure to a monitor HTTP request.

\dflb{no}{yes}{yes}{no}

  Arguments:
\begin{description}
\item[if <cond>]     the monitor request will fail if the condition is satisfied,
                  and will succeed otherwise. The condition should describe a
                  combined test which must induce a failure if all conditions
                  are met, for instance a low number of servers both in a
                  backend and its backup.

\item[unless <cond>] the monitor request will succeed only if the condition is
                  satisfied, and will fail otherwise. Such a condition may be
                  based on a test on the presence of a minimum number of active
                  servers in a list of backends.
\end{description}

  This statement adds a condition which can force the response to a monitor
  request to report a failure. By default, when an external component queries
  the URI dedicated to monitoring, a 200 response is returned. When one of the
  conditions above is met, haproxy will return 503 instead of 200. This is
  very useful to report a site failure to an external component which may base
  routing advertisements between multiple sites on the availability reported by
  haproxy. In this case, one would rely on an ACL involving the "nbsrv"
  criterion. Note that "monitor fail" only works in HTTP mode. Both status
  messages may be tweaked using "errorfile" or "errorloc" if needed.

  Example:
 \begin{verbatim}
    frontend www
        mode http
        acl site_dead nbsrv(dynamic) lt 2
        acl site_dead nbsrv(static)  lt 2
        monitor-uri   /site_alive
        monitor fail  if site_dead
\end{verbatim}

  See also: "monitor-net", "monitor-uri", "errorfile", "errorloc"

\subsubsection[monitor-net]{monitor-net <source>}
\index{monitor-net}
  Declare a source network which is limited to monitor requests

\dflb{yes}{yes}{yes}{no}

  Arguments:
\begin{description}
\item[ <source>]  is the source IPv4 address or network which will only be able to
              get monitor responses to any request. It can be either an IPv4
              address, a host name, or an address followed by a slash ('/')
              followed by a mask.
\end{description}

  In TCP mode, any connection coming from a source matching <source> will cause
  the connection to be immediately closed without any log. This allows another
  equipment to probe the port and verify that it is still listening, without
  forwarding the connection to a remote server.

  In HTTP mode, a connection coming from a source matching <source> will be
  accepted, the following response will be sent without waiting for a request,
  then the connection will be closed : "HTTP/1.0 200 OK". This is normally
  enough for any front-end HTTP probe to detect that the service is UP and
  running without forwarding the request to a backend server.

  Monitor requests are processed very early. It is not possible to block nor
  divert them using ACLs. They cannot be logged either, and it is the intended
  purpose. They are only used to report HAProxy's health to an upper component,
  nothing more. Right now, it is not possible to set failure conditions on
  requests caught by "monitor-net".

  Last, please note that only one "monitor-net" statement can be specified in
  a frontend. If more than one is found, only the last one will be considered.

  Example:
\begin{verbatim}
    # addresses .252 and .253 are just probing us.
    frontend www
        monitor-net 192.168.0.252/31
\end{verbatim}

  See also : "monitor fail", "monitor-uri"

\subsubsection[monitor-uri]{monitor-uri <uri>}
\index{monitor-uri}
  Intercept a URI used by external components' monitor requests

\dflb{yes}{yes}{yes}{no}

  Arguments:
\begin{description}
\item[<uri>]     is the exact URI which we want to intercept to return HAProxy's
              health status instead of forwarding the request.
\end{description}

  When an HTTP request referencing <uri> will be received on a frontend,
  HAProxy will not forward it nor log it, but instead will return either
  "HTTP/1.0 200 OK" or "HTTP/1.0 503 Service unavailable", depending on failure
  conditions defined with "monitor fail". This is normally enough for any
  front-end HTTP probe to detect that the service is UP and running without
  forwarding the request to a backend server. Note that the HTTP method, the
  version and all headers are ignored, but the request must at least be valid
  at the HTTP level. This keyword may only be used with an HTTP-mode frontend.

  Monitor requests are processed very early. It is not possible to block nor
  divert them using ACLs. They cannot be logged either, and it is the intended
  purpose. They are only used to report HAProxy's health to an upper component,
  nothing more. However, it is possible to add any number of conditions using
  "monitor fail" and ACLs so that the result can be adjusted to whatever check
  can be imagined (most often the number of available servers in a backend).

  Example :
\begin{verbatim}
    # Use /haproxy_test to report haproxy's status
    frontend www
        mode http
        monitor-uri /haproxy_test
\end{verbatim}

  See also: "monitor fail", "monitor-net"

\subsubsection{option abortonclose}
\subsubsection{no option abortonclose}
\index{abortonclose}
  
  Enable or disable early dropping of aborted requests pending in queues.

   \dflb{yes}{no}{yes}{yes}
   
  Arguments: none

  In presence of very high loads, the servers will take some time to respond.
  The per-instance connection queue will inflate, and the response time will
  increase respective to the size of the queue times the average per-session
  response time. When clients will wait for more than a few seconds, they will
  often hit the "STOP" button on their browser, leaving a useless request in
  the queue, and slowing down other users, and the servers as well, because the
  request will eventually be served, then aborted at the first error
  encountered while delivering the response.

  As there is no way to distinguish between a full STOP and a simple output
  close on the client side, HTTP agents should be conservative and consider
  that the client might only have closed its output channel while waiting for
  the response. However, this introduces risks of congestion when lots of users
  do the same, and is completely useless nowadays because probably no client at
  all will close the session while waiting for the response. Some HTTP agents
  support this behaviour (Squid, Apache, HAProxy), and others do not (TUX, most
  hardware-based load balancers). So the probability for a closed input channel
  to represent a user hitting the "STOP" button is close to 100%, and the risk
  of being the single component to break rare but valid traffic is extremely
  low, which adds to the temptation to be able to abort a session early while
  still not served and not pollute the servers.

  In HAProxy, the user can choose the desired behaviour using the option
  "abortonclose". By default (without the option) the behaviour is HTTP
  compliant and aborted requests will be served. But when the option is
  specified, a session with an incoming channel closed will be aborted while
  it is still possible, either pending in the queue for a connection slot, or
  during the connection establishment if the server has not yet acknowledged
  the connection request. This considerably reduces the queue size and the load
  on saturated servers when users are tempted to click on STOP, which in turn
  reduces the response time for other users.

  If this option has been enabled in a "defaults" section, it can be disabled
  in a specific instance by prepending the "no" keyword before it.

  See also: "timeout queue" and server's "maxconn" and "maxqueue" parameters

\subsubsection{option accept-invalid-http-request}
\subsubsection{no option accept-invalid-http-request}
\index{accept-invalid-http-request}
  Enable or disable relaxing of HTTP request parsing

\dflb{yes}{yes}{yes}{no}

  Arguments: none

  By default, HAProxy complies with RFC2616 in terms of message parsing. This
  means that invalid characters in header names are not permitted and cause an
  error to be returned to the client. This is the desired behavior as such
  forbidden characters are essentially used to build attacks exploiting server
  weaknesses, and bypass security filtering. Sometimes, a buggy browser or
  server will emit invalid header names for whatever reason (configuration,
  implementation) and the issue will not be immediately fixed. In such a case,
  it is possible to relax HAProxy's header name parser to accept any character
  even if that does not make sense, by specifying this option. Similarly, the
  list of characters allowed to appear in a URI is well defined by RFC3986, and
  chars 0-31, 32 (space), 34 (\verb|'"'|), 60 (\verb|<|), 62 (\verb|>|), 92 (\verb|'\'|), 94 (\verb|'^'|), 96
  (\verb|'`'|), 123 (\verb|'{'|), 124 (\verb:'|':), 125 (\verb|'}'|), 127 (delete) and anything above are
  not allowed at all. Haproxy always blocks a number of them (0..32, 127). The
  remaining ones are blocked by default unless this option is enabled.

  This option should never be enabled by default as it hides application bugs
  and open security breaches. It should only be deployed after a problem has
  been confirmed.

  When this option is enabled, erroneous header names will still be accepted in
  requests, but the complete request will be captured in order to permit later
  analysis using the "show errors" request on the UNIX stats socket. Similarly,
  requests containing invalid chars in the URI part will be logged. Doing this
  also helps confirming that the issue has been solved.

  If this option has been enabled in a "defaults" section, it can be disabled
  in a specific instance by prepending the "no" keyword before it.

  See also: "option accept-invalid-http-response" and "show errors" on the
             stats socket.

\subsubsection{option accept-invalid-http-response}
\subsubsection{no option accept-invalid-http-response}

\index{accept-invalid-http-response}
  
  Enable or disable relaxing of HTTP response parsing

\dflb{yes}{no}{yes}{yes}

  Arguments: none

  By default, HAProxy complies with RFC2616 in terms of message parsing. This
  means that invalid characters in header names are not permitted and cause an
  error to be returned to the client. This is the desired behavior as such
  forbidden characters are essentially used to build attacks exploiting server
  weaknesses, and bypass security filtering. Sometimes, a buggy browser or
  server will emit invalid header names for whatever reason (configuration,
  implementation) and the issue will not be immediately fixed. In such a case,
  it is possible to relax HAProxy's header name parser to accept any character
  even if that does not make sense, by specifying this option.

  This option should never be enabled by default as it hides application bugs
  and open security breaches. It should only be deployed after a problem has
  been confirmed.

  When this option is enabled, erroneous header names will still be accepted in
  responses, but the complete response will be captured in order to permit
  later analysis using the "show errors" request on the UNIX stats socket.
  Doing this also helps confirming that the issue has been solved.

  If this option has been enabled in a "defaults" section, it can be disabled
  in a specific instance by prepending the "no" keyword before it.

  See also: "option accept-invalid-http-request" and "show errors" on the
             stats socket.

\subsubsection{option allbackups}
\subsubsection{no option allbackups}

\index{allbackups}

  Use either all backup servers at a time or only the first one

\dflb{yes}{no}{yes}{yes}

  Arguments: none

  By default, the first operational backup server gets all traffic when normal
  servers are all down. Sometimes, it may be preferred to use multiple backups
  at once, because one will not be enough. When "option allbackups" is enabled,
  the load balancing will be performed among all backup servers when all normal
  ones are unavailable. The same load balancing algorithm will be used and the
  servers' weights will be respected. Thus, there will not be any priority
  order between the backup servers anymore.

  This option is mostly used with static server farms dedicated to return a
  "sorry" page when an application is completely offline.

  If this option has been enabled in a "defaults" section, it can be disabled
  in a specific instance by prepending the "no" keyword before it.

\subsubsection{option checkcache}
\subsubsection{no option checkcache}

\index{checkcache}

  Analyze all server responses and block requests with cacheable cookies

\dflb{yes}{no}{yes}{yes}

  Arguments : none

  Some high-level frameworks set application cookies everywhere and do not
  always let enough control to the developer to manage how the responses should
  be cached. When a session cookie is returned on a cacheable object, there is a
  high risk of session crossing or stealing between users traversing the same
  caches. In some situations, it is better to block the response than to let
  some sensitive session information go in the wild.

  The option "checkcache" enables deep inspection of all server responses for
  strict compliance with HTTP specification in terms of cacheability. It
  carefully checks "Cache-control", "Pragma" and "Set-cookie" headers in server
  response to check if there's a risk of caching a cookie on a client-side
  proxy. When this option is enabled, the only responses which can be delivered
  to the client are:

\begin{itemize}
\item[-] all those without "Set-Cookie" header ;
\item[-] all those with a return code other than 200, 203, 206, 300, 301, 410,
      provided that the server has not set a "Cache-control: public" header ;
\item[-] all those that come from a POST request, provided that the server has not
      set a 'Cache-Control: public' header ;
\item[-] those with a 'Pragma: no-cache' header
\item[-] those with a 'Cache-control: private' header
\item[-] those with a 'Cache-control: no-store' header
\item[-] those with a 'Cache-control: max-age=0' header
\item[-] those with a 'Cache-control: s-maxage=0' header
\item[-] those with a 'Cache-control: no-cache' header
\item[-] those with a 'Cache-control: no-cache="set-cookie"' header
\item[-] those with a 'Cache-control: no-cache="set-cookie,' header
      (allowing other fields after set-cookie)
\end{itemize}

  If a response doesn't respect these requirements, then it will be blocked
  just as if it was from an "rspdeny" filter, with an "HTTP 502 bad gateway".
  The session state shows "PH--" meaning that the proxy blocked the response
  during headers processing. Additionally, an alert will be sent in the logs so
  that admins are informed that there's something to be fixed.

  Due to the high impact on the application, the application should be tested
  in depth with the option enabled before going to production. It is also a
  good practice to always activate it during tests, even if it is not used in
  production, as it will report potentially dangerous application behaviors.

  If this option has been enabled in a "defaults" section, it can be disabled
  in a specific instance by prepending the "no" keyword before it.

\subsubsection{option clitcpka}
\subsubsection{no option clitcpka}

\index{clitcpka}

  Enable or disable the sending of TCP keepalive packets on the client side

\dflb{yes}{yes}{yes}{no}
                                 
  Arguments: none

  When there is a firewall or any session-aware component between a client and
  a server, and when the protocol involves very long sessions with long idle
  periods (eg: remote desktops), there is a risk that one of the intermediate
  components decides to expire a session which has remained idle for too long.

  Enabling socket-level TCP keep-alives makes the system regularly send packets
  to the other end of the connection, leaving it active. The delay between
  keep-alive probes is controlled by the system only and depends both on the
  operating system and its tuning parameters.

  It is important to understand that keep-alive packets are neither emitted nor
  received at the application level. It is only the network stacks which sees
  them. For this reason, even if one side of the proxy already uses keep-alives
  to maintain its connection alive, those keep-alive packets will not be
  forwarded to the other side of the proxy.

  Please note that this has nothing to do with HTTP keep-alive.

  Using option "clitcpka" enables the emission of TCP keep-alive probes on the
  client side of a connection, which should help when session expirations are
  noticed between HAProxy and a client.

  If this option has been enabled in a "defaults" section, it can be disabled
  in a specific instance by prepending the "no" keyword before it.

  See also: "option srvtcpka", "option tcpka"

\subsubsection{option contstats}

\index{contstats}

  Enable continuous traffic statistics updates

\dflb{yes}{yes}{yes}{no}

  Arguments: none

  By default, counters used for statistics calculation are incremented
  only when a session finishes. It works quite well when serving small
  objects, but with big ones (for example large images or archives) or
  with A/V streaming, a graph generated from haproxy counters looks like
  a hedgehog. With this option enabled counters get incremented continuously,
  during a whole session. Recounting touches a hotpath directly so
  it is not enabled by default, as it has small performance impact (~0.5\%).

\subsubsection{option dontlog-normal}
\subsubsection{no option dontlog-normal}

\index{dontlog-normal}

  Enable or disable logging of normal, successful connections

\dflb{yes}{yes}{yes}{no}

  Arguments: none

  There are large sites dealing with several thousand connections per second
  and for which logging is a major pain. Some of them are even forced to turn
  logs off and cannot debug production issues. Setting this option ensures that
  normal connections, those which experience no error, no timeout, no retry nor
  redispatch, will not be logged. This leaves disk space for anomalies. In HTTP
  mode, the response status code is checked and return codes 5xx will still be
  logged.

  It is strongly discouraged to use this option as most of the time, the key to
  complex issues is in the normal logs which will not be logged here. If you
  need to separate logs, see the "log-separate-errors" option instead.

  See also: "log", "dontlognull", "log-separate-errors" and section 8 about
             logging.

\subsubsection{option dontlognull}
\subsubsection{no option dontlognull}

\index{dontlognull}

  Enable or disable logging of null connections

\dflb{yes}{yes}{yes}{no}

  Arguments: none

  In certain environments, there are components which will regularly connect to
  various systems to ensure that they are still alive. It can be the case from
  another load balancer as well as from monitoring systems. By default, even a
  simple port probe or scan will produce a log. If those connections pollute
  the logs too much, it is possible to enable option "dontlognull" to indicate
  that a connection on which no data has been transferred will not be logged,
  which typically corresponds to those probes.

  It is generally recommended not to use this option in uncontrolled
  environments (eg: internet), otherwise scans and other malicious activities
  would not be logged.

  If this option has been enabled in a "defaults" section, it can be disabled
  in a specific instance by prepending the "no" keyword before it.

  See also: "log", "monitor-net", "monitor-uri" and section 8 about logging.

\subsubsection{option forceclose}
\subsubsection{no option forceclose}

\index{forceclose}

  Enable or disable active connection closing after response is transferred.

\dflb{yes}{yes}{yes}{yes}

  Arguments: none

  Some HTTP servers do not necessarily close the connections when they receive
  the "Connection: close" set by "option httpclose", and if the client does not
  close either, then the connection remains open till the timeout expires. This
  causes high number of simultaneous connections on the servers and shows high
  global session times in the logs.

  When this happens, it is possible to use "option forceclose". It will
  actively close the outgoing server channel as soon as the server has finished
  to respond. This option implicitly enables the "httpclose" option. Note that
  this option also enables the parsing of the full request and response, which
  means we can close the connection to the server very quickly, releasing some
  resources earlier than with httpclose.

  This option may also be combined with "option http-pretend-keepalive", which
  will disable sending of the "Connection: close" header, but will still cause
  the connection to be closed once the whole response is received.

  If this option has been enabled in a "defaults" section, it can be disabled
  in a specific instance by prepending the "no" keyword before it.

  See also: "option httpclose" and "option http-pretend-keepalive"


\subsubsection[option forwardfor]{option forwardfor [ except <network> ] [ header <name> ] [ if-none ]}

\index{forwardfor}

  Enable insertion of the X-Forwarded-For header to requests sent to servers

\dflb{yes}{yes}{yes}{yes}

  Arguments:
\begin{description}
\item[<network>] is an optional argument used to disable this option for sources
              matching <network>
\item[<name>]    an optional argument to specify a different "X-Forwarded-For"
              header name.
\end{description}

  Since HAProxy works in reverse-proxy mode, the servers see its IP address as
  their client address. This is sometimes annoying when the client's IP address
  is expected in server logs. To solve this problem, the well-known HTTP header
  "X-Forwarded-For" may be added by HAProxy to all requests sent to the server.
  This header contains a value representing the client's IP address. Since this
  header is always appended at the end of the existing header list, the server
  must be configured to always use the last occurrence of this header only. See
  the server's manual to find how to enable use of this standard header. Note
  that only the last occurrence of the header must be used, since it is really
  possible that the client has already brought one.

  The keyword "header" may be used to supply a different header name to replace
  the default "X-Forwarded-For". This can be useful where you might already
  have a "X-Forwarded-For" header from a different application (eg: stunnel),
  and you need preserve it. Also if your backend server doesn't use the
  "X-Forwarded-For" header and requires different one (eg: Zeus Web Servers
  require "X-Cluster-Client-IP").

  Sometimes, a same HAProxy instance may be shared between a direct client
  access and a reverse-proxy access (for instance when an SSL reverse-proxy is
  used to decrypt HTTPS traffic). It is possible to disable the addition of the
  header for a known source address or network by adding the "except" keyword
  followed by the network address. In this case, any source IP matching the
  network will not cause an addition of this header. Most common uses are with
  private networks or 127.0.0.1.

  Alternatively, the keyword "if-none" states that the header will only be
  added if it is not present. This should only be used in perfectly trusted
  environment, as this might cause a security issue if headers reaching haproxy
  are under the control of the end-user.

  This option may be specified either in the frontend or in the backend. If at
  least one of them uses it, the header will be added. Note that the backend's
  setting of the header subargument takes precedence over the frontend's if
  both are defined. In the case of the "if-none" argument, if at least one of
  the frontend or the backend does not specify it, it wants the addition to be
  mandatory, so it wins.

  It is important to note that by default, HAProxy works in tunnel mode and
  only inspects the first request of a connection, meaning that only the first
  request will have the header appended, which is certainly not what you want.
  In order to fix this, ensure that any of the "httpclose", "forceclose" or
  "http-server-close" options is set when using this option.

  Examples:
\begin{verbatim}
    # Public HTTP address also used by stunnel on the same machine
    frontend www
        mode http
        option forwardfor except 127.0.0.1  # stunnel already adds the header

    # Those servers want the IP Address in X-Client
    backend www
        mode http
        option forwardfor header X-Client
\end{verbatim}

  See also: "option httpclose", "option http-server-close",
             "option forceclose"

\subsubsection{option http-no-delay}
\subsubsection{no option http-no-delay}

\index{http-no-delay}

  Instruct the system to favor low interactive delays over performance in HTTP

\dflb{yes}{yes}{yes}{yes}

  Arguments: none

  In HTTP, each payload is unidirectional and has no notion of interactivity.
  Any agent is expected to queue data somewhat for a reasonably low delay.
  There are some very rare server-to-server applications that abuse the HTTP
  protocol and expect the payload phase to be highly interactive, with many
  interleaved data chunks in both directions within a single request. This is
  absolutely not supported by the HTTP specification and will not work across
  most proxies or servers. When such applications attempt to do this through
  haproxy, it works but they will experience high delays due to the network
  optimizations which favor performance by instructing the system to wait for
  enough data to be available in order to only send full packets. Typical
  delays are around 200 ms per round trip. Note that this only happens with
  abnormal uses. Normal uses such as CONNECT requests nor WebSockets are not
  affected.

  When "option http-no-delay" is present in either the frontend or the backend
  used by a connection, all such optimizations will be disabled in order to
  make the exchanges as fast as possible. Of course this offers no guarantee on
  the functionality, as it may break at any other place. But if it works via
  HAProxy, it will work as fast as possible. This option should never be used
  by default, and should never be used at all unless such a buggy application
  is discovered. The impact of using this option is an increase of bandwidth
  usage and CPU usage, which may significantly lower performance in high
  latency environments.

\subsubsection{option http-pretend-keepalive}
\subsubsection{no option http-pretend-keepalive}

\index{http-pretend-keepalive}

  Define whether haproxy will announce keepalive to the server or not

\dflb{yes}{yes}{yes}{yes}

  Arguments: none

  When running with "option http-server-close" or "option forceclose", haproxy
  adds a "Connection: close" header to the request forwarded to the server.
  Unfortunately, when some servers see this header, they automatically refrain
  from using the chunked encoding for responses of unknown length, while this
  is totally unrelated. The immediate effect is that this prevents haproxy from
  maintaining the client connection alive. A second effect is that a client or
  a cache could receive an incomplete response without being aware of it, and
  consider the response complete.

  By setting "option http-pretend-keepalive", haproxy will make the server
  believe it will keep the connection alive. The server will then not fall back
  to the abnormal undesired above. When haproxy gets the whole response, it
  will close the connection with the server just as it would do with the
  "forceclose" option. That way the client gets a normal response and the
  connection is correctly closed on the server side.

  It is recommended not to enable this option by default, because most servers
  will more efficiently close the connection themselves after the last packet,
  and release its buffers slightly earlier. Also, the added packet on the
  network could slightly reduce the overall peak performance. However it is
  worth noting that when this option is enabled, haproxy will have slightly
  less work to do. So if haproxy is the bottleneck on the whole architecture,
  enabling this option might save a few CPU cycles.

  This option may be set both in a frontend and in a backend. It is enabled if
  at least one of the frontend or backend holding a connection has it enabled.
  This option may be combined with "option httpclose", which will cause
  keepalive to be announced to the server and close to be announced to the
  client. This practice is discouraged though.

  If this option has been enabled in a "defaults" section, it can be disabled
  in a specific instance by prepending the "no" keyword before it.

  See also : "option forceclose" and "option http-server-close"

\subsubsection{option http-server-close}
\subsubsection{no option http-server-close}

\index{http-server-close}

  Enable or disable HTTP connection closing on the server side

\dflb{yes}{yes}{yes}{yes}

  Arguments: none

  By default, when a client communicates with a server, HAProxy will only
  analyze, log, and process the first request of each connection. Setting
  "option http-server-close" enables HTTP connection-close mode on the server
  side while keeping the ability to support HTTP keep-alive and pipelining on
  the client side.  This provides the lowest latency on the client side (slow
  network) and the fastest session reuse on the server side to save server
  resources, similarly to "option forceclose". It also permits non-keepalive
  capable servers to be served in keep-alive mode to the clients if they
  conform to the requirements of RFC2616. Please note that some servers do not
  always conform to those requirements when they see "Connection: close" in the
  request. The effect will be that keep-alive will never be used. A workaround
  consists in enabling "option http-pretend-keepalive".

  At the moment, logs will not indicate whether requests came from the same
  session or not. The accept date reported in the logs corresponds to the end
  of the previous request, and the request time corresponds to the time spent
  waiting for a new request. The keep-alive request time is still bound to the
  timeout defined by "timeout http-keep-alive" or "timeout http-request" if
  not set.

  This option may be set both in a frontend and in a backend. It is enabled if
  at least one of the frontend or backend holding a connection has it enabled.
  It is worth noting that "option forceclose" has precedence over "option
  http-server-close" and that combining "http-server-close" with "httpclose"
  basically achieve the same result as "forceclose".

  If this option has been enabled in a "defaults" section, it can be disabled
  in a specific instance by prepending the "no" keyword before it.

  See also: "option forceclose", "option http-pretend-keepalive",
             "option httpclose" and "1.1. The HTTP transaction model".

\subsubsection{option http-use-proxy-header}
\subsubsection{no option http-use-proxy-header}

\index{http-use-proxy-header}

  Make use of non-standard Proxy-Connection header instead of Connection

\dflb{yes}{yes}{yes}{no}

  Arguments: none

  While RFC2616 explicitly states that HTTP/1.1 agents must use the
  Connection header to indicate their wish of persistent or non-persistent
  connections, both browsers and proxies ignore this header for proxied
  connections and make use of the undocumented, non-standard Proxy-Connection
  header instead. The issue begins when trying to put a load balancer between
  browsers and such proxies, because there will be a difference between what
  haproxy understands and what the client and the proxy agree on.

  By setting this option in a frontend, haproxy can automatically switch to use
  that non-standard header if it sees proxied requests. A proxied request is
  defined here as one where the URI begins with neither a '/' nor a '*'. The
  choice of header only affects requests passing through proxies making use of
  one of the "httpclose", "forceclose" and "http-server-close" options. Note
  that this option can only be specified in a frontend and will affect the
  request along its whole life.

  Also, when this option is set, a request which requires authentication will
  automatically switch to use proxy authentication headers if it is itself a
  proxied request. That makes it possible to check or enforce authentication in
  front of an existing proxy.

  This option should normally never be used, except in front of a proxy.

  See also: "option httpclose", "option forceclose" and "option
             http-server-close".


\subsubsection{option httpchk}
\subsubsection*{option httpchk <uri>}
\subsubsection*{option httpchk <method> <uri>}
\subsubsection*{option httpchk <method> <uri> <version>}

\index{httpchk}

  Enable HTTP protocol to check on the servers health

\dflb{yes}{no}{yes}{yes}

  Arguments:
\begin{description}
\item[<method>]  is the optional HTTP method used with the requests. When not set,
              the "OPTIONS" method is used, as it generally requires low server
              processing and is easy to filter out from the logs. Any method
              may be used, though it is not recommended to invent non-standard
              ones.

\item[<uri>]     is the URI referenced in the HTTP requests. It defaults to " / "
              which is accessible by default on almost any server, but may be
              changed to any other URI. Query strings are permitted.

\item[<version>] is the optional HTTP version string. It defaults to "HTTP/1.0"
              but some servers might behave incorrectly in HTTP 1.0, so turning
              it to HTTP/1.1 may sometimes help. Note that the Host field is
              mandatory in HTTP/1.1, and as a trick, it is possible to pass it
              after "\textbackslash{}r\textbackslash{}n" following the version string.
\end{description}

  By default, server health checks only consist in trying to establish a TCP
  connection. When "option httpchk" is specified, a complete HTTP request is
  sent once the TCP connection is established, and responses 2xx and 3xx are
  considered valid, while all other ones indicate a server failure, including
  the lack of any response.

  The port and interval are specified in the server configuration.

  This option does not necessarily require an HTTP backend, it also works with
  plain TCP backends. This is particularly useful to check simple scripts bound
  to some dedicated ports using the inetd daemon.

  Examples :
\begin{verbatim}
      # Relay HTTPS traffic to Apache instance and check service availability
      # using HTTP request "OPTIONS * HTTP/1.1" on port 80.
      backend https_relay
          mode tcp
          option httpchk OPTIONS * HTTP/1.1\r\nHost:\ www
          server apache1 192.168.1.1:443 check port 80
\end{verbatim}

  See also: "option ssl-hello-chk", "option smtpchk", "option mysql-check",
             "option pgsql-check", "http-check" and the "check", "port" and
             "inter" server options.

\subsubsection{option httpclose}
\subsubsection{no option httpclose}

\index{httpclose}

  Enable or disable passive HTTP connection closing

\dflb{yes}{yes}{yes}{yes}

  Arguments: none

  By default, when a client communicates with a server, HAProxy will only
  analyze, log, and process the first request of each connection. If "option
  httpclose" is set, it will check if a "Connection: close" header is already
  set in each direction, and will add one if missing. Each end should react to
  this by actively closing the TCP connection after each transfer, thus
  resulting in a switch to the HTTP close mode. Any "Connection" header
  different from "close" will also be removed.

  It seldom happens that some servers incorrectly ignore this header and do not
  close the connection eventhough they reply "Connection: close". For this
  reason, they are not compatible with older HTTP 1.0 browsers. If this happens
  it is possible to use the "option forceclose" which actively closes the
  request connection once the server responds. Option "forceclose" also
  releases the server connection earlier because it does not have to wait for
  the client to acknowledge it.

  This option may be set both in a frontend and in a backend. It is enabled if
  at least one of the frontend or backend holding a connection has it enabled.
  If "option forceclose" is specified too, it has precedence over "httpclose".
  If "option http-server-close" is enabled at the same time as "httpclose", it
  basically achieves the same result as "option forceclose".

  If this option has been enabled in a "defaults" section, it can be disabled
  in a specific instance by prepending the "no" keyword before it.

  See also : "option forceclose", "option http-server-close" and
             "1.1. The HTTP transaction model".

\subsubsection[option httplog]{option httplog [ clf ]}

\index{httplog}

  Enable logging of HTTP request, session state and timers

\dflb{yes}{yes}{yes}{yes}

  Arguments:
  
\begin{description}
\item[clf]       if the "clf" argument is added, then the output format will be
              the CLF format instead of HAProxy's default HTTP format. You can
              use this when you need to feed HAProxy's logs through a specific
              log analyser which only support the CLF format and which is not
              extensible.
\end{description}

  By default, the log output format is very poor, as it only contains the
  source and destination addresses, and the instance name. By specifying
  "option httplog", each log line turns into a much richer format including,
  but not limited to, the HTTP request, the connection timers, the session
  status, the connections numbers, the captured headers and cookies, the
  frontend, backend and server name, and of course the source address and
  ports.

  This option may be set either in the frontend or the backend.

  If this option has been enabled in a "defaults" section, it can be disabled
  in a specific instance by prepending the "no" keyword before it. Specifying
  only "option httplog" will automatically clear the 'clf' mode if it was set
  by default.

  See also:  section 8 about logging.

\subsubsection{option http\_proxy}
\subsubsection{no option http\_proxy}

\index{http\_proxy}

  Enable or disable plain HTTP proxy mode

\dflb{yes}{yes}{yes}{yes}
                                 
  Arguments: none

  It sometimes happens that people need a pure HTTP proxy which understands
  basic proxy requests without caching nor any fancy feature. In this case,
  it may be worth setting up an HAProxy instance with the "option http\_proxy"
  set. In this mode, no server is declared, and the connection is forwarded to
  the IP address and port found in the URL after the "http://" scheme.

  No host address resolution is performed, so this only works when pure IP
  addresses are passed. Since this option's usage perimeter is rather limited,
  it will probably be used only by experts who know they need exactly it. Last,
  if the clients are susceptible of sending keep-alive requests, it will be
  needed to add "option httpclose" to ensure that all requests will correctly
  be analyzed.

  If this option has been enabled in a "defaults" section, it can be disabled
  in a specific instance by prepending the "no" keyword before it.

  Example:
\begin{verbatim}
    # this backend understands HTTP proxy requests and forwards them directly.
    backend direct_forward
        option httpclose
        option http_proxy
\end{verbatim}

  See also: "option httpclose"

\subsubsection{option independant-streams}
\subsubsection{no option independant-streams}

\index{independant-streams}

  Enable or disable independant timeout processing for both directions

\dflb{yes}{yes}{yes}{yes}

  Arguments: none

  By default, when data is sent over a socket, both the write timeout and the
  read timeout for that socket are refreshed, because we consider that there is
  activity on that socket, and we have no other means of guessing if we should
  receive data or not.

  While this default behavior is desirable for almost all applications, there
  exists a situation where it is desirable to disable it, and only refresh the
  read timeout if there are incoming data. This happens on sessions with large
  timeouts and low amounts of exchanged data such as telnet session. If the
  server suddenly disappears, the output data accumulates in the system's
  socket buffers, both timeouts are correctly refreshed, and there is no way
  to know the server does not receive them, so we don't timeout. However, when
  the underlying protocol always echoes sent data, it would be enough by itself
  to detect the issue using the read timeout. Note that this problem does not
  happen with more verbose protocols because data won't accumulate long in the
  socket buffers.

  When this option is set on the frontend, it will disable read timeout updates
  on data sent to the client. There probably is little use of this case. When
  the option is set on the backend, it will disable read timeout updates on
  data sent to the server. Doing so will typically break large HTTP posts from
  slow lines, so use it with caution.

  See also: "timeout client", "timeout server" and "timeout tunnel"

\subsubsection{option ldap-check}

\index{ldap-check}

  Use LDAPv3 health checks for server testing

\dflb{yes}{no}{yes}{yes}

  Arguments: none

  It is possible to test that the server correctly talks LDAPv3 instead of just
  testing that it accepts the TCP connection. When this option is set, an
  LDAPv3 anonymous simple bind message is sent to the server, and the response
  is analyzed to find an LDAPv3 bind response message.

  The server is considered valid only when the LDAP response contains success
  resultCode (http://tools.ietf.org/html/rfc4511#section-4.1.9).

  Logging of bind requests is server dependent see your documentation how to
  configure it.

  Example:
\begin{verbatim}
        option ldap-check
\end{verbatim}

  See also: "option httpchk"


\subsubsection{option log-health-checks}
\subsubsection{no option log-health-checks}

\index{log-health-checks}

  Enable or disable logging of health checks

\dflb{yes}{no}{yes}{yes}

  Arguments : none

  Enable health checks logging so it possible to check for example what
  was happening before a server crash. Failed health check are logged if
  server is UP and succeeded health checks if server is DOWN, so the amount
  of additional information is limited.

  If health check logging is enabled no health check status is printed
  when servers is set up UP/DOWN/ENABLED/DISABLED.

  See also: "log" and section 8 about logging.

\subsubsection{option log-separate-errors}
\subsubsection{no option log-separate-errors}

\index{log-separate-errors}

  Change log level for non-completely successful connections

\dflb{yes}{yes}{yes}{no}

  Arguments: none

  Sometimes looking for errors in logs is not easy. This option makes haproxy
  raise the level of logs containing potentially interesting information such
  as errors, timeouts, retries, redispatches, or HTTP status codes 5xx. The
  level changes from "info" to "err". This makes it possible to log them
  separately to a different file with most syslog daemons. Be careful not to
  remove them from the original file, otherwise you would lose ordering which
  provides very important information.

  Using this option, large sites dealing with several thousand connections per
  second may log normal traffic to a rotating buffer and only archive smaller
  error logs.

  See also: "log", "dontlognull", "dontlog-normal" and section 8 about
             logging.

\subsubsection{option logasap}
\subsubsection{no option logasap}

\index{logasap}

  Enable or disable early logging of HTTP requests

\dflb{yes}{yes}{yes}{no}

  Arguments: none

  By default, HTTP requests are logged upon termination so that the total
  transfer time and the number of bytes appear in the logs. When large objects
  are being transferred, it may take a while before the request appears in the
  logs. Using "option logasap", the request gets logged as soon as the server
  sends the complete headers. The only missing information in the logs will be
  the total number of bytes which will indicate everything except the amount
  of data transferred, and the total time which will not take the transfer
  time into account. In such a situation, it's a good practice to capture the
  "Content-Length" response header so that the logs at least indicate how many
  bytes are expected to be transferred.

  Examples:
\begin{verbatim}
      listen http_proxy 0.0.0.0:80
          mode http
          option httplog
          option logasap
          log 192.168.2.200 local3

    >>> Feb  6 12:14:14 localhost \
          haproxy[14389]: 10.0.1.2:33317 [06/Feb/2009:12:14:14.655] http-in \
          static/srv1 9/10/7/14/+30 200 +243 - - ---- 3/1/1/1/0 1/0 \
          "GET /image.iso HTTP/1.0"
\end{verbatim}

  See also : "option httplog", "capture response header", and section 8 about
             logging.

\subsubsection[option mysql-check]{option mysql-check [ user <username> ]}
\index{mysql-check}

  Use MySQL health checks for server testing

\dflb{yes}{no}{yes}{yes}

  Arguments:
\begin{description}
\item[<username>] This is the username which will be used when connecting to MySQL
               server.
\end{description}

  If you specify a username, the check consists of sending two MySQL packet,
  one Client Authentication packet, and one QUIT packet, to correctly close
  MySQL session. We then parse the MySQL Handshake Initialisation packet and/or
  Error packet. It is a basic but useful test which does not produce error nor
  aborted connect on the server. However, it requires adding an authorization
  in the MySQL table, like this:

\begin{verbatim}
      USE mysql;
      INSERT INTO user (Host,User) values ('<ip_of_haproxy>','<username>');
      FLUSH PRIVILEGES;
\end{verbatim}

  If you don't specify a username (it is deprecated and not recommended), the
  check only consists in parsing the Mysql Handshake Initialization packet or
  Error packet, we don't send anything in this mode. It was reported that it
  can generate lockout if check is too frequent and/or if there is not enough
  traffic. In fact, you need in this case to check MySQL "max\_connect\_errors"
  value as if a connection is established successfully within fewer than MySQL
  "max\_connect\_errors" attempts after a previous connection was interrupted,
  the error count for the host is cleared to zero. If HAProxy's server get
  blocked, the "FLUSH HOSTS" statement is the only way to unblock it.

  Remember that this does not check database presence nor database consistency.
  To do this, you can use an external check with xinetd for example.

  The check requires MySQL >=3.22, for older version, please use TCP check.

  Most often, an incoming MySQL server needs to see the client's IP address for
  various purposes, including IP privilege matching and connection logging.
  When possible, it is often wise to masquerade the client's IP address when
  connecting to the server using the "usesrc" argument of the "source" keyword,
  which requires the cttproxy feature to be compiled in, and the MySQL server
  to route the client via the machine hosting haproxy.

  See also: "option httpchk"

\subsubsection[option pgsql-check]{option pgsql-check [ user <username> ]}

\index{pgsql-check}

  Use PostgreSQL health checks for server testing

\dflb{yes}{no}{yes}{yes}

  Arguments:
\begin{description}
\item[<username>] This is the username which will be used when connecting to
               PostgreSQL server.
\end{description}

  The check sends a PostgreSQL StartupMessage and waits for either
  Authentication request or ErrorResponse message. It is a basic but useful
  test which does not produce error nor aborted connect on the server.
  This check is identical with the "mysql-check".

  See also: "option httpchk"

\subsubsection{option nolinger}
\subsubsection{no option nolinger}

\index{nolinger}

  Enable or disable immediate session resource cleaning after close

\dflb{yes}{yes}{yes}{yes}

  Arguments: none

  When clients or servers abort connections in a dirty way (eg: they are
  physically disconnected), the session timeouts triggers and the session is
  closed. But it will remain in FIN\_WAIT1 state for some time in the system,
  using some resources and possibly limiting the ability to establish newer
  connections.

  When this happens, it is possible to activate "option nolinger" which forces
  the system to immediately remove any socket's pending data on close. Thus,
  the session is instantly purged from the system's tables. This usually has
  side effects such as increased number of TCP resets due to old retransmits
  getting immediately rejected. Some firewalls may sometimes complain about
  this too.

  For this reason, it is not recommended to use this option when not absolutely
  needed. You know that you need it when you have thousands of FIN\_WAIT1
  sessions on your system (TIME\_WAIT ones do not count).

  This option may be used both on frontends and backends, depending on the side
  where it is required. Use it on the frontend for clients, and on the backend
  for servers.

  If this option has been enabled in a "defaults" section, it can be disabled
  in a specific instance by prepending the "no" keyword before it.

\subsubsection[option originalto]{option originalto [ except <network> ] [ header <name> ]}

\index{originalto}

  Enable insertion of the X-Original-To header to requests sent to servers

\dflb{yes}{yes}{yes}{yes}

  Arguments:
\begin{description}
\item[<network>] is an optional argument used to disable this option for sources
              matching <network>
\item[<name>] an optional argument to specify a different "X-Original-To"
              header name.
\end{description}

  Since HAProxy can work in transparent mode, every request from a client can
  be redirected to the proxy and HAProxy itself can proxy every request to a
  complex SQUID environment and the destination host from SO\_ORIGINAL\_DST will
  be lost. This is annoying when you want access rules based on destination ip
  addresses. To solve this problem, a new HTTP header "X-Original-To" may be
  added by HAProxy to all requests sent to the server. This header contains a
  value representing the original destination IP address. Since this must be
  configured to always use the last occurrence of this header only. Note that
  only the last occurrence of the header must be used, since it is really
  possible that the client has already brought one.

  The keyword "header" may be used to supply a different header name to replace
  the default "X-Original-To". This can be useful where you might already
  have a "X-Original-To" header from a different application, and you need
  preserve it. Also if your backend server doesn't use the "X-Original-To"
  header and requires different one.

  Sometimes, a same HAProxy instance may be shared between a direct client
  access and a reverse-proxy access (for instance when an SSL reverse-proxy is
  used to decrypt HTTPS traffic). It is possible to disable the addition of the
  header for a known source address or network by adding the "except" keyword
  followed by the network address. In this case, any source IP matching the
  network will not cause an addition of this header. Most common uses are with
  private networks or 127.0.0.1.

  This option may be specified either in the frontend or in the backend. If at
  least one of them uses it, the header will be added. Note that the backend's
  setting of the header subargument takes precedence over the frontend's if
  both are defined.

  It is important to note that by default, HAProxy works in tunnel mode and
  only inspects the first request of a connection, meaning that only the first
  request will have the header appended, which is certainly not what you want.
  In order to fix this, ensure that any of the "httpclose", "forceclose" or
  "http-server-close" options is set when using this option.

  Examples :
\begin{verbatim}
    # Original Destination address
    frontend www
        mode http
        option originalto except 127.0.0.1

    # Those servers want the IP Address in X-Client-Dst
    backend www
        mode http
        option originalto header X-Client-Dst
\end{verbatim}

  See also: "option httpclose", "option http-server-close",
             "option forceclose"

\subsubsection{option persist}
\subsubsection{no option persist}

\index{persist}

  Enable or disable forced persistence on down servers

\dflb{yes}{no}{yes}{yes}

  Arguments: none

  When an HTTP request reaches a backend with a cookie which references a dead
  server, by default it is redispatched to another server. It is possible to
  force the request to be sent to the dead server first using "option persist"
  if absolutely needed. A common use case is when servers are under extreme
  load and spend their time flapping. In this case, the users would still be
  directed to the server they opened the session on, in the hope they would be
  correctly served. It is recommended to use "option redispatch" in conjunction
  with this option so that in the event it would not be possible to connect to
  the server at all (server definitely dead), the client would finally be
  redirected to another valid server.

  If this option has been enabled in a "defaults" section, it can be disabled
  in a specific instance by prepending the "no" keyword before it.

  See also: "option redispatch", "retries", "force-persist"

\subsubsection{option redispatch}
\subsubsection{no option redispatch}

\index{redispatch}

  Enable or disable session redistribution in case of connection failure

\dflb{yes}{no}{yes}{yes}

  Arguments: none

  In HTTP mode, if a server designated by a cookie is down, clients may
  definitely stick to it because they cannot flush the cookie, so they will not
  be able to access the service anymore.

  Specifying "option redispatch" will allow the proxy to break their
  persistence and redistribute them to a working server.

  It also allows to retry last connection to another server in case of multiple
  connection failures. Of course, it requires having "retries" set to a nonzero
  value.

  This form is the preferred form, which replaces both the "redispatch" and
  "redisp" keywords.

  If this option has been enabled in a "defaults" section, it can be disabled
  in a specific instance by prepending the "no" keyword before it.

  See also: "redispatch", "retries", "force-persist"

\subsubsection{option redis-check}

\index{redis-check}

  Use redis health checks for server testing

\dflb{yes}{no}{yes}{yes}

  Arguments: none

  It is possible to test that the server correctly talks REDIS protocol instead
  of just testing that it accepts the TCP connection. When this option is set,
  a PING redis command is sent to the server, and the response is analyzed to
  find the "+PONG" response message.

  Example:
\begin{verbatim}
        option redis-check
\end{verbatim}

  See also: "option httpchk"

\subsubsection{option smtpchk}
\subsubsection*{option smtpchk <hello> <domain>}

\index{smtpchk}

  Use SMTP health checks for server testing

\dflb{yes}{no}{yes}{yes}

  Arguments:
  
\begin{description}
\item[<hello>]   is an optional argument. It is the "hello" command to use. It can
              be either "HELO" (for SMTP) or "EHLO" (for ESTMP). All other
              values will be turned into the default command ("HELO").

\item[<domain>] is the domain name to present to the server. It may only be
              specified (and is mandatory) if the hello command has been
              specified. By default, "localhost" is used.
\end{description}

  When "option smtpchk" is set, the health checks will consist in TCP
  connections followed by an SMTP command. By default, this command is
  "HELO localhost". The server's return code is analyzed and only return codes
  starting with a "2" will be considered as valid. All other responses,
  including a lack of response will constitute an error and will indicate a
  dead server.

  This test is meant to be used with SMTP servers or relays. Depending on the
  request, it is possible that some servers do not log each connection attempt,
  so you may want to experiment to improve the behavior. Using telnet on port
  25 is often easier than adjusting the configuration.

  Most often, an incoming SMTP server needs to see the client's IP address for
  various purposes, including spam filtering, anti-spoofing and logging. When
  possible, it is often wise to masquerade the client's IP address when
  connecting to the server using the "usesrc" argument of the "source" keyword,
  which requires the cttproxy feature to be compiled in.

  Example:
\begin{verbatim}
        option smtpchk HELO mydomain.org
\end{verbatim}

  See also: "option httpchk", "source"

\subsubsection{option socket-stats}
\subsubsection{no option socket-stats}

\index{socket-stats}

  Enable or disable collecting \& providing separate statistics for each socket.

\dflb{yes}{yes}{yes}{no}

  Arguments : none

\endinput

option splice-auto
no option splice-auto
  Enable or disable automatic kernel acceleration on sockets in both directions
  May be used in sections :   defaults | frontend | listen | backend
                                 yes   |    yes   |   yes  |   yes
  Arguments : none

  When this option is enabled either on a frontend or on a backend, haproxy
  will automatically evaluate the opportunity to use kernel tcp splicing to
  forward data between the client and the server, in either direction. Haproxy
  uses heuristics to estimate if kernel splicing might improve performance or
  not. Both directions are handled independently. Note that the heuristics used
  are not much aggressive in order to limit excessive use of splicing. This
  option requires splicing to be enabled at compile time, and may be globally
  disabled with the global option "nosplice". Since splice uses pipes, using it
  requires that there are enough spare pipes.

  Important note: kernel-based TCP splicing is a Linux-specific feature which
  first appeared in kernel 2.6.25. It offers kernel-based acceleration to
  transfer data between sockets without copying these data to user-space, thus
  providing noticeable performance gains and CPU cycles savings. Since many
  early implementations are buggy, corrupt data and/or are inefficient, this
  feature is not enabled by default, and it should be used with extreme care.
  While it is not possible to detect the correctness of an implementation,
  2.6.29 is the first version offering a properly working implementation. In
  case of doubt, splicing may be globally disabled using the global "nosplice"
  keyword.

  Example :
        option splice-auto

  If this option has been enabled in a "defaults" section, it can be disabled
  in a specific instance by prepending the "no" keyword before it.

  See also : "option splice-request", "option splice-response", and global
             options "nosplice" and "maxpipes"


option splice-request
no option splice-request
  Enable or disable automatic kernel acceleration on sockets for requests
  May be used in sections :   defaults | frontend | listen | backend
                                 yes   |    yes   |   yes  |   yes
  Arguments : none

  When this option is enabled either on a frontend or on a backend, haproxy
  will user kernel tcp splicing whenever possible to forward data going from
  the client to the server. It might still use the recv/send scheme if there
  are no spare pipes left. This option requires splicing to be enabled at
  compile time, and may be globally disabled with the global option "nosplice".
  Since splice uses pipes, using it requires that there are enough spare pipes.

  Important note: see "option splice-auto" for usage limitations.

  Example :
        option splice-request

  If this option has been enabled in a "defaults" section, it can be disabled
  in a specific instance by prepending the "no" keyword before it.

  See also : "option splice-auto", "option splice-response", and global options
             "nosplice" and "maxpipes"


option splice-response
no option splice-response
  Enable or disable automatic kernel acceleration on sockets for responses
  May be used in sections :   defaults | frontend | listen | backend
                                 yes   |    yes   |   yes  |   yes
  Arguments : none

  When this option is enabled either on a frontend or on a backend, haproxy
  will user kernel tcp splicing whenever possible to forward data going from
  the server to the client. It might still use the recv/send scheme if there
  are no spare pipes left. This option requires splicing to be enabled at
  compile time, and may be globally disabled with the global option "nosplice".
  Since splice uses pipes, using it requires that there are enough spare pipes.

  Important note: see "option splice-auto" for usage limitations.

  Example :
        option splice-response

  If this option has been enabled in a "defaults" section, it can be disabled
  in a specific instance by prepending the "no" keyword before it.

  See also : "option splice-auto", "option splice-request", and global options
             "nosplice" and "maxpipes"


option srvtcpka
no option srvtcpka
  Enable or disable the sending of TCP keepalive packets on the server side
  May be used in sections :   defaults | frontend | listen | backend
                                 yes   |    no    |   yes  |   yes
  Arguments : none

  When there is a firewall or any session-aware component between a client and
  a server, and when the protocol involves very long sessions with long idle
  periods (eg: remote desktops), there is a risk that one of the intermediate
  components decides to expire a session which has remained idle for too long.

  Enabling socket-level TCP keep-alives makes the system regularly send packets
  to the other end of the connection, leaving it active. The delay between
  keep-alive probes is controlled by the system only and depends both on the
  operating system and its tuning parameters.

  It is important to understand that keep-alive packets are neither emitted nor
  received at the application level. It is only the network stacks which sees
  them. For this reason, even if one side of the proxy already uses keep-alives
  to maintain its connection alive, those keep-alive packets will not be
  forwarded to the other side of the proxy.

  Please note that this has nothing to do with HTTP keep-alive.

  Using option "srvtcpka" enables the emission of TCP keep-alive probes on the
  server side of a connection, which should help when session expirations are
  noticed between HAProxy and a server.

  If this option has been enabled in a "defaults" section, it can be disabled
  in a specific instance by prepending the "no" keyword before it.

  See also : "option clitcpka", "option tcpka"


option ssl-hello-chk
  Use SSLv3 client hello health checks for server testing
  May be used in sections :   defaults | frontend | listen | backend
                                 yes   |    no    |   yes  |   yes
  Arguments : none

  When some SSL-based protocols are relayed in TCP mode through HAProxy, it is
  possible to test that the server correctly talks SSL instead of just testing
  that it accepts the TCP connection. When "option ssl-hello-chk" is set, pure
  SSLv3 client hello messages are sent once the connection is established to
  the server, and the response is analyzed to find an SSL server hello message.
  The server is considered valid only when the response contains this server
  hello message.

  All servers tested till there correctly reply to SSLv3 client hello messages,
  and most servers tested do not even log the requests containing only hello
  messages, which is appreciable.

  See also: "option httpchk"


option tcp-smart-accept
no option tcp-smart-accept
  Enable or disable the saving of one ACK packet during the accept sequence
  May be used in sections :   defaults | frontend | listen | backend
                                 yes   |    yes   |   yes  |    no
  Arguments : none

  When an HTTP connection request comes in, the system acknowledges it on
  behalf of HAProxy, then the client immediately sends its request, and the
  system acknowledges it too while it is notifying HAProxy about the new
  connection. HAProxy then reads the request and responds. This means that we
  have one TCP ACK sent by the system for nothing, because the request could
  very well be acknowledged by HAProxy when it sends its response.

  For this reason, in HTTP mode, HAProxy automatically asks the system to avoid
  sending this useless ACK on platforms which support it (currently at least
  Linux). It must not cause any problem, because the system will send it anyway
  after 40 ms if the response takes more time than expected to come.

  During complex network debugging sessions, it may be desirable to disable
  this optimization because delayed ACKs can make troubleshooting more complex
  when trying to identify where packets are delayed. It is then possible to
  fall back to normal behaviour by specifying "no option tcp-smart-accept".

  It is also possible to force it for non-HTTP proxies by simply specifying
  "option tcp-smart-accept". For instance, it can make sense with some services
  such as SMTP where the server speaks first.

  It is recommended to avoid forcing this option in a defaults section. In case
  of doubt, consider setting it back to automatic values by prepending the
  "default" keyword before it, or disabling it using the "no" keyword.

  See also : "option tcp-smart-connect"


option tcp-smart-connect
no option tcp-smart-connect
  Enable or disable the saving of one ACK packet during the connect sequence
  May be used in sections :   defaults | frontend | listen | backend
                                 yes   |    no    |   yes  |   yes
  Arguments : none

  On certain systems (at least Linux), HAProxy can ask the kernel not to
  immediately send an empty ACK upon a connection request, but to directly
  send the buffer request instead. This saves one packet on the network and
  thus boosts performance. It can also be useful for some servers, because they
  immediately get the request along with the incoming connection.

  This feature is enabled when "option tcp-smart-connect" is set in a backend.
  It is not enabled by default because it makes network troubleshooting more
  complex.

  It only makes sense to enable it with protocols where the client speaks first
  such as HTTP. In other situations, if there is no data to send in place of
  the ACK, a normal ACK is sent.

  If this option has been enabled in a "defaults" section, it can be disabled
  in a specific instance by prepending the "no" keyword before it.

  See also : "option tcp-smart-accept"


option tcpka
  Enable or disable the sending of TCP keepalive packets on both sides
  May be used in sections :   defaults | frontend | listen | backend
                                 yes   |    yes   |   yes  |   yes
  Arguments : none

  When there is a firewall or any session-aware component between a client and
  a server, and when the protocol involves very long sessions with long idle
  periods (eg: remote desktops), there is a risk that one of the intermediate
  components decides to expire a session which has remained idle for too long.

  Enabling socket-level TCP keep-alives makes the system regularly send packets
  to the other end of the connection, leaving it active. The delay between
  keep-alive probes is controlled by the system only and depends both on the
  operating system and its tuning parameters.

  It is important to understand that keep-alive packets are neither emitted nor
  received at the application level. It is only the network stacks which sees
  them. For this reason, even if one side of the proxy already uses keep-alives
  to maintain its connection alive, those keep-alive packets will not be
  forwarded to the other side of the proxy.

  Please note that this has nothing to do with HTTP keep-alive.

  Using option "tcpka" enables the emission of TCP keep-alive probes on both
  the client and server sides of a connection. Note that this is meaningful
  only in "defaults" or "listen" sections. If this option is used in a
  frontend, only the client side will get keep-alives, and if this option is
  used in a backend, only the server side will get keep-alives. For this
  reason, it is strongly recommended to explicitly use "option clitcpka" and
  "option srvtcpka" when the configuration is split between frontends and
  backends.

  See also : "option clitcpka", "option srvtcpka"


option tcplog
  Enable advanced logging of TCP connections with session state and timers
  May be used in sections :   defaults | frontend | listen | backend
                                 yes   |    yes   |   yes  |   yes
  Arguments : none

  By default, the log output format is very poor, as it only contains the
  source and destination addresses, and the instance name. By specifying
  "option tcplog", each log line turns into a much richer format including, but
  not limited to, the connection timers, the session status, the connections
  numbers, the frontend, backend and server name, and of course the source
  address and ports. This option is useful for pure TCP proxies in order to
  find which of the client or server disconnects or times out. For normal HTTP
  proxies, it's better to use "option httplog" which is even more complete.

  This option may be set either in the frontend or the backend.

  See also :  "option httplog", and section 8 about logging.


option transparent
no option transparent
  Enable client-side transparent proxying
  May be used in sections :   defaults | frontend | listen | backend
                                 yes   |    no    |   yes  |   yes
  Arguments : none

  This option was introduced in order to provide layer 7 persistence to layer 3
  load balancers. The idea is to use the OS's ability to redirect an incoming
  connection for a remote address to a local process (here HAProxy), and let
  this process know what address was initially requested. When this option is
  used, sessions without cookies will be forwarded to the original destination
  IP address of the incoming request (which should match that of another
  equipment), while requests with cookies will still be forwarded to the
  appropriate server.

  Note that contrary to a common belief, this option does NOT make HAProxy
  present the client's IP to the server when establishing the connection.

  See also: the "usesrc" argument of the "source" keyword, and the
            "transparent" option of the "bind" keyword.


persist rdp-cookie
persist rdp-cookie(<name>)
  Enable RDP cookie-based persistence
  May be used in sections :   defaults | frontend | listen | backend
                                 yes   |    no    |   yes  |   yes
  Arguments :
    <name>    is the optional name of the RDP cookie to check. If omitted, the
              default cookie name "msts" will be used. There currently is no
              valid reason to change this name.

  This statement enables persistence based on an RDP cookie. The RDP cookie
  contains all information required to find the server in the list of known
  servers. So when this option is set in the backend, the request is analysed
  and if an RDP cookie is found, it is decoded. If it matches a known server
  which is still UP (or if "option persist" is set), then the connection is
  forwarded to this server.

  Note that this only makes sense in a TCP backend, but for this to work, the
  frontend must have waited long enough to ensure that an RDP cookie is present
  in the request buffer. This is the same requirement as with the "rdp-cookie"
  load-balancing method. Thus it is highly recommended to put all statements in
  a single "listen" section.

  Also, it is important to understand that the terminal server will emit this
  RDP cookie only if it is configured for "token redirection mode", which means
  that the "IP address redirection" option is disabled.

  Example :
        listen tse-farm
            bind :3389
            # wait up to 5s for an RDP cookie in the request
            tcp-request inspect-delay 5s
            tcp-request content accept if RDP_COOKIE
            # apply RDP cookie persistence
            persist rdp-cookie
            # if server is unknown, let's balance on the same cookie.
            # alternatively, "balance leastconn" may be useful too.
            balance rdp-cookie
            server srv1 1.1.1.1:3389
            server srv2 1.1.1.2:3389

  See also : "balance rdp-cookie", "tcp-request", the "req_rdp_cookie" ACL and
  the rdp_cookie pattern fetch function.


rate-limit sessions <rate>
  Set a limit on the number of new sessions accepted per second on a frontend
  May be used in sections :   defaults | frontend | listen | backend
                                 yes   |    yes   |   yes  |   no
  Arguments :
    <rate>    The <rate> parameter is an integer designating the maximum number
              of new sessions per second to accept on the frontend.

  When the frontend reaches the specified number of new sessions per second, it
  stops accepting new connections until the rate drops below the limit again.
  During this time, the pending sessions will be kept in the socket's backlog
  (in system buffers) and haproxy will not even be aware that sessions are
  pending. When applying very low limit on a highly loaded service, it may make
  sense to increase the socket's backlog using the "backlog" keyword.

  This feature is particularly efficient at blocking connection-based attacks
  or service abuse on fragile servers. Since the session rate is measured every
  millisecond, it is extremely accurate. Also, the limit applies immediately,
  no delay is needed at all to detect the threshold.

  Example : limit the connection rate on SMTP to 10 per second max
        listen smtp
            mode tcp
            bind :25
            rate-limit sessions 10
            server 127.0.0.1:1025

  Note : when the maximum rate is reached, the frontend's status is not changed
         but its sockets appear as "WAITING" in the statistics if the
         "socket-stats" option is enabled.

  See also : the "backlog" keyword and the "fe_sess_rate" ACL criterion.


redirect location <to> [code <code>] <option> [{if | unless} <condition>]
redirect prefix   <to> [code <code>] <option> [{if | unless} <condition>]
  Return an HTTP redirection if/unless a condition is matched
  May be used in sections :   defaults | frontend | listen | backend
                                 no    |    yes   |   yes  |   yes

  If/unless the condition is matched, the HTTP request will lead to a redirect
  response. If no condition is specified, the redirect applies unconditionally.

  Arguments :
    <to>      With "redirect location", the exact value in <to> is placed into
              the HTTP "Location" header. In case of "redirect prefix", the
              "Location" header is built from the concatenation of <to> and the
              complete URI, including the query string, unless the "drop-query"
              option is specified (see below). As a special case, if <to>
              equals exactly "/" in prefix mode, then nothing is inserted
              before the original URI. It allows one to redirect to the same
              URL.

    <code>    The code is optional. It indicates which type of HTTP redirection
              is desired. Only codes 301, 302 and 303 are supported, and 302 is
              used if no code is specified. 301 means "Moved permanently", and
              a browser may cache the Location. 302 means "Moved permanently"
              and means that the browser should not cache the redirection. 303
              is equivalent to 302 except that the browser will fetch the
              location with a GET method.

    <option>  There are several options which can be specified to adjust the
              expected behaviour of a redirection :

      - "drop-query"
        When this keyword is used in a prefix-based redirection, then the
        location will be set without any possible query-string, which is useful
        for directing users to a non-secure page for instance. It has no effect
        with a location-type redirect.

      - "append-slash"
        This keyword may be used in conjunction with "drop-query" to redirect
        users who use a URL not ending with a '/' to the same one with the '/'.
        It can be useful to ensure that search engines will only see one URL.
        For this, a return code 301 is preferred.

      - "set-cookie NAME[=value]"
        A "Set-Cookie" header will be added with NAME (and optionally "=value")
        to the response. This is sometimes used to indicate that a user has
        been seen, for instance to protect against some types of DoS. No other
        cookie option is added, so the cookie will be a session cookie. Note
        that for a browser, a sole cookie name without an equal sign is
        different from a cookie with an equal sign.

      - "clear-cookie NAME[=]"
        A "Set-Cookie" header will be added with NAME (and optionally "="), but
        with the "Max-Age" attribute set to zero. This will tell the browser to
        delete this cookie. It is useful for instance on logout pages. It is
        important to note that clearing the cookie "NAME" will not remove a
        cookie set with "NAME=value". You have to clear the cookie "NAME=" for
        that, because the browser makes the difference.

  Example: move the login URL only to HTTPS.
        acl clear      dst_port  80
        acl secure     dst_port  8080
        acl login_page url_beg   /login
        acl logout     url_beg   /logout
        acl uid_given  url_reg   /login?userid=[^&]+
        acl cookie_set hdr_sub(cookie) SEEN=1

        redirect prefix   https://mysite.com set-cookie SEEN=1 if !cookie_set
        redirect prefix   https://mysite.com           if login_page !secure
        redirect prefix   http://mysite.com drop-query if login_page !uid_given
        redirect location http://mysite.com/           if !login_page secure
        redirect location / clear-cookie USERID=       if logout

  Example: send redirects for request for articles without a '/'.
        acl missing_slash path_reg ^/article/[^/]*$
        redirect code 301 prefix / drop-query append-slash if missing_slash

  See section 7 about ACL usage.


redisp (deprecated)
redispatch (deprecated)
  Enable or disable session redistribution in case of connection failure
  May be used in sections:    defaults | frontend | listen | backend
                                 yes   |    no    |   yes  |   yes
  Arguments : none

  In HTTP mode, if a server designated by a cookie is down, clients may
  definitely stick to it because they cannot flush the cookie, so they will not
  be able to access the service anymore.

  Specifying "redispatch" will allow the proxy to break their persistence and
  redistribute them to a working server.

  It also allows to retry last connection to another server in case of multiple
  connection failures. Of course, it requires having "retries" set to a nonzero
  value.

  This form is deprecated, do not use it in any new configuration, use the new
  "option redispatch" instead.

  See also : "option redispatch"


reqadd  <string> [{if | unless} <cond>]
  Add a header at the end of the HTTP request
  May be used in sections :   defaults | frontend | listen | backend
                                 no    |    yes   |   yes  |   yes
  Arguments :
    <string>  is the complete line to be added. Any space or known delimiter
              must be escaped using a backslash ('\'). Please refer to section
              6 about HTTP header manipulation for more information.

    <cond>    is an optional matching condition built from ACLs. It makes it
              possible to ignore this rule when other conditions are not met.

  A new line consisting in <string> followed by a line feed will be added after
  the last header of an HTTP request.

  Header transformations only apply to traffic which passes through HAProxy,
  and not to traffic generated by HAProxy, such as health-checks or error
  responses.

  Example : add "X-Proto: SSL" to requests coming via port 81
     acl is-ssl  dst_port       81
     reqadd      X-Proto:\ SSL  if is-ssl

  See also: "rspadd", section 6 about HTTP header manipulation, and section 7
            about ACLs.


reqallow  <search> [{if | unless} <cond>]
reqiallow <search> [{if | unless} <cond>] (ignore case)
  Definitely allow an HTTP request if a line matches a regular expression
  May be used in sections :   defaults | frontend | listen | backend
                                 no    |    yes   |   yes  |   yes
  Arguments :
    <search>  is the regular expression applied to HTTP headers and to the
              request line. This is an extended regular expression. Parenthesis
              grouping is supported and no preliminary backslash is required.
              Any space or known delimiter must be escaped using a backslash
              ('\'). The pattern applies to a full line at a time. The
              "reqallow" keyword strictly matches case while "reqiallow"
              ignores case.

    <cond>    is an optional matching condition built from ACLs. It makes it
              possible to ignore this rule when other conditions are not met.

  A request containing any line which matches extended regular expression
  <search> will mark the request as allowed, even if any later test would
  result in a deny. The test applies both to the request line and to request
  headers. Keep in mind that URLs in request line are case-sensitive while
  header names are not.

  It is easier, faster and more powerful to use ACLs to write access policies.
  Reqdeny, reqallow and reqpass should be avoided in new designs.

  Example :
     # allow www.* but refuse *.local
     reqiallow ^Host:\ www\.
     reqideny  ^Host:\ .*\.local

  See also: "reqdeny", "block", section 6 about HTTP header manipulation, and
            section 7 about ACLs.


reqdel  <search> [{if | unless} <cond>]
reqidel <search> [{if | unless} <cond>]  (ignore case)
  Delete all headers matching a regular expression in an HTTP request
  May be used in sections :   defaults | frontend | listen | backend
                                 no    |    yes   |   yes  |   yes
  Arguments :
    <search>  is the regular expression applied to HTTP headers and to the
              request line. This is an extended regular expression. Parenthesis
              grouping is supported and no preliminary backslash is required.
              Any space or known delimiter must be escaped using a backslash
              ('\'). The pattern applies to a full line at a time. The "reqdel"
              keyword strictly matches case while "reqidel" ignores case.

    <cond>    is an optional matching condition built from ACLs. It makes it
              possible to ignore this rule when other conditions are not met.

  Any header line matching extended regular expression <search> in the request
  will be completely deleted. Most common use of this is to remove unwanted
  and/or dangerous headers or cookies from a request before passing it to the
  next servers.

  Header transformations only apply to traffic which passes through HAProxy,
  and not to traffic generated by HAProxy, such as health-checks or error
  responses. Keep in mind that header names are not case-sensitive.

  Example :
     # remove X-Forwarded-For header and SERVER cookie
     reqidel ^X-Forwarded-For:.*
     reqidel ^Cookie:.*SERVER=

  See also: "reqadd", "reqrep", "rspdel", section 6 about HTTP header
            manipulation, and section 7 about ACLs.


reqdeny  <search> [{if | unless} <cond>]
reqideny <search> [{if | unless} <cond>]  (ignore case)
  Deny an HTTP request if a line matches a regular expression
  May be used in sections :   defaults | frontend | listen | backend
                                 no    |    yes   |   yes  |   yes
  Arguments :
    <search>  is the regular expression applied to HTTP headers and to the
              request line. This is an extended regular expression. Parenthesis
              grouping is supported and no preliminary backslash is required.
              Any space or known delimiter must be escaped using a backslash
              ('\'). The pattern applies to a full line at a time. The
              "reqdeny" keyword strictly matches case while "reqideny" ignores
              case.

    <cond>    is an optional matching condition built from ACLs. It makes it
              possible to ignore this rule when other conditions are not met.

  A request containing any line which matches extended regular expression
  <search> will mark the request as denied, even if any later test would
  result in an allow. The test applies both to the request line and to request
  headers. Keep in mind that URLs in request line are case-sensitive while
  header names are not.

  A denied request will generate an "HTTP 403 forbidden" response once the
  complete request has been parsed. This is consistent with what is practiced
  using ACLs.

  It is easier, faster and more powerful to use ACLs to write access policies.
  Reqdeny, reqallow and reqpass should be avoided in new designs.

  Example :
     # refuse *.local, then allow www.*
     reqideny  ^Host:\ .*\.local
     reqiallow ^Host:\ www\.

  See also: "reqallow", "rspdeny", "block", section 6 about HTTP header
            manipulation, and section 7 about ACLs.


reqpass  <search> [{if | unless} <cond>]
reqipass <search> [{if | unless} <cond>]  (ignore case)
  Ignore any HTTP request line matching a regular expression in next rules
  May be used in sections :   defaults | frontend | listen | backend
                                 no    |    yes   |   yes  |   yes
  Arguments :
    <search>  is the regular expression applied to HTTP headers and to the
              request line. This is an extended regular expression. Parenthesis
              grouping is supported and no preliminary backslash is required.
              Any space or known delimiter must be escaped using a backslash
              ('\'). The pattern applies to a full line at a time. The
              "reqpass" keyword strictly matches case while "reqipass" ignores
              case.

    <cond>    is an optional matching condition built from ACLs. It makes it
              possible to ignore this rule when other conditions are not met.

  A request containing any line which matches extended regular expression
  <search> will skip next rules, without assigning any deny or allow verdict.
  The test applies both to the request line and to request headers. Keep in
  mind that URLs in request line are case-sensitive while header names are not.

  It is easier, faster and more powerful to use ACLs to write access policies.
  Reqdeny, reqallow and reqpass should be avoided in new designs.

  Example :
     # refuse *.local, then allow www.*, but ignore "www.private.local"
     reqipass  ^Host:\ www.private\.local
     reqideny  ^Host:\ .*\.local
     reqiallow ^Host:\ www\.

  See also: "reqallow", "reqdeny", "block", section 6 about HTTP header
            manipulation, and section 7 about ACLs.


reqrep  <search> <string> [{if | unless} <cond>]
reqirep <search> <string> [{if | unless} <cond>]   (ignore case)
  Replace a regular expression with a string in an HTTP request line
  May be used in sections :   defaults | frontend | listen | backend
                                 no    |    yes   |   yes  |   yes
  Arguments :
    <search>  is the regular expression applied to HTTP headers and to the
              request line. This is an extended regular expression. Parenthesis
              grouping is supported and no preliminary backslash is required.
              Any space or known delimiter must be escaped using a backslash
              ('\'). The pattern applies to a full line at a time. The "reqrep"
              keyword strictly matches case while "reqirep" ignores case.

    <string>  is the complete line to be added. Any space or known delimiter
              must be escaped using a backslash ('\'). References to matched
              pattern groups are possible using the common \N form, with N
              being a single digit between 0 and 9. Please refer to section
              6 about HTTP header manipulation for more information.

    <cond>    is an optional matching condition built from ACLs. It makes it
              possible to ignore this rule when other conditions are not met.

  Any line matching extended regular expression <search> in the request (both
  the request line and header lines) will be completely replaced with <string>.
  Most common use of this is to rewrite URLs or domain names in "Host" headers.

  Header transformations only apply to traffic which passes through HAProxy,
  and not to traffic generated by HAProxy, such as health-checks or error
  responses. Note that for increased readability, it is suggested to add enough
  spaces between the request and the response. Keep in mind that URLs in
  request line are case-sensitive while header names are not.

  Example :
     # replace "/static/" with "/" at the beginning of any request path.
     reqrep ^([^\ :]*)\ /static/(.*)     \1\ /\2
     # replace "www.mydomain.com" with "www" in the host name.
     reqirep ^Host:\ www.mydomain.com   Host:\ www

  See also: "reqadd", "reqdel", "rsprep", section 6 about HTTP header
            manipulation, and section 7 about ACLs.


reqtarpit  <search> [{if | unless} <cond>]
reqitarpit <search> [{if | unless} <cond>]  (ignore case)
  Tarpit an HTTP request containing a line matching a regular expression
  May be used in sections :   defaults | frontend | listen | backend
                                 no    |    yes   |   yes  |   yes
  Arguments :
    <search>  is the regular expression applied to HTTP headers and to the
              request line. This is an extended regular expression. Parenthesis
              grouping is supported and no preliminary backslash is required.
              Any space or known delimiter must be escaped using a backslash
              ('\'). The pattern applies to a full line at a time. The
              "reqtarpit" keyword strictly matches case while "reqitarpit"
              ignores case.

    <cond>    is an optional matching condition built from ACLs. It makes it
              possible to ignore this rule when other conditions are not met.

  A request containing any line which matches extended regular expression
  <search> will be tarpitted, which means that it will connect to nowhere, will
  be kept open for a pre-defined time, then will return an HTTP error 500 so
  that the attacker does not suspect it has been tarpitted. The status 500 will
  be reported in the logs, but the completion flags will indicate "PT". The
  delay is defined by "timeout tarpit", or "timeout connect" if the former is
  not set.

  The goal of the tarpit is to slow down robots attacking servers with
  identifiable requests. Many robots limit their outgoing number of connections
  and stay connected waiting for a reply which can take several minutes to
  come. Depending on the environment and attack, it may be particularly
  efficient at reducing the load on the network and firewalls.

  Examples :
     # ignore user-agents reporting any flavour of "Mozilla" or "MSIE", but
     # block all others.
     reqipass   ^User-Agent:\.*(Mozilla|MSIE)
     reqitarpit ^User-Agent:

     # block bad guys
     acl badguys src 10.1.0.3 172.16.13.20/28
     reqitarpit . if badguys

  See also: "reqallow", "reqdeny", "reqpass", section 6 about HTTP header
            manipulation, and section 7 about ACLs.


retries <value>
  Set the number of retries to perform on a server after a connection failure
  May be used in sections:    defaults | frontend | listen | backend
                                 yes   |    no    |   yes  |   yes
  Arguments :
    <value>   is the number of times a connection attempt should be retried on
              a server when a connection either is refused or times out. The
              default value is 3.

  It is important to understand that this value applies to the number of
  connection attempts, not full requests. When a connection has effectively
  been established to a server, there will be no more retry.

  In order to avoid immediate reconnections to a server which is restarting,
  a turn-around timer of 1 second is applied before a retry occurs.

  When "option redispatch" is set, the last retry may be performed on another
  server even if a cookie references a different server.

  See also : "option redispatch"


rspadd <string> [{if | unless} <cond>]
  Add a header at the end of the HTTP response
  May be used in sections :   defaults | frontend | listen | backend
                                 no    |    yes   |   yes  |   yes
  Arguments :
    <string>  is the complete line to be added. Any space or known delimiter
              must be escaped using a backslash ('\'). Please refer to section
              6 about HTTP header manipulation for more information.

    <cond>    is an optional matching condition built from ACLs. It makes it
              possible to ignore this rule when other conditions are not met.

  A new line consisting in <string> followed by a line feed will be added after
  the last header of an HTTP response.

  Header transformations only apply to traffic which passes through HAProxy,
  and not to traffic generated by HAProxy, such as health-checks or error
  responses.

  See also: "reqadd", section 6 about HTTP header manipulation, and section 7
            about ACLs.


rspdel  <search> [{if | unless} <cond>]
rspidel <search> [{if | unless} <cond>]  (ignore case)
  Delete all headers matching a regular expression in an HTTP response
  May be used in sections :   defaults | frontend | listen | backend
                                 no    |    yes   |   yes  |   yes
  Arguments :
    <search>  is the regular expression applied to HTTP headers and to the
              response line. This is an extended regular expression, so
              parenthesis grouping is supported and no preliminary backslash
              is required. Any space or known delimiter must be escaped using
              a backslash ('\'). The pattern applies to a full line at a time.
              The "rspdel" keyword strictly matches case while "rspidel"
              ignores case.

    <cond>    is an optional matching condition built from ACLs. It makes it
              possible to ignore this rule when other conditions are not met.

  Any header line matching extended regular expression <search> in the response
  will be completely deleted. Most common use of this is to remove unwanted
  and/or sensitive headers or cookies from a response before passing it to the
  client.

  Header transformations only apply to traffic which passes through HAProxy,
  and not to traffic generated by HAProxy, such as health-checks or error
  responses. Keep in mind that header names are not case-sensitive.

  Example :
     # remove the Server header from responses
     reqidel ^Server:.*

  See also: "rspadd", "rsprep", "reqdel", section 6 about HTTP header
            manipulation, and section 7 about ACLs.


rspdeny  <search> [{if | unless} <cond>]
rspideny <search> [{if | unless} <cond>]  (ignore case)
  Block an HTTP response if a line matches a regular expression
  May be used in sections :   defaults | frontend | listen | backend
                                 no    |    yes   |   yes  |   yes
  Arguments :
    <search>  is the regular expression applied to HTTP headers and to the
              response line. This is an extended regular expression, so
              parenthesis grouping is supported and no preliminary backslash
              is required. Any space or known delimiter must be escaped using
              a backslash ('\'). The pattern applies to a full line at a time.
              The "rspdeny" keyword strictly matches case while "rspideny"
              ignores case.

    <cond>    is an optional matching condition built from ACLs. It makes it
              possible to ignore this rule when other conditions are not met.

  A response containing any line which matches extended regular expression
  <search> will mark the request as denied. The test applies both to the
  response line and to response headers. Keep in mind that header names are not
  case-sensitive.

  Main use of this keyword is to prevent sensitive information leak and to
  block the response before it reaches the client. If a response is denied, it
  will be replaced with an HTTP 502 error so that the client never retrieves
  any sensitive data.

  It is easier, faster and more powerful to use ACLs to write access policies.
  Rspdeny should be avoided in new designs.

  Example :
     # Ensure that no content type matching ms-word will leak
     rspideny  ^Content-type:\.*/ms-word

  See also: "reqdeny", "acl", "block", section 6 about HTTP header manipulation
            and section 7 about ACLs.


rsprep  <search> <string> [{if | unless} <cond>]
rspirep <search> <string> [{if | unless} <cond>]  (ignore case)
  Replace a regular expression with a string in an HTTP response line
  May be used in sections :   defaults | frontend | listen | backend
                                 no    |    yes   |   yes  |   yes
  Arguments :
    <search>  is the regular expression applied to HTTP headers and to the
              response line. This is an extended regular expression, so
              parenthesis grouping is supported and no preliminary backslash
              is required. Any space or known delimiter must be escaped using
              a backslash ('\'). The pattern applies to a full line at a time.
              The "rsprep" keyword strictly matches case while "rspirep"
              ignores case.

    <string>  is the complete line to be added. Any space or known delimiter
              must be escaped using a backslash ('\'). References to matched
              pattern groups are possible using the common \N form, with N
              being a single digit between 0 and 9. Please refer to section
              6 about HTTP header manipulation for more information.

    <cond>    is an optional matching condition built from ACLs. It makes it
              possible to ignore this rule when other conditions are not met.

  Any line matching extended regular expression <search> in the response (both
  the response line and header lines) will be completely replaced with
  <string>. Most common use of this is to rewrite Location headers.

  Header transformations only apply to traffic which passes through HAProxy,
  and not to traffic generated by HAProxy, such as health-checks or error
  responses. Note that for increased readability, it is suggested to add enough
  spaces between the request and the response. Keep in mind that header names
  are not case-sensitive.

  Example :
     # replace "Location: 127.0.0.1:8080" with "Location: www.mydomain.com"
     rspirep ^Location:\ 127.0.0.1:8080    Location:\ www.mydomain.com

  See also: "rspadd", "rspdel", "reqrep", section 6 about HTTP header
            manipulation, and section 7 about ACLs.


server <name> <address>[:[port]] [param*]
  Declare a server in a backend
  May be used in sections :   defaults | frontend | listen | backend
                                 no    |    no    |   yes  |   yes
  Arguments :
    <name>    is the internal name assigned to this server. This name will
              appear in logs and alerts.  If "http-send-server-name" is
              set, it will be added to the request header sent to the server.

    <address> is the IPv4 or IPv6 address of the server. Alternatively, a
              resolvable hostname is supported, but this name will be resolved
              during start-up. Address "0.0.0.0" or "*" has a special meaning.
              It indicates that the connection will be forwarded to the same IP
              address as the one from the client connection. This is useful in
              transparent proxy architectures where the client's connection is
              intercepted and haproxy must forward to the original destination
              address. This is more or less what the "transparent" keyword does
              except that with a server it's possible to limit concurrency and
              to report statistics.

    <ports>   is an optional port specification. If set, all connections will
              be sent to this port. If unset, the same port the client
              connected to will be used. The port may also be prefixed by a "+"
              or a "-". In this case, the server's port will be determined by
              adding this value to the client's port.

    <param*>  is a list of parameters for this server. The "server" keywords
              accepts an important number of options and has a complete section
              dedicated to it. Please refer to section 5 for more details.

  Examples :
        server first  10.1.1.1:1080 cookie first  check inter 1000
        server second 10.1.1.2:1080 cookie second check inter 1000

  See also: "default-server", "http-send-name-header" and section 5 about
             server options


source <addr>[:<port>] [usesrc { <addr2>[:<port2>] | client | clientip } ]
source <addr>[:<port>] [usesrc { <addr2>[:<port2>] | hdr_ip(<hdr>[,<occ>]) } ]
source <addr>[:<port>] [interface <name>]
  Set the source address for outgoing connections
  May be used in sections :   defaults | frontend | listen | backend
                                 yes   |    no    |   yes  |   yes
  Arguments :
    <addr>    is the IPv4 address HAProxy will bind to before connecting to a
              server. This address is also used as a source for health checks.
              The default value of 0.0.0.0 means that the system will select
              the most appropriate address to reach its destination.

    <port>    is an optional port. It is normally not needed but may be useful
              in some very specific contexts. The default value of zero means
              the system will select a free port. Note that port ranges are not
              supported in the backend. If you want to force port ranges, you
              have to specify them on each "server" line.

    <addr2>   is the IP address to present to the server when connections are
              forwarded in full transparent proxy mode. This is currently only
              supported on some patched Linux kernels. When this address is
              specified, clients connecting to the server will be presented
              with this address, while health checks will still use the address
              <addr>.

    <port2>   is the optional port to present to the server when connections
              are forwarded in full transparent proxy mode (see <addr2> above).
              The default value of zero means the system will select a free
              port.

    <hdr>     is the name of a HTTP header in which to fetch the IP to bind to.
              This is the name of a comma-separated header list which can
              contain multiple IP addresses. By default, the last occurrence is
              used. This is designed to work with the X-Forwarded-For header
              and to automatically bind to the the client's IP address as seen
              by previous proxy, typically Stunnel. In order to use another
              occurrence from the last one, please see the <occ> parameter
              below. When the header (or occurrence) is not found, no binding
              is performed so that the proxy's default IP address is used. Also
              keep in mind that the header name is case insensitive, as for any
              HTTP header.

    <occ>     is the occurrence number of a value to be used in a multi-value
              header. This is to be used in conjunction with "hdr_ip(<hdr>)",
              in order to specificy which occurrence to use for the source IP
              address. Positive values indicate a position from the first
              occurrence, 1 being the first one. Negative values indicate
              positions relative to the last one, -1 being the last one. This
              is helpful for situations where an X-Forwarded-For header is set
              at the entry point of an infrastructure and must be used several
              proxy layers away. When this value is not specified, -1 is
              assumed. Passing a zero here disables the feature.

    <name>    is an optional interface name to which to bind to for outgoing
              traffic. On systems supporting this features (currently, only
              Linux), this allows one to bind all traffic to the server to
              this interface even if it is not the one the system would select
              based on routing tables. This should be used with extreme care.
              Note that using this option requires root privileges.

  The "source" keyword is useful in complex environments where a specific
  address only is allowed to connect to the servers. It may be needed when a
  private address must be used through a public gateway for instance, and it is
  known that the system cannot determine the adequate source address by itself.

  An extension which is available on certain patched Linux kernels may be used
  through the "usesrc" optional keyword. It makes it possible to connect to the
  servers with an IP address which does not belong to the system itself. This
  is called "full transparent proxy mode". For this to work, the destination
  servers have to route their traffic back to this address through the machine
  running HAProxy, and IP forwarding must generally be enabled on this machine.

  In this "full transparent proxy" mode, it is possible to force a specific IP
  address to be presented to the servers. This is not much used in fact. A more
  common use is to tell HAProxy to present the client's IP address. For this,
  there are two methods :

    - present the client's IP and port addresses. This is the most transparent
      mode, but it can cause problems when IP connection tracking is enabled on
      the machine, because a same connection may be seen twice with different
      states. However, this solution presents the huge advantage of not
      limiting the system to the 64k outgoing address+port couples, because all
      of the client ranges may be used.

    - present only the client's IP address and select a spare port. This
      solution is still quite elegant but slightly less transparent (downstream
      firewalls logs will not match upstream's). It also presents the downside
      of limiting the number of concurrent connections to the usual 64k ports.
      However, since the upstream and downstream ports are different, local IP
      connection tracking on the machine will not be upset by the reuse of the
      same session.

  Note that depending on the transparent proxy technology used, it may be
  required to force the source address. In fact, cttproxy version 2 requires an
  IP address in <addr> above, and does not support setting of "0.0.0.0" as the
  IP address because it creates NAT entries which much match the exact outgoing
  address. Tproxy version 4 and some other kernel patches which work in pure
  forwarding mode generally will not have this limitation.

  This option sets the default source for all servers in the backend. It may
  also be specified in a "defaults" section. Finer source address specification
  is possible at the server level using the "source" server option. Refer to
  section 5 for more information.

  Examples :
        backend private
            # Connect to the servers using our 192.168.1.200 source address
            source 192.168.1.200

        backend transparent_ssl1
            # Connect to the SSL farm from the client's source address
            source 192.168.1.200 usesrc clientip

        backend transparent_ssl2
            # Connect to the SSL farm from the client's source address and port
            # not recommended if IP conntrack is present on the local machine.
            source 192.168.1.200 usesrc client

        backend transparent_ssl3
            # Connect to the SSL farm from the client's source address. It
            # is more conntrack-friendly.
            source 192.168.1.200 usesrc clientip

        backend transparent_smtp
            # Connect to the SMTP farm from the client's source address/port
            # with Tproxy version 4.
            source 0.0.0.0 usesrc clientip

        backend transparent_http
            # Connect to the servers using the client's IP as seen by previous
            # proxy.
            source 0.0.0.0 usesrc hdr_ip(x-forwarded-for,-1)

  See also : the "source" server option in section 5, the Tproxy patches for
             the Linux kernel on www.balabit.com, the "bind" keyword.


srvtimeout <timeout> (deprecated)
  Set the maximum inactivity time on the server side.
  May be used in sections :   defaults | frontend | listen | backend
                                 yes   |    no    |   yes  |   yes
  Arguments :
    <timeout> is the timeout value specified in milliseconds by default, but
              can be in any other unit if the number is suffixed by the unit,
              as explained at the top of this document.

  The inactivity timeout applies when the server is expected to acknowledge or
  send data. In HTTP mode, this timeout is particularly important to consider
  during the first phase of the server's response, when it has to send the
  headers, as it directly represents the server's processing time for the
  request. To find out what value to put there, it's often good to start with
  what would be considered as unacceptable response times, then check the logs
  to observe the response time distribution, and adjust the value accordingly.

  The value is specified in milliseconds by default, but can be in any other
  unit if the number is suffixed by the unit, as specified at the top of this
  document. In TCP mode (and to a lesser extent, in HTTP mode), it is highly
  recommended that the client timeout remains equal to the server timeout in
  order to avoid complex situations to debug. Whatever the expected server
  response times, it is a good practice to cover at least one or several TCP
  packet losses by specifying timeouts that are slightly above multiples of 3
  seconds (eg: 4 or 5 seconds minimum).

  This parameter is specific to backends, but can be specified once for all in
  "defaults" sections. This is in fact one of the easiest solutions not to
  forget about it. An unspecified timeout results in an infinite timeout, which
  is not recommended. Such a usage is accepted and works but reports a warning
  during startup because it may results in accumulation of expired sessions in
  the system if the system's timeouts are not configured either.

  This parameter is provided for compatibility but is currently deprecated.
  Please use "timeout server" instead.

  See also : "timeout server", "timeout tunnel", "timeout client" and
             "clitimeout".


stats admin { if | unless } <cond>
  Enable statistics admin level if/unless a condition is matched
  May be used in sections :   defaults | frontend | listen | backend
                                 no    |    no    |   yes  |   yes

  This statement enables the statistics admin level if/unless a condition is
  matched.

  The admin level allows to enable/disable servers from the web interface. By
  default, statistics page is read-only for security reasons.

  Note : Consider not using this feature in multi-process mode (nbproc > 1)
         unless you know what you do : memory is not shared between the
         processes, which can result in random behaviours.

  Currently, the POST request is limited to the buffer size minus the reserved
  buffer space, which means that if the list of servers is too long, the
  request won't be processed. It is recommended to alter few servers at a
  time.

  Example :
    # statistics admin level only for localhost
    backend stats_localhost
        stats enable
        stats admin if LOCALHOST

  Example :
    # statistics admin level always enabled because of the authentication
    backend stats_auth
        stats enable
        stats auth  admin:AdMiN123
        stats admin if TRUE

  Example :
    # statistics admin level depends on the authenticated user
    userlist stats-auth
        group admin    users admin
        user  admin    insecure-password AdMiN123
        group readonly users haproxy
        user  haproxy  insecure-password haproxy

    backend stats_auth
        stats enable
        acl AUTH       http_auth(stats-auth)
        acl AUTH_ADMIN http_auth_group(stats-auth) admin
        stats http-request auth unless AUTH
        stats admin if AUTH_ADMIN

  See also : "stats enable", "stats auth", "stats http-request", "nbproc",
             "bind-process", section 3.4 about userlists and section 7 about
             ACL usage.


stats auth <user>:<passwd>
  Enable statistics with authentication and grant access to an account
  May be used in sections :   defaults | frontend | listen | backend
                                 yes   |    no    |   yes  |   yes
  Arguments :
    <user>    is a user name to grant access to

    <passwd>  is the cleartext password associated to this user

  This statement enables statistics with default settings, and restricts access
  to declared users only. It may be repeated as many times as necessary to
  allow as many users as desired. When a user tries to access the statistics
  without a valid account, a "401 Forbidden" response will be returned so that
  the browser asks the user to provide a valid user and password. The real
  which will be returned to the browser is configurable using "stats realm".

  Since the authentication method is HTTP Basic Authentication, the passwords
  circulate in cleartext on the network. Thus, it was decided that the
  configuration file would also use cleartext passwords to remind the users
  that those ones should not be sensitive and not shared with any other account.

  It is also possible to reduce the scope of the proxies which appear in the
  report using "stats scope".

  Though this statement alone is enough to enable statistics reporting, it is
  recommended to set all other settings in order to avoid relying on default
  unobvious parameters.

  Example :
    # public access (limited to this backend only)
    backend public_www
        server srv1 192.168.0.1:80
        stats enable
        stats hide-version
        stats scope   .
        stats uri     /admin?stats
        stats realm   Haproxy\ Statistics
        stats auth    admin1:AdMiN123
        stats auth    admin2:AdMiN321

    # internal monitoring access (unlimited)
    backend private_monitoring
        stats enable
        stats uri     /admin?stats
        stats refresh 5s

  See also : "stats enable", "stats realm", "stats scope", "stats uri"


stats enable
  Enable statistics reporting with default settings
  May be used in sections :   defaults | frontend | listen | backend
                                 yes   |    no    |   yes  |   yes
  Arguments : none

  This statement enables statistics reporting with default settings defined
  at build time. Unless stated otherwise, these settings are used :
    - stats uri   : /haproxy?stats
    - stats realm : "HAProxy Statistics"
    - stats auth  : no authentication
    - stats scope : no restriction

  Though this statement alone is enough to enable statistics reporting, it is
  recommended to set all other settings in order to avoid relying on default
  unobvious parameters.

  Example :
    # public access (limited to this backend only)
    backend public_www
        server srv1 192.168.0.1:80
        stats enable
        stats hide-version
        stats scope   .
        stats uri     /admin?stats
        stats realm   Haproxy\ Statistics
        stats auth    admin1:AdMiN123
        stats auth    admin2:AdMiN321

    # internal monitoring access (unlimited)
    backend private_monitoring
        stats enable
        stats uri     /admin?stats
        stats refresh 5s

  See also : "stats auth", "stats realm", "stats uri"


stats hide-version
  Enable statistics and hide HAProxy version reporting
  May be used in sections :   defaults | frontend | listen | backend
                                 yes   |    no    |   yes  |   yes
  Arguments : none

  By default, the stats page reports some useful status information along with
  the statistics. Among them is HAProxy's version. However, it is generally
  considered dangerous to report precise version to anyone, as it can help them
  target known weaknesses with specific attacks. The "stats hide-version"
  statement removes the version from the statistics report. This is recommended
  for public sites or any site with a weak login/password.

  Though this statement alone is enough to enable statistics reporting, it is
  recommended to set all other settings in order to avoid relying on default
  unobvious parameters.

  Example :
    # public access (limited to this backend only)
    backend public_www
        server srv1 192.168.0.1:80
        stats enable
        stats hide-version
        stats scope   .
        stats uri     /admin?stats
        stats realm   Haproxy\ Statistics
        stats auth    admin1:AdMiN123
        stats auth    admin2:AdMiN321

    # internal monitoring access (unlimited)
    backend private_monitoring
        stats enable
        stats uri     /admin?stats
        stats refresh 5s

  See also : "stats auth", "stats enable", "stats realm", "stats uri"


stats http-request { allow | deny | auth [realm <realm>] }
             [ { if | unless } <condition> ]
  Access control for statistics

  May be used in sections:   defaults | frontend | listen | backend
                                no    |    no    |   yes  |   yes

  As "http-request", these set of options allow to fine control access to
  statistics. Each option may be followed by if/unless and acl.
  First option with matched condition (or option without condition) is final.
  For "deny" a 403 error will be returned, for "allow" normal processing is
  performed, for "auth" a 401/407 error code is returned so the client
  should be asked to enter a username and password.

  There is no fixed limit to the number of http-request statements per
  instance.

  See also : "http-request", section 3.4 about userlists and section 7
             about ACL usage.


stats realm <realm>
  Enable statistics and set authentication realm
  May be used in sections :   defaults | frontend | listen | backend
                                 yes   |    no    |   yes  |   yes
  Arguments :
    <realm>   is the name of the HTTP Basic Authentication realm reported to
              the browser. The browser uses it to display it in the pop-up
              inviting the user to enter a valid username and password.

  The realm is read as a single word, so any spaces in it should be escaped
  using a backslash ('\').

  This statement is useful only in conjunction with "stats auth" since it is
  only related to authentication.

  Though this statement alone is enough to enable statistics reporting, it is
  recommended to set all other settings in order to avoid relying on default
  unobvious parameters.

  Example :
    # public access (limited to this backend only)
    backend public_www
        server srv1 192.168.0.1:80
        stats enable
        stats hide-version
        stats scope   .
        stats uri     /admin?stats
        stats realm   Haproxy\ Statistics
        stats auth    admin1:AdMiN123
        stats auth    admin2:AdMiN321

    # internal monitoring access (unlimited)
    backend private_monitoring
        stats enable
        stats uri     /admin?stats
        stats refresh 5s

  See also : "stats auth", "stats enable", "stats uri"


stats refresh <delay>
  Enable statistics with automatic refresh
  May be used in sections :   defaults | frontend | listen | backend
                                 yes   |    no    |   yes  |   yes
  Arguments :
    <delay>   is the suggested refresh delay, specified in seconds, which will
              be returned to the browser consulting the report page. While the
              browser is free to apply any delay, it will generally respect it
              and refresh the page this every seconds. The refresh interval may
              be specified in any other non-default time unit, by suffixing the
              unit after the value, as explained at the top of this document.

  This statement is useful on monitoring displays with a permanent page
  reporting the load balancer's activity. When set, the HTML report page will
  include a link "refresh"/"stop refresh" so that the user can select whether
  he wants automatic refresh of the page or not.

  Though this statement alone is enough to enable statistics reporting, it is
  recommended to set all other settings in order to avoid relying on default
  unobvious parameters.

  Example :
    # public access (limited to this backend only)
    backend public_www
        server srv1 192.168.0.1:80
        stats enable
        stats hide-version
        stats scope   .
        stats uri     /admin?stats
        stats realm   Haproxy\ Statistics
        stats auth    admin1:AdMiN123
        stats auth    admin2:AdMiN321

    # internal monitoring access (unlimited)
    backend private_monitoring
        stats enable
        stats uri     /admin?stats
        stats refresh 5s

  See also : "stats auth", "stats enable", "stats realm", "stats uri"


stats scope { <name> | "." }
  Enable statistics and limit access scope
  May be used in sections :   defaults | frontend | listen | backend
                                 yes   |    no    |   yes  |   yes
  Arguments :
    <name>    is the name of a listen, frontend or backend section to be
              reported. The special name "." (a single dot) designates the
              section in which the statement appears.

  When this statement is specified, only the sections enumerated with this
  statement will appear in the report. All other ones will be hidden. This
  statement may appear as many times as needed if multiple sections need to be
  reported. Please note that the name checking is performed as simple string
  comparisons, and that it is never checked that a give section name really
  exists.

  Though this statement alone is enough to enable statistics reporting, it is
  recommended to set all other settings in order to avoid relying on default
  unobvious parameters.

  Example :
    # public access (limited to this backend only)
    backend public_www
        server srv1 192.168.0.1:80
        stats enable
        stats hide-version
        stats scope   .
        stats uri     /admin?stats
        stats realm   Haproxy\ Statistics
        stats auth    admin1:AdMiN123
        stats auth    admin2:AdMiN321

    # internal monitoring access (unlimited)
    backend private_monitoring
        stats enable
        stats uri     /admin?stats
        stats refresh 5s

  See also : "stats auth", "stats enable", "stats realm", "stats uri"


stats show-desc [ <desc> ]
  Enable reporting of a description on the statistics page.
  May be used in sections :   defaults | frontend | listen | backend
                                 yes   |    no    |   yes  |   yes

    <desc>    is an optional description to be reported. If unspecified, the
              description from global section is automatically used instead.

  This statement is useful for users that offer shared services to their
  customers, where node or description should be different for each customer.

  Though this statement alone is enough to enable statistics reporting, it is
  recommended to set all other settings in order to avoid relying on default
  unobvious parameters.  By default description is not shown.

  Example :
    # internal monitoring access (unlimited)
    backend private_monitoring
        stats enable
        stats show-desc Master node for Europe, Asia, Africa
        stats uri       /admin?stats
        stats refresh   5s

  See also: "show-node", "stats enable", "stats uri" and "description" in
            global section.


stats show-legends
  Enable reporting additional informations on the statistics page :
    - cap: capabilities (proxy)
    - mode: one of tcp, http or health (proxy)
    - id: SNMP ID (proxy, socket, server)
    - IP (socket, server)
    - cookie (backend, server)

  Though this statement alone is enough to enable statistics reporting, it is
  recommended to set all other settings in order to avoid relying on default
  unobvious parameters.  Default behaviour is not to show this information.

  See also: "stats enable", "stats uri".


stats show-node [ <name> ]
  Enable reporting of a host name on the statistics page.
  May be used in sections :   defaults | frontend | listen | backend
                                 yes   |    no    |   yes  |   yes
  Arguments:
    <name>    is an optional name to be reported. If unspecified, the
              node name from global section is automatically used instead.

  This statement is useful for users that offer shared services to their
  customers, where node or description might be different on a stats page
  provided for each customer.  Default behaviour is not to show host name.

  Though this statement alone is enough to enable statistics reporting, it is
  recommended to set all other settings in order to avoid relying on default
  unobvious parameters.

  Example:
    # internal monitoring access (unlimited)
    backend private_monitoring
        stats enable
        stats show-node Europe-1
        stats uri       /admin?stats
        stats refresh   5s

  See also: "show-desc", "stats enable", "stats uri", and "node" in global
            section.


stats uri <prefix>
  Enable statistics and define the URI prefix to access them
  May be used in sections :   defaults | frontend | listen | backend
                                 yes   |    no    |   yes  |   yes
  Arguments :
    <prefix>  is the prefix of any URI which will be redirected to stats. This
              prefix may contain a question mark ('?') to indicate part of a
              query string.

  The statistics URI is intercepted on the relayed traffic, so it appears as a
  page within the normal application. It is strongly advised to ensure that the
  selected URI will never appear in the application, otherwise it will never be
  possible to reach it in the application.

  The default URI compiled in haproxy is "/haproxy?stats", but this may be
  changed at build time, so it's better to always explicitly specify it here.
  It is generally a good idea to include a question mark in the URI so that
  intermediate proxies refrain from caching the results. Also, since any string
  beginning with the prefix will be accepted as a stats request, the question
  mark helps ensuring that no valid URI will begin with the same words.

  It is sometimes very convenient to use "/" as the URI prefix, and put that
  statement in a "listen" instance of its own. That makes it easy to dedicate
  an address or a port to statistics only.

  Though this statement alone is enough to enable statistics reporting, it is
  recommended to set all other settings in order to avoid relying on default
  unobvious parameters.

  Example :
    # public access (limited to this backend only)
    backend public_www
        server srv1 192.168.0.1:80
        stats enable
        stats hide-version
        stats scope   .
        stats uri     /admin?stats
        stats realm   Haproxy\ Statistics
        stats auth    admin1:AdMiN123
        stats auth    admin2:AdMiN321

    # internal monitoring access (unlimited)
    backend private_monitoring
        stats enable
        stats uri     /admin?stats
        stats refresh 5s

  See also : "stats auth", "stats enable", "stats realm"


stick match <pattern> [table <table>] [{if | unless} <cond>]
  Define a request pattern matching condition to stick a user to a server
  May be used in sections :   defaults | frontend | listen | backend
                                 no    |    no    |   yes  |   yes

  Arguments :
    <pattern>  is a pattern extraction rule as described in section 7.8. It
               describes what elements of the incoming request or connection
               will be analysed in the hope to find a matching entry in a
               stickiness table. This rule is mandatory.

    <table>    is an optional stickiness table name. If unspecified, the same
               backend's table is used. A stickiness table is declared using
               the "stick-table" statement.

    <cond>     is an optional matching condition. It makes it possible to match
               on a certain criterion only when other conditions are met (or
               not met). For instance, it could be used to match on a source IP
               address except when a request passes through a known proxy, in
               which case we'd match on a header containing that IP address.

  Some protocols or applications require complex stickiness rules and cannot
  always simply rely on cookies nor hashing. The "stick match" statement
  describes a rule to extract the stickiness criterion from an incoming request
  or connection. See section 7 for a complete list of possible patterns and
  transformation rules.

  The table has to be declared using the "stick-table" statement. It must be of
  a type compatible with the pattern. By default it is the one which is present
  in the same backend. It is possible to share a table with other backends by
  referencing it using the "table" keyword. If another table is referenced,
  the server's ID inside the backends are used. By default, all server IDs
  start at 1 in each backend, so the server ordering is enough. But in case of
  doubt, it is highly recommended to force server IDs using their "id" setting.

  It is possible to restrict the conditions where a "stick match" statement
  will apply, using "if" or "unless" followed by a condition. See section 7 for
  ACL based conditions.

  There is no limit on the number of "stick match" statements. The first that
  applies and matches will cause the request to be directed to the same server
  as was used for the request which created the entry. That way, multiple
  matches can be used as fallbacks.

  The stick rules are checked after the persistence cookies, so they will not
  affect stickiness if a cookie has already been used to select a server. That
  way, it becomes very easy to insert cookies and match on IP addresses in
  order to maintain stickiness between HTTP and HTTPS.

  Note : Consider not using this feature in multi-process mode (nbproc > 1)
         unless you know what you do : memory is not shared between the
         processes, which can result in random behaviours.

  Example :
    # forward SMTP users to the same server they just used for POP in the
    # last 30 minutes
    backend pop
        mode tcp
        balance roundrobin
        stick store-request src
        stick-table type ip size 200k expire 30m
        server s1 192.168.1.1:110
        server s2 192.168.1.1:110

    backend smtp
        mode tcp
        balance roundrobin
        stick match src table pop
        server s1 192.168.1.1:25
        server s2 192.168.1.1:25

  See also : "stick-table", "stick on", "nbproc", "bind-process" and section 7
             about ACLs and pattern extraction.


stick on <pattern> [table <table>] [{if | unless} <condition>]
  Define a request pattern to associate a user to a server
  May be used in sections :   defaults | frontend | listen | backend
                                 no    |    no    |   yes  |   yes

  Note : This form is exactly equivalent to "stick match" followed by
         "stick store-request", all with the same arguments. Please refer
         to both keywords for details. It is only provided as a convenience
         for writing more maintainable configurations.

  Note : Consider not using this feature in multi-process mode (nbproc > 1)
         unless you know what you do : memory is not shared between the
         processes, which can result in random behaviours.

  Examples :
    # The following form ...
    stick on src table pop if !localhost

    # ...is strictly equivalent to this one :
    stick match src table pop if !localhost
    stick store-request src table pop if !localhost


    # Use cookie persistence for HTTP, and stick on source address for HTTPS as
    # well as HTTP without cookie. Share the same table between both accesses.
    backend http
        mode http
        balance roundrobin
        stick on src table https
        cookie SRV insert indirect nocache
        server s1 192.168.1.1:80 cookie s1
        server s2 192.168.1.1:80 cookie s2

    backend https
        mode tcp
        balance roundrobin
        stick-table type ip size 200k expire 30m
        stick on src
        server s1 192.168.1.1:443
        server s2 192.168.1.1:443

  See also : "stick match", "stick store-request", "nbproc" and "bind-process".


stick store-request <pattern> [table <table>] [{if | unless} <condition>]
  Define a request pattern used to create an entry in a stickiness table
  May be used in sections :   defaults | frontend | listen | backend
                                 no    |    no    |   yes  |   yes

  Arguments :
    <pattern>  is a pattern extraction rule as described in section 7.8. It
               describes what elements of the incoming request or connection
               will be analysed, extracted and stored in the table once a
               server is selected.

    <table>    is an optional stickiness table name. If unspecified, the same
               backend's table is used. A stickiness table is declared using
               the "stick-table" statement.

    <cond>     is an optional storage condition. It makes it possible to store
               certain criteria only when some conditions are met (or not met).
               For instance, it could be used to store the source IP address
               except when the request passes through a known proxy, in which
               case we'd store a converted form of a header containing that IP
               address.

  Some protocols or applications require complex stickiness rules and cannot
  always simply rely on cookies nor hashing. The "stick store-request" statement
  describes a rule to decide what to extract from the request and when to do
  it, in order to store it into a stickiness table for further requests to
  match it using the "stick match" statement. Obviously the extracted part must
  make sense and have a chance to be matched in a further request. Storing a
  client's IP address for instance often makes sense. Storing an ID found in a
  URL parameter also makes sense. Storing a source port will almost never make
  any sense because it will be randomly matched. See section 7 for a complete
  list of possible patterns and transformation rules.

  The table has to be declared using the "stick-table" statement. It must be of
  a type compatible with the pattern. By default it is the one which is present
  in the same backend. It is possible to share a table with other backends by
  referencing it using the "table" keyword. If another table is referenced,
  the server's ID inside the backends are used. By default, all server IDs
  start at 1 in each backend, so the server ordering is enough. But in case of
  doubt, it is highly recommended to force server IDs using their "id" setting.

  It is possible to restrict the conditions where a "stick store-request"
  statement will apply, using "if" or "unless" followed by a condition. This
  condition will be evaluated while parsing the request, so any criteria can be
  used. See section 7 for ACL based conditions.

  There is no limit on the number of "stick store-request" statements, but
  there is a limit of 8 simultaneous stores per request or response. This
  makes it possible to store up to 8 criteria, all extracted from either the
  request or the response, regardless of the number of rules. Only the 8 first
  ones which match will be kept. Using this, it is possible to feed multiple
  tables at once in the hope to increase the chance to recognize a user on
  another protocol or access method.

  The "store-request" rules are evaluated once the server connection has been
  established, so that the table will contain the real server that processed
  the request.

  Note : Consider not using this feature in multi-process mode (nbproc > 1)
         unless you know what you do : memory is not shared between the
         processes, which can result in random behaviours.

  Example :
    # forward SMTP users to the same server they just used for POP in the
    # last 30 minutes
    backend pop
        mode tcp
        balance roundrobin
        stick store-request src
        stick-table type ip size 200k expire 30m
        server s1 192.168.1.1:110
        server s2 192.168.1.1:110

    backend smtp
        mode tcp
        balance roundrobin
        stick match src table pop
        server s1 192.168.1.1:25
        server s2 192.168.1.1:25

  See also : "stick-table", "stick on", "nbproc", "bind-process" and section 7
             about ACLs and pattern extraction.


stick-table type {ip | integer | string [len <length>] | binary [len <length>]}
            size <size> [expire <expire>] [nopurge] [peers <peersect>]
            [store <data_type>]*
  Configure the stickiness table for the current backend
  May be used in sections :   defaults | frontend | listen | backend
                                 no    |    yes   |   yes  |   yes

  Arguments :
    ip         a table declared with "type ip" will only store IPv4 addresses.
               This form is very compact (about 50 bytes per entry) and allows
               very fast entry lookup and stores with almost no overhead. This
               is mainly used to store client source IP addresses.

    ipv6       a table declared with "type ipv6" will only store IPv6 addresses.
               This form is very compact (about 60 bytes per entry) and allows
               very fast entry lookup and stores with almost no overhead. This
               is mainly used to store client source IP addresses.

    integer    a table declared with "type integer" will store 32bit integers
               which can represent a client identifier found in a request for
               instance.

    string     a table declared with "type string" will store substrings of up
               to <len> characters. If the string provided by the pattern
               extractor is larger than <len>, it will be truncated before
               being stored. During matching, at most <len> characters will be
               compared between the string in the table and the extracted
               pattern. When not specified, the string is automatically limited
               to 32 characters.

    binary     a table declared with "type binary" will store binary blocks
               of <len> bytes. If the block provided by the pattern
               extractor is larger than <len>, it will be truncated before
               being stored. If the block provided by the pattern extractor
               is shorter than <len>, it will be padded by 0. When not
               specified, the block is automatically limited to 32 bytes.

    <length>   is the maximum number of characters that will be stored in a
               "string" type table (See type "string" above). Or the number
               of bytes of the block in "binary" type table. Be careful when
               changing this parameter as memory usage will proportionally
               increase.

    <size>     is the maximum number of entries that can fit in the table. This
               value directly impacts memory usage. Count approximately
               50 bytes per entry, plus the size of a string if any. The size
               supports suffixes "k", "m", "g" for 2^10, 2^20 and 2^30 factors.

    [nopurge]  indicates that we refuse to purge older entries when the table
               is full. When not specified and the table is full when haproxy
               wants to store an entry in it, it will flush a few of the oldest
               entries in order to release some space for the new ones. This is
               most often the desired behaviour. In some specific cases, it
               be desirable to refuse new entries instead of purging the older
               ones. That may be the case when the amount of data to store is
               far above the hardware limits and we prefer not to offer access
               to new clients than to reject the ones already connected. When
               using this parameter, be sure to properly set the "expire"
               parameter (see below).

    <peersect> is the name of the peers section to use for replication. Entries
               which associate keys to server IDs are kept synchronized with
               the remote peers declared in this section. All entries are also
               automatically learned from the local peer (old process) during a
               soft restart.

               NOTE : peers can't be used in multi-process mode.

    <expire>   defines the maximum duration of an entry in the table since it
               was last created, refreshed or matched. The expiration delay is
               defined using the standard time format, similarly as the various
               timeouts. The maximum duration is slightly above 24 days. See
               section 2.2 for more information. If this delay is not specified,
               the session won't automatically expire, but older entries will
               be removed once full. Be sure not to use the "nopurge" parameter
               if not expiration delay is specified.

   <data_type> is used to store additional information in the stick-table. This
               may be used by ACLs in order to control various criteria related
               to the activity of the client matching the stick-table. For each
               item specified here, the size of each entry will be inflated so
               that the additional data can fit. Several data types may be
               stored with an entry. Multiple data types may be specified after
               the "store" keyword, as a comma-separated list. Alternatively,
               it is possible to repeat the "store" keyword followed by one or
               several data types. Except for the "server_id" type which is
               automatically detected and enabled, all data types must be
               explicitly declared to be stored. If an ACL references a data
               type which is not stored, the ACL will simply not match. Some
               data types require an argument which must be passed just after
               the type between parenthesis. See below for the supported data
               types and their arguments.

  The data types that can be stored with an entry are the following :
    - server_id : this is an integer which holds the numeric ID of the server a
      request was assigned to. It is used by the "stick match", "stick store",
      and "stick on" rules. It is automatically enabled when referenced.

    - gpc0 : first General Purpose Counter. It is a positive 32-bit integer
      integer which may be used for anything. Most of the time it will be used
      to put a special tag on some entries, for instance to note that a
      specific behaviour was detected and must be known for future matches.

    - conn_cnt : Connection Count. It is a positive 32-bit integer which counts
      the absolute number of connections received from clients which matched
      this entry. It does not mean the connections were accepted, just that
      they were received.

    - conn_cur : Current Connections. It is a positive 32-bit integer which
      stores the concurrent connection counts for the entry. It is incremented
      once an incoming connection matches the entry, and decremented once the
      connection leaves. That way it is possible to know at any time the exact
      number of concurrent connections for an entry.

    - conn_rate(<period>) : frequency counter (takes 12 bytes). It takes an
      integer parameter <period> which indicates in milliseconds the length
      of the period over which the average is measured. It reports the average
      incoming connection rate over that period, in connections per period. The
      result is an integer which can be matched using ACLs.

    - sess_cnt : Session Count. It is a positive 32-bit integer which counts
      the absolute number of sessions received from clients which matched this
      entry. A session is a connection that was accepted by the layer 4 rules.

    - sess_rate(<period>) : frequency counter (takes 12 bytes). It takes an
      integer parameter <period> which indicates in milliseconds the length
      of the period over which the average is measured. It reports the average
      incoming session rate over that period, in sessions per period. The
      result is an integer which can be matched using ACLs.

    - http_req_cnt : HTTP request Count. It is a positive 32-bit integer which
      counts the absolute number of HTTP requests received from clients which
      matched this entry. It does not matter whether they are valid requests or
      not. Note that this is different from sessions when keep-alive is used on
      the client side.

    - http_req_rate(<period>) : frequency counter (takes 12 bytes). It takes an
      integer parameter <period> which indicates in milliseconds the length
      of the period over which the average is measured. It reports the average
      HTTP request rate over that period, in requests per period. The result is
      an integer which can be matched using ACLs. It does not matter whether
      they are valid requests or not. Note that this is different from sessions
      when keep-alive is used on the client side.

    - http_err_cnt : HTTP Error Count. It is a positive 32-bit integer which
      counts the absolute number of HTTP requests errors induced by clients
      which matched this entry. Errors are counted on invalid and truncated
      requests, as well as on denied or tarpitted requests, and on failed
      authentications. If the server responds with 4xx, then the request is
      also counted as an error since it's an error triggered by the client
      (eg: vulnerability scan).

    - http_err_rate(<period>) : frequency counter (takes 12 bytes). It takes an
      integer parameter <period> which indicates in milliseconds the length
      of the period over which the average is measured. It reports the average
      HTTP request error rate over that period, in requests per period (see
      http_err_cnt above for what is accounted as an error). The result is an
      integer which can be matched using ACLs.

    - bytes_in_cnt : client to server byte count. It is a positive 64-bit
      integer which counts the cumulated amount of bytes received from clients
      which matched this entry. Headers are included in the count. This may be
      used to limit abuse of upload features on photo or video servers.

    - bytes_in_rate(<period>) : frequency counter (takes 12 bytes). It takes an
      integer parameter <period> which indicates in milliseconds the length
      of the period over which the average is measured. It reports the average
      incoming bytes rate over that period, in bytes per period. It may be used
      to detect users which upload too much and too fast. Warning: with large
      uploads, it is possible that the amount of uploaded data will be counted
      once upon termination, thus causing spikes in the average transfer speed
      instead of having a smooth one. This may partially be smoothed with
      "option contstats" though this is not perfect yet. Use of byte_in_cnt is
      recommended for better fairness.

    - bytes_out_cnt : server to client byte count. It is a positive 64-bit
      integer which counts the cumulated amount of bytes sent to clients which
      matched this entry. Headers are included in the count. This may be used
      to limit abuse of bots sucking the whole site.

    - bytes_out_rate(<period>) : frequency counter (takes 12 bytes). It takes
      an integer parameter <period> which indicates in milliseconds the length
      of the period over which the average is measured. It reports the average
      outgoing bytes rate over that period, in bytes per period. It may be used
      to detect users which download too much and too fast. Warning: with large
      transfers, it is possible that the amount of transferred data will be
      counted once upon termination, thus causing spikes in the average
      transfer speed instead of having a smooth one. This may partially be
      smoothed with "option contstats" though this is not perfect yet. Use of
      byte_out_cnt is recommended for better fairness.

  There is only one stick-table per proxy. At the moment of writing this doc,
  it does not seem useful to have multiple tables per proxy. If this happens
  to be required, simply create a dummy backend with a stick-table in it and
  reference it.

  It is important to understand that stickiness based on learning information
  has some limitations, including the fact that all learned associations are
  lost upon restart. In general it can be good as a complement but not always
  as an exclusive stickiness.

  Last, memory requirements may be important when storing many data types.
  Indeed, storing all indicators above at once in each entry requires 116 bytes
  per entry, or 116 MB for a 1-million entries table. This is definitely not
  something that can be ignored.

  Example:
        # Keep track of counters of up to 1 million IP addresses over 5 minutes
        # and store a general purpose counter and the average connection rate
        # computed over a sliding window of 30 seconds.
        stick-table type ip size 1m expire 5m store gpc0,conn_rate(30s)

  See also : "stick match", "stick on", "stick store-request", section 2.2
             about time format and section 7 about ACLs.


stick store-response <pattern> [table <table>] [{if | unless} <condition>]
  Define a request pattern used to create an entry in a stickiness table
  May be used in sections :   defaults | frontend | listen | backend
                                 no    |    no    |   yes  |   yes

  Arguments :
    <pattern>  is a pattern extraction rule as described in section 7.8. It
               describes what elements of the response or connection will
               be analysed, extracted and stored in the table once a
               server is selected.

    <table>    is an optional stickiness table name. If unspecified, the same
               backend's table is used. A stickiness table is declared using
               the "stick-table" statement.

    <cond>     is an optional storage condition. It makes it possible to store
               certain criteria only when some conditions are met (or not met).
               For instance, it could be used to store the SSL session ID only
               when the response is a SSL server hello.

  Some protocols or applications require complex stickiness rules and cannot
  always simply rely on cookies nor hashing. The "stick store-response"
  statement  describes a rule to decide what to extract from the response and
  when to do it, in order to store it into a stickiness table for further
  requests to match it using the "stick match" statement. Obviously the
  extracted part must make sense and have a chance to be matched in a further
  request. Storing an ID found in a header of a response makes sense.
  See section 7 for a complete list of possible patterns and transformation
  rules.

  The table has to be declared using the "stick-table" statement. It must be of
  a type compatible with the pattern. By default it is the one which is present
  in the same backend. It is possible to share a table with other backends by
  referencing it using the "table" keyword. If another table is referenced,
  the server's ID inside the backends are used. By default, all server IDs
  start at 1 in each backend, so the server ordering is enough. But in case of
  doubt, it is highly recommended to force server IDs using their "id" setting.

  It is possible to restrict the conditions where a "stick store-response"
  statement will apply, using "if" or "unless" followed by a condition. This
  condition will be evaluated while parsing the response, so any criteria can
  be used. See section 7 for ACL based conditions.

  There is no limit on the number of "stick store-response" statements, but
  there is a limit of 8 simultaneous stores per request or response. This
  makes it possible to store up to 8 criteria, all extracted from either the
  request or the response, regardless of the number of rules. Only the 8 first
  ones which match will be kept. Using this, it is possible to feed multiple
  tables at once in the hope to increase the chance to recognize a user on
  another protocol or access method.

  The table will contain the real server that processed the request.

  Example :
    # Learn SSL session ID from both request and response and create affinity.
    backend https
        mode tcp
        balance roundrobin
        # maximum SSL session ID length is 32 bytes.
        stick-table type binary len 32 size 30k expire 30m

        acl clienthello req_ssl_hello_type 1
        acl serverhello rep_ssl_hello_type 2

        # use tcp content accepts to detects ssl client and server hello.
        tcp-request inspect-delay 5s
        tcp-request content accept if clienthello

        # no timeout on response inspect delay by default.
        tcp-response content accept if serverhello

        # SSL session ID (SSLID) may be present on a client or server hello.
        # Its length is coded on 1 byte at offset 43 and its value starts
        # at offset 44.

        # Match and learn on request if client hello.
        stick on payload_lv(43,1) if clienthello

        # Learn on response if server hello.
        stick store-response payload_lv(43,1) if serverhello

        server s1 192.168.1.1:443
        server s2 192.168.1.1:443

  See also : "stick-table", "stick on", and section 7 about ACLs and pattern
             extraction.


tcp-request connection <action> [{if | unless} <condition>]
  Perform an action on an incoming connection depending on a layer 4 condition
  May be used in sections :   defaults | frontend | listen | backend
                                 no    |    yes   |   yes  |   no
  Arguments :
    <action>    defines the action to perform if the condition applies. Valid
                actions include : "accept", "reject", "track-sc1", "track-sc2".
                See below for more details.

    <condition> is a standard layer4-only ACL-based condition (see section 7).

  Immediately after acceptance of a new incoming connection, it is possible to
  evaluate some conditions to decide whether this connection must be accepted
  or dropped or have its counters tracked. Those conditions cannot make use of
  any data contents because the connection has not been read from yet, and the
  buffers are not yet allocated. This is used to selectively and very quickly
  accept or drop connections from various sources with a very low overhead. If
  some contents need to be inspected in order to take the decision, the
  "tcp-request content" statements must be used instead.

  The "tcp-request connection" rules are evaluated in their exact declaration
  order. If no rule matches or if there is no rule, the default action is to
  accept the incoming connection. There is no specific limit to the number of
  rules which may be inserted.

  Three types of actions are supported :
    - accept :
        accepts the connection if the condition is true (when used with "if")
        or false (when used with "unless"). The first such rule executed ends
        the rules evaluation.

    - reject :
        rejects the connection if the condition is true (when used with "if")
        or false (when used with "unless"). The first such rule executed ends
        the rules evaluation. Rejected connections do not even become a
        session, which is why they are accounted separately for in the stats,
        as "denied connections". They are not considered for the session
        rate-limit and are not logged either. The reason is that these rules
        should only be used to filter extremely high connection rates such as
        the ones encountered during a massive DDoS attack. Under these extreme
        conditions, the simple action of logging each event would make the
        system collapse and would considerably lower the filtering capacity. If
        logging is absolutely desired, then "tcp-request content" rules should
        be used instead.

    - { track-sc1 | track-sc2 } <key> [table <table>] :
        enables tracking of sticky counters from current connection. These
        rules do not stop evaluation and do not change default action. Two sets
        of counters may be simultaneously tracked by the same connection. The
        first "track-sc1" rule executed enables tracking of the counters of the
        specified table as the first set. The first "track-sc2" rule executed
        enables tracking of the counters of the specified table as the second
        set. It is a recommended practice to use the first set of counters for
        the per-frontend counters and the second set for the per-backend ones.

        These actions take one or two arguments :
          <key>   is mandatory, and defines the criterion the tracking key will
                  be derived from. At the moment, only "src" is supported. With
                  it, the key will be the connection's source IPv4 address.

         <table>  is an optional table to be used instead of the default one,
                  which is the stick-table declared in the current proxy. All
                  the counters for the matches and updates for the key will
                  then be performed in that table until the session ends.

        Once a "track-sc*" rule is executed, the key is looked up in the table
        and if it is not found, an entry is allocated for it. Then a pointer to
        that entry is kept during all the session's life, and this entry's
        counters are updated as often as possible, every time the session's
        counters are updated, and also systematically when the session ends.
        If the entry tracks concurrent connection counters, one connection is
        counted for as long as the entry is tracked, and the entry will not
        expire during that time. Tracking counters also provides a performance
        advantage over just checking the keys, because only one table lookup is
        performed for all ACL checks that make use of it.

  Note that the "if/unless" condition is optional. If no condition is set on
  the action, it is simply performed unconditionally. That can be useful for
  "track-sc*" actions as well as for changing the default action to a reject.

  Example: accept all connections from white-listed hosts, reject too fast
           connection without counting them, and track accepted connections.
           This results in connection rate being capped from abusive sources.

        tcp-request connection accept if { src -f /etc/haproxy/whitelist.lst }
        tcp-request connection reject if { src_conn_rate gt 10 }
        tcp-request connection track-sc1 src

  Example: accept all connections from white-listed hosts, count all other
           connections and reject too fast ones. This results in abusive ones
           being blocked as long as they don't slow down.

        tcp-request connection accept if { src -f /etc/haproxy/whitelist.lst }
        tcp-request connection track-sc1 src
        tcp-request connection reject if { sc1_conn_rate gt 10 }

  See section 7 about ACL usage.

  See also : "tcp-request content", "stick-table"


tcp-request content <action> [{if | unless} <condition>]
  Perform an action on a new session depending on a layer 4-7 condition
  May be used in sections :   defaults | frontend | listen | backend
                                 no    |    yes   |   yes  |   yes
  Arguments :
    <action>    defines the action to perform if the condition applies. Valid
                actions include : "accept", "reject", "track-sc1", "track-sc2".
                See "tcp-request connection" above for their signification.

    <condition> is a standard layer 4-7 ACL-based condition (see section 7).

  A request's contents can be analysed at an early stage of request processing
  called "TCP content inspection". During this stage, ACL-based rules are
  evaluated every time the request contents are updated, until either an
  "accept" or a "reject" rule matches, or the TCP request inspection delay
  expires with no matching rule.

  The first difference between these rules and "tcp-request connection" rules
  is that "tcp-request content" rules can make use of contents to take a
  decision. Most often, these decisions will consider a protocol recognition or
  validity. The second difference is that content-based rules can be used in
  both frontends and backends. In frontends, they will be evaluated upon new
  connections. In backends, they will be evaluated once a session is assigned
  a backend. This means that a single frontend connection may be evaluated
  several times by one or multiple backends when a session gets reassigned
  (for instance after a client-side HTTP keep-alive request).

  Content-based rules are evaluated in their exact declaration order. If no
  rule matches or if there is no rule, the default action is to accept the
  contents. There is no specific limit to the number of rules which may be
  inserted.

  Three types of actions are supported :
    - accept :
    - reject :
    - { track-sc1 | track-sc2 } <key> [table <table>]

  They have the same meaning as their counter-parts in "tcp-request connection"
  so please refer to that section for a complete description.

  Also, it is worth noting that if sticky counters are tracked from a rule
  defined in a backend, this tracking will automatically end when the session
  releases the backend. That allows per-backend counter tracking even in case
  of HTTP keep-alive requests when the backend changes. While there is nothing
  mandatory about it, it is recommended to use the track-sc1 pointer to track
  per-frontend counters and track-sc2 to track per-backend counters.

  Note that the "if/unless" condition is optional. If no condition is set on
  the action, it is simply performed unconditionally. That can be useful for
  "track-sc*" actions as well as for changing the default action to a reject.

  It is perfectly possible to match layer 7 contents with "tcp-request content"
  rules, since HTTP-specific ACL matches are able to preliminarily parse the
  contents of a buffer before extracting the required data. If the buffered
  contents do not parse as a valid HTTP message, then the ACL does not match.
  The parser which is involved there is exactly the same as for all other HTTP
  processing, so there is no risk of parsing something differently.

  Example:
        # Accept HTTP requests containing a Host header saying "example.com"
        # and reject everything else.
        acl is_host_com hdr(Host) -i example.com
        tcp-request inspect-delay 30s
        tcp-request content accept if is_host_com
        tcp-request content reject

  Example:
        # reject SMTP connection if client speaks first
        tcp-request inspect-delay 30s
        acl content_present req_len gt 0
        tcp-request content reject if content_present

        # Forward HTTPS connection only if client speaks
        tcp-request inspect-delay 30s
        acl content_present req_len gt 0
        tcp-request content accept if content_present
        tcp-request content reject

  Example: track per-frontend and per-backend counters, block abusers at the
           frontend when the backend detects abuse.

        frontend http
            # Use General Purpose Couter 0 in SC1 as a global abuse counter
            # protecting all our sites
            stick-table type ip size 1m expire 5m store gpc0
            tcp-request connection track-sc1 src
            tcp-request connection reject if { sc1_get_gpc0 gt 0 }
            ...
            use_backend http_dynamic if { path_end .php }

        backend http_dynamic
            # if a source makes too fast requests to this dynamic site (tracked
            # by SC2), block it globally in the frontend.
            stick-table type ip size 1m expire 5m store http_req_rate(10s)
            acl click_too_fast sc2_http_req_rate gt 10
            acl mark_as_abuser sc1_inc_gpc0
            tcp-request content track-sc2 src
            tcp-request content reject if click_too_fast mark_as_abuser

  See section 7 about ACL usage.

  See also : "tcp-request connection", "tcp-request inspect-delay"


tcp-request inspect-delay <timeout>
  Set the maximum allowed time to wait for data during content inspection
  May be used in sections :   defaults | frontend | listen | backend
                                 no    |    yes   |   yes  |   yes
  Arguments :
    <timeout> is the timeout value specified in milliseconds by default, but
              can be in any other unit if the number is suffixed by the unit,
              as explained at the top of this document.

  People using haproxy primarily as a TCP relay are often worried about the
  risk of passing any type of protocol to a server without any analysis. In
  order to be able to analyze the request contents, we must first withhold
  the data then analyze them. This statement simply enables withholding of
  data for at most the specified amount of time.

  TCP content inspection applies very early when a connection reaches a
  frontend, then very early when the connection is forwarded to a backend. This
  means that a connection may experience a first delay in the frontend and a
  second delay in the backend if both have tcp-request rules.

  Note that when performing content inspection, haproxy will evaluate the whole
  rules for every new chunk which gets in, taking into account the fact that
  those data are partial. If no rule matches before the aforementioned delay,
  a last check is performed upon expiration, this time considering that the
  contents are definitive. If no delay is set, haproxy will not wait at all
  and will immediately apply a verdict based on the available information.
  Obviously this is unlikely to be very useful and might even be racy, so such
  setups are not recommended.

  As soon as a rule matches, the request is released and continues as usual. If
  the timeout is reached and no rule matches, the default policy will be to let
  it pass through unaffected.

  For most protocols, it is enough to set it to a few seconds, as most clients
  send the full request immediately upon connection. Add 3 or more seconds to
  cover TCP retransmits but that's all. For some protocols, it may make sense
  to use large values, for instance to ensure that the client never talks
  before the server (eg: SMTP), or to wait for a client to talk before passing
  data to the server (eg: SSL). Note that the client timeout must cover at
  least the inspection delay, otherwise it will expire first. If the client
  closes the connection or if the buffer is full, the delay immediately expires
  since the contents will not be able to change anymore.

  See also : "tcp-request content accept", "tcp-request content reject",
             "timeout client".


tcp-response content <action> [{if | unless} <condition>]
  Perform an action on a session response depending on a layer 4-7 condition
  May be used in sections :   defaults | frontend | listen | backend
                                 no    |    no    |   yes  |   yes
  Arguments :
    <action>    defines the action to perform if the condition applies. Valid
                actions include : "accept", "reject".
                See "tcp-request connection" above for their signification.

    <condition> is a standard layer 4-7 ACL-based condition (see section 7).

  Response contents can be analysed at an early stage of response processing
  called "TCP content inspection". During this stage, ACL-based rules are
  evaluated every time the response contents are updated, until either an
  "accept" or a "reject" rule matches, or a TCP response inspection delay is
  set and expires with no matching rule.

  Most often, these decisions will consider a protocol recognition or validity.

  Content-based rules are evaluated in their exact declaration order. If no
  rule matches or if there is no rule, the default action is to accept the
  contents. There is no specific limit to the number of rules which may be
  inserted.

  Two types of actions are supported :
    - accept :
        accepts the response if the condition is true (when used with "if")
        or false (when used with "unless"). The first such rule executed ends
        the rules evaluation.

    - reject :
        rejects the response if the condition is true (when used with "if")
        or false (when used with "unless"). The first such rule executed ends
        the rules evaluation. Rejected session are immediatly closed.

  Note that the "if/unless" condition is optional. If no condition is set on
  the action, it is simply performed unconditionally. That can be useful for
  for changing the default action to a reject.

  It is perfectly possible to match layer 7 contents with "tcp-reponse content"
  rules, but then it is important to ensure that a full response has been
  buffered, otherwise no contents will match. In order to achieve this, the
  best solution involves detecting the HTTP protocol during the inspection
  period.

  See section 7 about ACL usage.

  See also : "tcp-request content", "tcp-response inspect-delay"


tcp-response inspect-delay <timeout>
  Set the maximum allowed time to wait for a response during content inspection
  May be used in sections :   defaults | frontend | listen | backend
                                 no    |    no    |   yes  |   yes
  Arguments :
    <timeout> is the timeout value specified in milliseconds by default, but
              can be in any other unit if the number is suffixed by the unit,
              as explained at the top of this document.

  See also : "tcp-response content", "tcp-request inspect-delay".


timeout check <timeout>
  Set additional check timeout, but only after a connection has been already
  established.

  May be used in sections:    defaults | frontend | listen | backend
                                 yes   |    no    |   yes  |   yes
  Arguments:
    <timeout> is the timeout value specified in milliseconds by default, but
              can be in any other unit if the number is suffixed by the unit,
              as explained at the top of this document.

  If set, haproxy uses min("timeout connect", "inter") as a connect timeout
  for check and "timeout check" as an additional read timeout. The "min" is
  used so that people running with *very* long "timeout connect" (eg. those
  who needed this due to the queue or tarpit) do not slow down their checks.
  (Please also note that there is no valid reason to have such long connect
  timeouts, because "timeout queue" and "timeout tarpit" can always be used to
  avoid that).

  If "timeout check" is not set haproxy uses "inter" for complete check
  timeout (connect + read) exactly like all <1.3.15 version.

  In most cases check request is much simpler and faster to handle than normal
  requests and people may want to kick out laggy servers so this timeout should
  be smaller than "timeout server".

  This parameter is specific to backends, but can be specified once for all in
  "defaults" sections. This is in fact one of the easiest solutions not to
  forget about it.

  See also: "timeout connect", "timeout queue", "timeout server",
            "timeout tarpit".


timeout client <timeout>
timeout clitimeout <timeout> (deprecated)
  Set the maximum inactivity time on the client side.
  May be used in sections :   defaults | frontend | listen | backend
                                 yes   |    yes   |   yes  |   no
  Arguments :
    <timeout> is the timeout value specified in milliseconds by default, but
              can be in any other unit if the number is suffixed by the unit,
              as explained at the top of this document.

  The inactivity timeout applies when the client is expected to acknowledge or
  send data. In HTTP mode, this timeout is particularly important to consider
  during the first phase, when the client sends the request, and during the
  response while it is reading data sent by the server. The value is specified
  in milliseconds by default, but can be in any other unit if the number is
  suffixed by the unit, as specified at the top of this document. In TCP mode
  (and to a lesser extent, in HTTP mode), it is highly recommended that the
  client timeout remains equal to the server timeout in order to avoid complex
  situations to debug. It is a good practice to cover one or several TCP packet
  losses by specifying timeouts that are slightly above multiples of 3 seconds
  (eg: 4 or 5 seconds). If some long-lived sessions are mixed with short-lived
  sessions (eg: WebSocket and HTTP), it's worth considering "timeout tunnel",
  which overrides "timeout client" and "timeout server" for tunnels.

  This parameter is specific to frontends, but can be specified once for all in
  "defaults" sections. This is in fact one of the easiest solutions not to
  forget about it. An unspecified timeout results in an infinite timeout, which
  is not recommended. Such a usage is accepted and works but reports a warning
  during startup because it may results in accumulation of expired sessions in
  the system if the system's timeouts are not configured either.

  This parameter replaces the old, deprecated "clitimeout". It is recommended
  to use it to write new configurations. The form "timeout clitimeout" is
  provided only by backwards compatibility but its use is strongly discouraged.

  See also : "clitimeout", "timeout server", "timeout tunnel".


timeout connect <timeout>
timeout contimeout <timeout> (deprecated)
  Set the maximum time to wait for a connection attempt to a server to succeed.
  May be used in sections :   defaults | frontend | listen | backend
                                 yes   |    no    |   yes  |   yes
  Arguments :
    <timeout> is the timeout value specified in milliseconds by default, but
              can be in any other unit if the number is suffixed by the unit,
              as explained at the top of this document.

  If the server is located on the same LAN as haproxy, the connection should be
  immediate (less than a few milliseconds). Anyway, it is a good practice to
  cover one or several TCP packet losses by specifying timeouts that are
  slightly above multiples of 3 seconds (eg: 4 or 5 seconds). By default, the
  connect timeout also presets both queue and tarpit timeouts to the same value
  if these have not been specified.

  This parameter is specific to backends, but can be specified once for all in
  "defaults" sections. This is in fact one of the easiest solutions not to
  forget about it. An unspecified timeout results in an infinite timeout, which
  is not recommended. Such a usage is accepted and works but reports a warning
  during startup because it may results in accumulation of failed sessions in
  the system if the system's timeouts are not configured either.

  This parameter replaces the old, deprecated "contimeout". It is recommended
  to use it to write new configurations. The form "timeout contimeout" is
  provided only by backwards compatibility but its use is strongly discouraged.

  See also: "timeout check", "timeout queue", "timeout server", "contimeout",
            "timeout tarpit".


timeout http-keep-alive <timeout>
  Set the maximum allowed time to wait for a new HTTP request to appear
  May be used in sections :   defaults | frontend | listen | backend
                                 yes   |    yes   |   yes  |   yes
  Arguments :
    <timeout> is the timeout value specified in milliseconds by default, but
              can be in any other unit if the number is suffixed by the unit,
              as explained at the top of this document.

  By default, the time to wait for a new request in case of keep-alive is set
  by "timeout http-request". However this is not always convenient because some
  people want very short keep-alive timeouts in order to release connections
  faster, and others prefer to have larger ones but still have short timeouts
  once the request has started to present itself.

  The "http-keep-alive" timeout covers these needs. It will define how long to
  wait for a new HTTP request to start coming after a response was sent. Once
  the first byte of request has been seen, the "http-request" timeout is used
  to wait for the complete request to come. Note that empty lines prior to a
  new request do not refresh the timeout and are not counted as a new request.

  There is also another difference between the two timeouts : when a connection
  expires during timeout http-keep-alive, no error is returned, the connection
  just closes. If the connection expires in "http-request" while waiting for a
  connection to complete, a HTTP 408 error is returned.

  In general it is optimal to set this value to a few tens to hundreds of
  milliseconds, to allow users to fetch all objects of a page at once but
  without waiting for further clicks. Also, if set to a very small value (eg:
  1 millisecond) it will probably only accept pipelined requests but not the
  non-pipelined ones. It may be a nice trade-off for very large sites running
  with tens to hundreds of thousands of clients.

  If this parameter is not set, the "http-request" timeout applies, and if both
  are not set, "timeout client" still applies at the lower level. It should be
  set in the frontend to take effect, unless the frontend is in TCP mode, in
  which case the HTTP backend's timeout will be used.

  See also : "timeout http-request", "timeout client".


timeout http-request <timeout>
  Set the maximum allowed time to wait for a complete HTTP request
  May be used in sections :   defaults | frontend | listen | backend
                                 yes   |    yes   |   yes  |   yes
  Arguments :
    <timeout> is the timeout value specified in milliseconds by default, but
              can be in any other unit if the number is suffixed by the unit,
              as explained at the top of this document.

  In order to offer DoS protection, it may be required to lower the maximum
  accepted time to receive a complete HTTP request without affecting the client
  timeout. This helps protecting against established connections on which
  nothing is sent. The client timeout cannot offer a good protection against
  this abuse because it is an inactivity timeout, which means that if the
  attacker sends one character every now and then, the timeout will not
  trigger. With the HTTP request timeout, no matter what speed the client
  types, the request will be aborted if it does not complete in time.

  Note that this timeout only applies to the header part of the request, and
  not to any data. As soon as the empty line is received, this timeout is not
  used anymore. It is used again on keep-alive connections to wait for a second
  request if "timeout http-keep-alive" is not set.

  Generally it is enough to set it to a few seconds, as most clients send the
  full request immediately upon connection. Add 3 or more seconds to cover TCP
  retransmits but that's all. Setting it to very low values (eg: 50 ms) will
  generally work on local networks as long as there are no packet losses. This
  will prevent people from sending bare HTTP requests using telnet.

  If this parameter is not set, the client timeout still applies between each
  chunk of the incoming request. It should be set in the frontend to take
  effect, unless the frontend is in TCP mode, in which case the HTTP backend's
  timeout will be used.

  See also : "timeout http-keep-alive", "timeout client".


timeout queue <timeout>
  Set the maximum time to wait in the queue for a connection slot to be free
  May be used in sections :   defaults | frontend | listen | backend
                                 yes   |    no    |   yes  |   yes
  Arguments :
    <timeout> is the timeout value specified in milliseconds by default, but
              can be in any other unit if the number is suffixed by the unit,
              as explained at the top of this document.

  When a server's maxconn is reached, connections are left pending in a queue
  which may be server-specific or global to the backend. In order not to wait
  indefinitely, a timeout is applied to requests pending in the queue. If the
  timeout is reached, it is considered that the request will almost never be
  served, so it is dropped and a 503 error is returned to the client.

  The "timeout queue" statement allows to fix the maximum time for a request to
  be left pending in a queue. If unspecified, the same value as the backend's
  connection timeout ("timeout connect") is used, for backwards compatibility
  with older versions with no "timeout queue" parameter.

  See also : "timeout connect", "contimeout".


timeout server <timeout>
timeout srvtimeout <timeout> (deprecated)
  Set the maximum inactivity time on the server side.
  May be used in sections :   defaults | frontend | listen | backend
                                 yes   |    no    |   yes  |   yes
  Arguments :
    <timeout> is the timeout value specified in milliseconds by default, but
              can be in any other unit if the number is suffixed by the unit,
              as explained at the top of this document.

  The inactivity timeout applies when the server is expected to acknowledge or
  send data. In HTTP mode, this timeout is particularly important to consider
  during the first phase of the server's response, when it has to send the
  headers, as it directly represents the server's processing time for the
  request. To find out what value to put there, it's often good to start with
  what would be considered as unacceptable response times, then check the logs
  to observe the response time distribution, and adjust the value accordingly.

  The value is specified in milliseconds by default, but can be in any other
  unit if the number is suffixed by the unit, as specified at the top of this
  document. In TCP mode (and to a lesser extent, in HTTP mode), it is highly
  recommended that the client timeout remains equal to the server timeout in
  order to avoid complex situations to debug. Whatever the expected server
  response times, it is a good practice to cover at least one or several TCP
  packet losses by specifying timeouts that are slightly above multiples of 3
  seconds (eg: 4 or 5 seconds minimum). If some long-lived sessions are mixed
  with short-lived sessions (eg: WebSocket and HTTP), it's worth considering
  "timeout tunnel", which overrides "timeout client" and "timeout server" for
  tunnels.

  This parameter is specific to backends, but can be specified once for all in
  "defaults" sections. This is in fact one of the easiest solutions not to
  forget about it. An unspecified timeout results in an infinite timeout, which
  is not recommended. Such a usage is accepted and works but reports a warning
  during startup because it may results in accumulation of expired sessions in
  the system if the system's timeouts are not configured either.

  This parameter replaces the old, deprecated "srvtimeout". It is recommended
  to use it to write new configurations. The form "timeout srvtimeout" is
  provided only by backwards compatibility but its use is strongly discouraged.

  See also : "srvtimeout", "timeout client" and "timeout tunnel".


timeout tarpit <timeout>
  Set the duration for which tarpitted connections will be maintained
  May be used in sections :   defaults | frontend | listen | backend
                                 yes   |    yes   |   yes  |   yes
  Arguments :
    <timeout> is the tarpit duration specified in milliseconds by default, but
              can be in any other unit if the number is suffixed by the unit,
              as explained at the top of this document.

  When a connection is tarpitted using "reqtarpit", it is maintained open with
  no activity for a certain amount of time, then closed. "timeout tarpit"
  defines how long it will be maintained open.

  The value is specified in milliseconds by default, but can be in any other
  unit if the number is suffixed by the unit, as specified at the top of this
  document. If unspecified, the same value as the backend's connection timeout
  ("timeout connect") is used, for backwards compatibility with older versions
  with no "timeout tarpit" parameter.

  See also : "timeout connect", "contimeout".


timeout tunnel <timeout>
  Set the maximum inactivity time on the client and server side for tunnels.
  May be used in sections :   defaults | frontend | listen | backend
                                 yes   |    no    |   yes  |   yes
  Arguments :
    <timeout> is the timeout value specified in milliseconds by default, but
              can be in any other unit if the number is suffixed by the unit,
              as explained at the top of this document.

  The tunnel timeout applies when a bidirectionnal connection is established
  between a client and a server, and the connection remains inactive in both
  directions. This timeout supersedes both the client and server timeouts once
  the connection becomes a tunnel. In TCP, this timeout is used as soon as no
  analyser remains attached to either connection (eg: tcp content rules are
  accepted). In HTTP, this timeout is used when a connection is upgraded (eg:
  when switching to the WebSocket protocol, or forwarding a CONNECT request
  to a proxy), or after the first response when no keepalive/close option is
  specified.

  The value is specified in milliseconds by default, but can be in any other
  unit if the number is suffixed by the unit, as specified at the top of this
  document. Whatever the expected normal idle time, it is a good practice to
  cover at least one or several TCP packet losses by specifying timeouts that
  are slightly above multiples of 3 seconds (eg: 4 or 5 seconds minimum).

  This parameter is specific to backends, but can be specified once for all in
  "defaults" sections. This is in fact one of the easiest solutions not to
  forget about it.

  Example :
        defaults http
            option http-server-close
            timeout connect 5s
            timeout client 30s
            timeout client 30s
            timeout server 30s
            timeout tunnel  1h    # timeout to use with WebSocket and CONNECT

  See also : "timeout client", "timeout server".


transparent (deprecated)
  Enable client-side transparent proxying
  May be used in sections :   defaults | frontend | listen | backend
                                 yes   |    no    |   yes  |   yes
  Arguments : none

  This keyword was introduced in order to provide layer 7 persistence to layer
  3 load balancers. The idea is to use the OS's ability to redirect an incoming
  connection for a remote address to a local process (here HAProxy), and let
  this process know what address was initially requested. When this option is
  used, sessions without cookies will be forwarded to the original destination
  IP address of the incoming request (which should match that of another
  equipment), while requests with cookies will still be forwarded to the
  appropriate server.

  The "transparent" keyword is deprecated, use "option transparent" instead.

  Note that contrary to a common belief, this option does NOT make HAProxy
  present the client's IP to the server when establishing the connection.

  See also: "option transparent"

unique-id-format <string>
  Generate a unique ID for each request.
  May be used in sections :   defaults | frontend | listen | backend
                                  yes  |    yes   |   yes  |   no
  Arguments :
    <string>   is a log-format string.

  This keyword creates a ID for each request using the custom log format. A
  unique ID is useful to trace a request passing through many components of
  a complex infrastructure. The newly created ID may also be logged using the
  %ID tag the log-format string.

  The format should be composed from elements that are guaranteed to be
  unique when combined together. For instance, if multiple haproxy instances
  are involved, it might be important to include the node name. It is often
  needed to log the incoming connection's source and destination addresses
  and ports. Note that since multiple requests may be performed over the same
  connection, including a request counter may help differentiate them.
  Similarly, a timestamp may protect against a rollover of the counter.
  Logging the process ID will avoid collisions after a service restart.

  It is recommended to use hexadecimal notation for many fields since it
  makes them more compact and saves space in logs.

  Example:

        unique-id-format %{+X}o\ %Ci:%Cp_%Fi:%Fp_%Ts_%rt:%pid

        will generate:

               7F000001:8296_7F00001E:1F90_4F7B0A69_0003:790A

  See also: "unique-id-header"

unique-id-header <name>
  Add a unique ID header in the HTTP request.
  May be used in sections :   defaults | frontend | listen | backend
                                  yes  |    yes   |   yes  |   no
  Arguments :
    <name>   is the name of the header.

  Add a unique-id header in the HTTP request sent to the server, using the
  unique-id-format. It can't work if the unique-id-format doesn't exist.

  Example:

        unique-id-format %{+X}o\ %Ci:%Cp_%Fi:%Fp_%Ts_%rt:%pid
        unique-id-header X-Unique-ID

        will generate:

           X-Unique-ID: 7F000001:8296_7F00001E:1F90_4F7B0A69_0003:790A

    See also: "unique-id-format"

use_backend <backend> if <condition>
use_backend <backend> unless <condition>
  Switch to a specific backend if/unless an ACL-based condition is matched.
  May be used in sections :   defaults | frontend | listen | backend
                                  no   |    yes   |   yes  |   no
  Arguments :
    <backend>   is the name of a valid backend or "listen" section.

    <condition> is a condition composed of ACLs, as described in section 7.

  When doing content-switching, connections arrive on a frontend and are then
  dispatched to various backends depending on a number of conditions. The
  relation between the conditions and the backends is described with the
  "use_backend" keyword. While it is normally used with HTTP processing, it can
  also be used in pure TCP, either without content using stateless ACLs (eg:
  source address validation) or combined with a "tcp-request" rule to wait for
  some payload.

  There may be as many "use_backend" rules as desired. All of these rules are
  evaluated in their declaration order, and the first one which matches will
  assign the backend.

  In the first form, the backend will be used if the condition is met. In the
  second form, the backend will be used if the condition is not met. If no
  condition is valid, the backend defined with "default_backend" will be used.
  If no default backend is defined, either the servers in the same section are
  used (in case of a "listen" section) or, in case of a frontend, no server is
  used and a 503 service unavailable response is returned.

  Note that it is possible to switch from a TCP frontend to an HTTP backend. In
  this case, either the frontend has already checked that the protocol is HTTP,
  and backend processing will immediately follow, or the backend will wait for
  a complete HTTP request to get in. This feature is useful when a frontend
  must decode several protocols on a unique port, one of them being HTTP.

  See also: "default_backend", "tcp-request", and section 7 about ACLs.


use-server <server> if <condition>
use-server <server> unless <condition>
  Only use a specific server if/unless an ACL-based condition is matched.
  May be used in sections :   defaults | frontend | listen | backend
                                  no   |    no    |   yes  |   yes
  Arguments :
    <server>    is the name of a valid server in the same backend section.

    <condition> is a condition composed of ACLs, as described in section 7.

  By default, connections which arrive to a backend are load-balanced across
  the available servers according to the configured algorithm, unless a
  persistence mechanism such as a cookie is used and found in the request.

  Sometimes it is desirable to forward a particular request to a specific
  server without having to declare a dedicated backend for this server. This
  can be achieved using the "use-server" rules. These rules are evaluated after
  the "redirect" rules and before evaluating cookies, and they have precedence
  on them. There may be as many "use-server" rules as desired. All of these
  rules are evaluated in their declaration order, and the first one which
  matches will assign the server.

  If a rule designates a server which is down, and "option persist" is not used
  and no force-persist rule was validated, it is ignored and evaluation goes on
  with the next rules until one matches.

  In the first form, the server will be used if the condition is met. In the
  second form, the server will be used if the condition is not met. If no
  condition is valid, the processing continues and the server will be assigned
  according to other persistence mechanisms.

  Note that even if a rule is matched, cookie processing is still performed but
  does not assign the server. This allows prefixed cookies to have their prefix
  stripped.

  The "use-server" statement works both in HTTP and TCP mode. This makes it
  suitable for use with content-based inspection. For instance, a server could
  be selected in a farm according to the TLS SNI field. And if these servers
  have their weight set to zero, they will not be used for other traffic.

  Example :
     # intercept incoming TLS requests based on the SNI field
     use-server www if { req_ssl_sni -i www.example.com }
     server     www 192.168.0.1:443 weight 0
     use-server mail if { req_ssl_sni -i mail.example.com }
     server     mail 192.168.0.1:587 weight 0
     use-server imap if { req_ssl_sni -i imap.example.com }
     server     mail 192.168.0.1:993 weight 0
     # all the rest is forwarded to this server
     server  default 192.168.0.2:443 check

  See also: "use_backend", serction 5 about server and section 7 about ACLs.
