\begin{verbatim}

SECTIONING

One digit section  = \chapter
Two digit section  = \section
Tree digit section = \subsection

Lesser section, mostly description of option line = \subsubsection
(no digit, but can be referenced and added to TOC)
\subsubsection*{} if no reference required.

Even lesser part of text inside \subsubsection with bold header and no newline after
Group of attributes for example = \paragraph{}

MACROS
Lets use most of your macros, but rename some of them

\httphdr     - HTTP header
\httpmethod  - HTTP method
\httpstatus  - HTTP status
\httpcode    - HTTP code

\site        - Link to site
\rfc         - Link to RFC. Uses \site to make link.
\char        - special character
\cmd         - command line or arguments

\kw          - keyword with hyperlink target (don't use \option, make it \kw too)
\kwl         - keyword with link to keyword target (in See also for example)
\kwi         - keyword item in keywords environment
\kwp         - parameter of keyword
\kwop        - optional parameter of keyword

\note        - nice note block

ENVIRONMENTS

\begin{keywords}  block of keyword descriptions (uses \begin{description})
\begin{example}   block of config file example code

\begin{itemize}
\item[-] list
\item[-] with
\item[-] dash
\item[-] bullets
\end{itemzie}

TABLE

\head - table header
\yes  - 'yes' field
\no   - 'no'  field
\dflb - Can be used in Defaults|Frontend|Listen|Backend block

\end{verbatim}